% Sun Dec  2 12:35:11 2012 - REORGANIZE the charts to look more like
% https://forum.learnnavi.org/navi-lernen/das-navi-zahlensystem/
\nchapter{Zahlen}

\noindent Die Na'vi-Sprache hat ein \textit{Oktal}-System, d. h. ein Zahlensystem zur Basis acht, wie einige wenige menschliche Sprachen.\footnote{Offenbar ein Ergebnis dessen, dass nicht die Finger, sondern die Zwischenräume gezählt werden.} Anstatt Zahlen in der Form $(m \times 10) + n$ zu berechnen (wie in $(4 \times 10) + 2 = 42_{10}$, \E{zweiundvierzig}), werden die Zahlen aus $(m \times 8) + n$ berechnet (wie in $(5 \times 8) + 2 = 52_8$, \N{mrrvomun}, $42_{10}$).

\section{Kardinalzahlen}
\index{Zahlen!Kardinalzahlen}

\subsection{Die Einerstelle} Die unabhängigen Formen der Ziffern von eins bis acht sind:

\begin{center}
	\begin{tabular}{ll}
		1 & \N{'aw} \\
		2 & \N{\ACC{mu}ne} \\
		3 & \N{pxey} \\
		4 & \N{tsìng} \\
	\end{tabular}
	\hskip 3em
	\begin{tabular}{ll}
		5 & \N{mrr} \\
		6 & \N{\ACC{pu}kap} \\
		7 & \N{\ACC{ki}nä} \\
		8 & \N{vol} \\
	\end{tabular}
\end{center}

\subsection{Potenzen von acht} Anstelle von ``Zehnern'' hat Na'vi ``Achter'':

\begin{center}
	\begin{tabular}{ll}
		8 (1 $\times$ 8) & \N{vol} \\
		16 (2 $\times$ 8) & \N{\ACC{me}vol} \\
		24 (3 $\times$ 8) & \N{\ACC{pxe}vol} \\
		32 (4 $\times$ 8) & \N{\ACC{tsì}vol} \\
	\end{tabular}
	\hskip 3em
	\begin{tabular}{ll}
		40 (5 $\times$ 8) & \N{\ACC{mrr}vol} \\
		48 (6 $\times$ 8) & \N{\ACC{pu}vol} \\
		56 (7 $\times$ 8) & \N{\ACC{ki}vol} \\
		64 (8 $\times$ 8) & \N{zam} \\
	\end{tabular}
\end{center}

\noindent Höhere Potenzen von acht sind \N{\ACC{vo}zam} (512, oktal 1000) und \N{\ACC{za}zam} (4096, oktal 10000).

\subsection{Zusammengesetzte Formen} Werden sie mit Achterpotenzen kombiniert, so werden die Zahlwörter einsilbig und lösen Lenition aus, wenn möglich: \label{numbers:dependent} \index{Lenition!Zahlen}

\begin{center}
	\begin{tabular}{ll}
		1 & \N{(l)-aw} \\
		2 & \N{-mun} \\
		3 & \N{-pey} \\
		4 & \N{-sìng} \\
	\end{tabular}
	\hskip 3em
	\begin{tabular}{ll}
		5 & \N{-mrr} \\
		6 & \N{-fu} \\
		7 & \N{-hin} \\
	\end{tabular}
\end{center}

\subsubsection{} Alle abhängigen Formen außer ``eins'', \N{(l)-aw}, verdrängen das finale \N{-l} der Formen ``acht''. In ähnlicher Weise entfällt das finale \N{-m} in den Formen \N{zam, vozam,} und \N{zazam} vor allen Formen außer ``eins'', \N{zamaw,} aber \N{za\ACC{mun}, za\ACC{pey}}, usw.

\subsubsection{} Die angehängten abhängigen Formen nehmen den Wortakzent auf. Kombiniert mit \N{vol} \E{acht}:

\begin{center}
	\begin{tabular}{ll}
		9 (1$\times$8 $+$ 1) & \N{vo\ACC{law}} \\
		10 (1$\times$8 $+$ 2) & \N{vo\ACC{mun}} \\
		11 (1$\times$8 $+$ 3) & \N{vo\ACC{pey}} \\
		12 (1$\times$8 $+$ 4) & \N{vo\ACC{sìng}} \\
	\end{tabular}
	\hskip 3em
	\begin{tabular}{ll}
		13 (1$\times$8 $+$ 5) & \N{vo\ACC{mrr}} \\
		14 (1$\times$8 $+$ 6) & \N{vo\ACC{fu}} \\
		15 (1$\times$8 $+$ 7) & \N{vo\ACC{hin}} \\
		16 (2$\times$8 $+$ 0) & \N{\ACC{me}vol} \\
	\end{tabular}
\end{center}

\noindent Das Muster wird auf diese Weise mit \N{\ACC{me}vol} fortgesetzt:
\N{mevo\ACC{law}}, \N{mevo\ACC{mun}}, \N{mevo\ACC{pey}}, etc.

Nach \N{zam} lautet die Zählung: \N{zam, za\ACC{maw}, za\ACC{mun}, za\ACC{pey}, za\ACC{sìng}, za\ACC{mrr}, za\ACC{fu}, za\ACC{hin}, \ACC{za}vol}, und dann weiter als \N{zavo\ACC{law}}, usw. Z. B. ist die Oktalzahl 211 \N{mezavolaw}.
\NTeri{1/4/2014}{https://naviteri.org/2014/03/value-and-worth/\#comment-2678}
\LNForum{27/1/2021}{https://forum.learnnavi.org/index.php?msg=674844}

\section{Ordinalzahlen}

\subsection{Suffix -ve} Die Ordinalzahlen werden mit dem Suffix \N{-ve} gebildet, das den Wortakzent nicht verändert, aber einige Zahlwortstämme. \index{-ve@\textbf{-ve}}
\index{Zahlen!Ordinalzahlen}

\begin{center}
	\begin{tabular}{rll}
		Ordinalzahl & Unabhängig & Abhängig \\
		\hline
		erste/r/s & \N{\ACC{'aw}ve} & \N{(l)-\ACC{aw}ve} \\
		zweite/r/s & \N{\ACC{mu}ve} & \N{-\ACC{mu}ve} \\
		dritte/r/s & \N{\ACC{pxey}ve} & \N{-\ACC{pey}ve} \\
		vierte/r/s & \N{\ACC{tsì}ve} & \N{-\ACC{sì}ve} \\
		fünfte/r/s & \N{\ACC{mrr}ve} & \N{-\ACC{mrr}ve} \\
		sechste/r/s & \N{\ACC{pu}ve} & \N{-\ACC{fu}ve} \\
		siebte/r/s & \N{\ACC{ki}ve} & \N{-\ACC{hi}ve} \\
		
	\end{tabular}
	\hskip2em
	\begin{tabular}{rll}
		\\
		achte/r/s & \N{\ACC{vol}ve} & \N{-\ACC{vol}ve} \\
		64. & \N{\ACC{za}ve} & \N{-\ACC{za}ve} \\
		512. & \N{vo\ACC{za}ve} & \N{-vo\ACC{za}ve} \\
		4096. & \N{za\ACC{za}ve} & \N{-za\ACC{za}ve} \\ 
	\end{tabular}
\end{center}

\subsubsection{} Ordnungszahlen werden wie Adjektive behandelt und nehmen \N{-a-} an, wenn sie attributiv verwendet werden (\horenref{morph:adj-attr}), wie in \Npawl{mrrvea ikran} \E{fünfter Ikran}.
\LNForum{27/1/2021}{https://forum.learnnavi.org/index.php?msg=674844}

\subsubsection{} Alle Ordnungszahlen können frei mit \N{nì-} kombiniert werden, um Adverbien zu bilden, \N{nì'awve} \E{zuerst, erstens}, \N{nìmuve} \E{zweitens}, usw.
\LNWiki{5/6/2013}{https://wiki.learnnavi.org/Canon/2013\%23Ordinals_.26_nume}


\section{Brüche}
\index{Zahlen!Brüche}\index{Brüche}

\subsection{Suffix -pxì} Mit Ausnahme von \E{Hälfte} und \E{Drittel}, die separate Lexeme darstellen, werden Brüche gebildet, indem das \N{-ve} einer Ordnungszahl durch \N{-pxì} ersetzt wird. Man beachte die Akzentverschiebung:
\index{-pxiì@\textbf{-pxì}}

\begin{center}
	\begin{tabular}{rl}
		Hälfte & \N{mawl} \\
		Drittel & \N{pan} \\
		Viertel & \N{tsì\ACC{pxì}} \\
		Fünftel & \N{mrr\ACC{pxì}} \\
	\end{tabular}
	\hskip 3em
	\begin{tabular}{rl}
		Sechstel & \N{pu\ACC{pxì}} \\
		Siebtel & \N{ki\ACC{pxì}} \\
		Achtel & \N{vo\ACC{pxì}} \\
	\end{tabular}
\end{center}

\subsubsection{Wortklasse} Im Gegensatz zu den Kardinal- und Ordinalzahlen sind die Bruchwörter Substantive und keine Adjektive
(siehe \horenref{syn:partitive-gen} für die Syntax).

\subsection{Zähler} Um Brüche mit höherem Zähler zu bilden, kombiniert man eine attributive Kardinalzahl mit einem Bruch-Substantiv, \N{munea mrrpxì} \E{zwei Fünftel}.

\subsubsection{Zwei Drittel} Der Bruch \E{zwei Drittel} hat eine besondere Schreibweise: \N{mefan}, der Dual von \N{pan}. \index{mefan@\textbf{mefan}}

\section{Andere Formen}

\subsection{Alo} Das Wort \N{\ACC{a}lo} \E{Mal} wird mit Zahlen kombiniert, um Häufigkeitsadverbien zu bilden. Vier davon bilden Verbindungen, \N{\ACC{'aw}lo} \E{einmal}, \N{\ACC{me}lo} \E{zweimal}, \N{\ACC{pxe}lo} \E{dreimal} und \N{\ACC{fra}lo} \E{jedes Mal}. Alle anderen werden wie normale attributive Adjektive kombiniert, \Npawl{\uwave{alo amrr} poan polawm} \E{er fragte \uwave{fünfmal}}. \index{melo@\textbf{melo}}
\index{'awlo@\textbf{'awlo}}\index{alo@\textbf{alo}}\index{fralo@\textbf{fralo}}
\index{pxelo@\textbf{pxelo}}

\subsection{-lie} Das Wort \N{'aw\ACC{li}e} bezieht sich auf ein einzelnes Ereignis in der Vergangenheit. \index{-lie@\textbf{-lie}}\index{'awlie@\textbf{'awlie}}

\subsection{Fremde Zahlwörter} Bei der Angabe von englischen Zahlwörtern verwendet Na'vi \N{'eyt} für \E{acht} und \N{nayn} für \E{neun}. Diese werden nicht zum Zählen verwendet, sondern z. B. für Telefonnummern. \index{'eyt@\textbf{'eyt}}\index{nayn@\textbf{nayn}}

\subsubsection{} \N{Kew} ist \E{null}. \QUAESTIO{Aus der aktuellen Dokumentation geht nicht hervor, ob diese Idee den Na'vi schon bekannt ist oder ob sie von den Menschen übernommen wurde.}
\index{kew@\textbf{kew}}
