\nchapter{Semantik}

In diesem Abschnitt werden einige Themen behandelt, die bereits in der Grammatik und im Wörterbuch beschrieben sind, die aber an einem Ort gesammelt werden sollten, um bestimmte Muster zu verdeutlichen.

\section{Die Kopula und die Prädikation} \index{lu@\textbf{lu}!Semantik}
Das wichtigste Verb der Substantiv- und Adjektivprädikation ist \N{lu} \E{sein}, zusammen mit dem Verb \N{slu} \E{werden}. Man nennt diese Verben ``Kopulaverben''.

\begin{quotation}
\noindent\N{Oe lu seykxel.} \E{Ich bin stark.} \\
\noindent\N{Oe slu seykxel.} \E{Ich werde stark.} \\
\noindent\N{Oe layu taronyu.} \E{Ich werde ein Jäger sein.} \\
\noindent\N{Oe slayu taronyu.} \E{Ich werde ein Jäger werden.}
\end{quotation}

\subsection{Existenz}
Das Verb \N{lu} wird auch verwendet, um Existenz anzuzeigen (wo im Deutschen \E{es gibt} gebraucht wird), z. B. \Npawl{äo fìutral lu tsmìm 'angtsìkä} \E{es gibt eine Hammerkopfspur unter diesem Baum}; \Npawl{frauvanìri lu yora'tu, lu snaytu} \E{in jedem Spiel gibt es einen Gewinner und einen Verlierer}.

\subsection{Besitz}
Schließlich wird \N{lu} mit dem Dativ verwendet, um Besitz anzuzeigen (wo im Deutschen das Verb \E{haben} gebraucht wird), wie in \Nfilm{lu oeru aylì'u frapor} \E{ich habe allen etwas zu sagen} (wörtlich: ``ich habe Worte für jeden'').

\subsection{Werden}
Aufgrund der freien Wortstellung im Na'vi kann es vorkommen, dass die Beziehungen zwischen zwei Substantiven mit dem Verb \N{slu} \E{werden} unklar sind. In dieser Situation wird die Adposition \N{ne} verwendet, um das Ziel zu verdeutlichen, \N{taronyu slu ne tsamsiyu} \E{der Jäger wird ein Krieger}.

\subsection{Standort} Na'vi hat ein separates Verb, \N{tok}, für \E{(räumlich) sein}, das anstelle des einfachen \N{lu} für eine Standortangabe verwendet wird,

\begin{interlin}
  \glll Awngal tok kelkut. \\
  awnga-l tok kelku-t \\
  wir.\I{incl}-\I{agt} sein.in Haus-\I{pat} \\
  \trans{Wir sind zu Hause.}\Ipawl{}
\end{interlin}

\noindent Man beachte, dass das Verb \N{tok} transitiv ist. Es benötigt ein Agens (\N{awngal}) und ein Patiens für den Ort (\N{kelkut}).

\subsection{Umgangssprachliche Auslassung}
Sowohl \N{lu} als auch \N{tok} können in der Umgangssprache entfallen (\horenref{prag:colloq:omit}). Die normalen Kasusbeziehungen bleiben bestehen, wie z. B. \N{oe seykxel} \E{ich bin stark} (\N{lu} entfällt) und \N{oel fìtsengit} \E{ich bin hier} (\N{tok} entfällt). \N{Lu} entfällt im Riff-Na'vi öfter als im Wald-Na'vi.

\section{Wahrnehmung} \index{Wahrnehmung}
Na'vi-Ausdrücke für Sinneswahrnehmungen unterscheiden zwischen Aktivität, Empfindung und Fähigkeit. Außerdem unterscheiden die Verben, ob man die Kontrolle über die Wahrnehmung hat (\E{ansehen}) oder nicht (\E{sehen}).

\begin{center}
	\begin{tabular}{c|ccccc}
		& \I{vtr} & \I{vtr} & \I{vin} & \I{n} & \I{n} \\
		& -Kontrolle & +Kontrolle & +Kontrolle & Empfindung & Fähigkeit \\
		\hline
		\multirow{2}{*}{sehen} & \N{tse'a}  & \N{nìn}  & \N{tìng nari}  & \N{'ur}  & \N{tse'atswo}  \\
		& {\scriptsize\E{sehen, erblicken}} & {\scriptsize\E{ansehen, hinsehen}} & {\scriptsize\E{ansehen, hinsehen}} & {\scriptsize\E{Aussehen}} & {\scriptsize\E{Sehvermögen}} \\
		\multirow{2}{*}{hören} & \N{stawm}  & \N{yune}  & \N{tìng mikyun}  & \N{pam} & \N{stawmtswo} \\
		& {\scriptsize\E{hören}} & {\scriptsize\E{zuhören}} & {\scriptsize\E{zuhören}} & {\scriptsize\E{Geräusch}}  & {\scriptsize\E{Hörvermögen, Gehör}} \\
		\multirow{2}{*}{riechen} & \N{hefi}  & \N{syam}  & \N{tìng ontu}  & \N{fahew}  & \N{hefitswo}  \\
		& {\scriptsize\E{riechen}} & {\scriptsize\E{riechen}} & {\scriptsize\E{riechen}} & {\scriptsize\E{Geruch}} & {\scriptsize\E{Geruchsvermögen}} \\
		\multirow{2}{*}{schmecken} & \N{ewku}  & \N{may'}  & \N{tìng ftxì}  & \N{sur}  & \N{ewktswo}  \\
		& {\scriptsize\E{schmecken}} & {\scriptsize\E{schmecken}} & {\scriptsize\E{schmecken}} & {\scriptsize\E{Geschmack, Aroma}} & {\scriptsize\E{Geschmackssinn}} \\
		\multirow{2}{*}{fühlen} & \N{zìm}  & \N{'ampi}  & \N{tìng zekwä}  & \N{zir}  & \N{zìmtswo}  \\
		& {\scriptsize\E{fühlen, anfühlen}} & {\scriptsize\E{berühren, anfassen}} & {\scriptsize\E{berühren, anfassen}} & {\scriptsize\E{Berührung, Gefühl}} & {\scriptsize\E{Tastsinn}}
	\end{tabular}
\end{center}

\noindent Man beachte hier insbesondere den Geruchs- und Geschmackssinn: Während im Deutschen die Verben \E{riechen} und \E{schmecken} für eine ganze Reihe von Aktivitäten verwendet werden, benutzt Na'vi andere Wörter dafür. Der zufällige Geruch von etwas erfordert \N{hefi}, während für das absichtliche Riechen von etwas \N{syam} verwendet wird.

Die zusammengesetzten Ausdrücke mit \N{tìng} werden meist verwendet, wenn es kein offensichtliches direktes Objekt gibt, \N{tìng nari!} \E{schau dir das an!}. Sie können aber auch mit dem Dativ für die wahrgenommene Sache verwendet werden, \Npawl{poru tìng nari!} \E{schau ihn an!}, obwohl \Npawl{poti nìn!} \E{schau ihn an!} geläufiger ist.
\NTeri{27/11/2012}{https://naviteri.org/2012/11/renu-ayinanfyaya-the-senses-paradigm/}

\subsection{Erscheinungen} \index{Wahrnehmung!von Erscheinungen} \index{fkan@\textbf{fkan}} \index{Wahrnehmung!von Erscheinungen!fkan@\textbf{fkan}}
Um subjektiv wahrgenommene Erscheinungen auszudrücken, wird das Verb \N{fkan} mit den Empfindungssubstantiven verwendet. \N{Fkan} selbst bedeutet in etwa \E{einer sensorischen Ausprägung ähneln, den Sinnen erscheinen als}.

\begin{interlin} \label{ex:percep2}
\glll Fìnaerìri sur fkan oeru kalin. \\
    fì-naer-ìri sur fkan oe-ru kalin \\
  	dies-Getränk-\I{top} Geschmack den.Sinnen.erscheinen.als ich-\I{dat} süß \\
\trans{Dieses Getränk schmeckt für mich süß.} \Ipawl{}
\end{interlin}

\noindent Man beachte in Beispiel \ref{ex:percep2}, dass der Erlebende der Wahrnehmung im Dativ steht. Dies kann bei allgemeinen Aussagen entfallen. Das Empfindungssubstantiv kann ebenfalls entfallen, wenn eine Interpretation der Wahrnehmung aus dem Kontext heraus offensichtlich ist.

\N{Fkan} wird mit \N{na} \E{wie, als} verwendet, um einen Vergleich anzustellen,

\begin{interlin}
\glll Raluri fahew fkan oeru na yerik. \\
    Ralu-ri fahew fkan oe-ru na yerik \\
    Ralu-\I{top} Geruch den.Sinnen.erscheinen.als ich-\I{dat} wie Hexapede \\
\trans{Ralu riecht für mich wie ein Hexapede.} \Ipawl{}
\end{interlin}

\noindent Auch hier können der Dativ (der Erlebende) und das Empfindungssubstantiv weggelassen werden, wenn die Bedeutung aus dem Kontext ableitbar ist.
