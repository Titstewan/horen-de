\nchapter{Morphologie}

\section{Das Substantiv}

\subsection{Kasus} Die Na'vi-Fallendungen ändern sich abhängig davon, ob das Wort auf einen Konsonanten, einen Vokal oder einen Diphthong endet.\footnote{Die von Frommer verwendeten Kasusbezeichnungen spiegeln die von Bernard Comrie in seinen Schriften über Ergativ-Sprachen verwendete Terminologie wider. In den meisten (deutschsprachigen) linguistischen Schriften wird Frommers ``Subjective'' (d. h. der Fall für das Subjekt eines intransitiven Satzes) als ``Absolutiv'' bezeichnet, der ``Agens'' ist der Ergativ und der ``Patiens'' der Akkusativ. Weil die deutschsprachige linguistische Terminologie keinen ``Subjektiv'' kennt, wird hier der Begriff ``Absolutiv'' verwendet; Frommers ``Agens'' und ``Patiens'' werden beibehalten.}
\index{Substantiv!Deklinierung}
\label{morph:decl}

\begin{center}
	\begin{tabular}{lccc}
		& Vokal  & Konsonant \& Pseudovokal & Diphthong \\
		\hline
		Absolutiv & --- & --- & --- \\
		Agens & \N{-l} & \N{-ìl} & \N{-ìl} \\
		Patiens & \N{-t}, \N{-ti} & \N{-it}, \N{-ti} & \N{-ti}, \N{-it}, \N{-ay-t}, \N{-ey-t} \\
		Dativ & \N{-r}, \N{-ru} & \N{-ur}, \N{-'-ru} & \N{-ru}, \N{-ur}, \N{-aw-r}, \N{-ew-r} \\
		Genitiv & \N{-yä}, \N{-o-ä}, \N{-u-ä} & \N{-ä} & \N{-ä} \\
		Topik  & \N{-ri} & \N{-ìri} & \N{-ri}  \\
\end{tabular}\end{center}

\noindent\LNWiki{24/3/2010}{http://wiki.learnnavi.org/index.php/Canon/2010/March-June\%23Declension_with_Diphthongs_and_Deixis}

% For gen. with i: http://forum.learnnavi.org/language-updates/genitive-case-refinement-declension-of-tsaw/msg150927/#msg150927

\subsubsection{} Man beachte, dass Wörter, die auf die Pseudovokale \N{ll} und \N{rr} enden, die Endungen für Konsonanten erhalten: \N{trr-ä}, \N{'ewll-it}.
\index{Pseudovokal!Deklinierung}\label{morph:decl:pseudovowel}

\subsubsection{} Nach den Vokalen \N{o} und \N{u} steht für den Genitiv einfach \N{-ä}, aber nach allen anderen Vokalen \N{-yä}. Das bedeutet, \N{tsulfätuä} von \N{tsulfätu}, aber \N{Na'viyä} von \N{Na'vi} und \N{lì'fyayä} von \N{lì'fya}.

\subsubsection{} Substantive, die auf \N{-ia} enden, erhalten die Genitiv-Endung \N{-iä}, wie in \N{soaiä} von \N{soaia}.
\NTeri{25/5/2011}{http://naviteri.org/2011/05/some-miscellaneous-vocabulary/}
% This used to be covered by a single example in following "case
% refinement" section link; later generalized.

\subsubsection{} Neben einigen Pronomen (\horenref{morph:pron:irreg-gen}) gibt es wenige Substantive mit unregelmäßigen Genitiven: \N{Omatikayaä} (von \N{Omatikaya}). Diese sind im Wörterbuch gekennzeichnet.
\LNForum{19/3/2010}{https://forum.learnnavi.org/index.php?msg=150927}

\subsubsection{} Aufgrund der lautlichen Ähnlichkeit zwischen \N{y} und \N{i} wird die Patiens-Endung \N{-it} vereinfacht, wenn sie an einen auf \N{y} endenden Diphthong angehängt wird, wie in \N{keyeyt} \E{die Fehler} anstelle von \N{keyeyit}. Und aufgrund der lautlichen Ähnlichkeit zwischen \N{w} und \N{u} kann die gleiche Vereinfachung für den Dativ \N{-ur} gelten, wie in \N{'etnawr} \E{zu/für eine Schulter}, alternativ zu \N{'etnawur}.
\NTeri{1/25/2013}{http://naviteri.org/2013/01/awvea-posti-zisita-amip-first-post-of-the-new-year/}

\subsubsection{} Ein Substantiv, das auf einen glottalen Plosiv endet, kann auch die Dativ-Endung \N{-ru} annehmen, wie \Npawl{lì'fyaolo\uwave{'ru}}. Ansonsten nehmen Substantive, die auf Konsonanten enden, \N{-ur} an.
\NTeri{6/10/2012}{https://naviteri.org/2012/10/audio-and-video-learning-materials-for-navi-101/}
\LNForum{25/12/2020}{https://forum.learnnavi.org/index.php?msg=674179}

\subsubsection{} Die kurzen und langen Varianten der Patiens- und Dativ-Endungen scheinen weitgehend eine Frage des Stils und des Wohlklangs zu sein.

\paragraph{Riff-Na'vi} \label{morph:reef-navi:pat} \index{Riff-Na'vi!Patiens}
Der Riff-Na'vi-Dialekt bevorzugt die Patiens-Form \N{-ti} gegenüber \N{-it} und \N{-t}. \Omaticon

\subsection{Entlehnte Substantive}\index{Substantiv!Deklinierung!entlehnt}
Wenn menschliche Wörter ins Na'vi entlehnt werden, müssen einige von ihnen an die Phonologie des Na'vi angepasst werden, wie z. B. \N{Kerìsmìsì} \E{Weihnachten}. Neben anderen Änderungen ist das finale \N{-ì} erforderlich, da Na'vi-Wörter nicht auf \N{s} enden können. Da einige der Kasusendungen Vokale haben, die sonst illegale Cluster bilden würden, wird das finale \N{-ì} solcher Entlehnungen weggelassen, wie z. B. beim Genitiv, \N{Kerìsmìsä}. Am Beispiel der deutschen Stadt Köln:

\begin{center}
	\begin{tabular}{lll}
		Absolutiv & \N{Kelnì} & \\
		Agens   & \N{Kelnìl} & \N{Keln-ìl}, nicht \N{Kelnì-l} \\
		Patiens & \N{Kelnit} & \\
		Dativ     & \N{Kelnur} & \\
		Genitiv   & \N{Kelnä} & \\
		Topik    & \N{Kelnìri} & \N{Keln-ìri}, nicht \N{Kelnì-ri}
	\end{tabular}
\end{center}

\noindent Man beachte, dass für den Patiens und den Dativ nur die Formen, die mit einem Vokal beginnen -- \N{-it} und \N{-ur} -- zulässig sind. Eine Form wie \N{*Kelnti} ist keine zulässige Wortform.
\NTeri{8/1/2022}{http://naviteri.org/2022/01/aawa-tipangkxotsyip-a-teri-horen-lifyaya-a-few-little-discussions-about-grammar/}

\subsection{Indefinit -o} Ein Substantiv kann das Indefinitsuffix \N{-o} \E{(irgend)ein, etwas} tragen. Kasusendungen folgen auf das \N{-o}, z. B. \N{puk-o-t}.
\index{-o@\textbf{-o}}\index{indefinites Substantiv}
\LNWiki{14/3/2010}{https://wiki.learnnavi.org/index.php/Canon/2010/March-June\%23
	A_Collection}
\NTeri{5/9/2011}{http://naviteri.org/2011/09/\%E2\%80\%9Cby-the-way-what-are-you-reading\%E2\%80\%9D/comment-page-1/\%23comment-1093}

\subsection{Numerus} Na'vi-Substantive und -Pronomen können im Singular, Dual, Trial oder Plural (vier oder mehr) stehen. Der Numerus wird durch Präfixe angegeben, die alle Lenition auslösen.\index{Dual}\index{Trial}\index{Plural}

\begin{center}
	\begin{tabular}{lll}
		Dual & \N{me+} & \N{mefo} ($<$ \N{me+} $+$ \N{po}) \\
		Trial & \N{pxe+} & \N{pxehilvan} ($<$ \N{pxe+} $+$ \N{kilvan}) \\
		Plural & \N{ay+} & \N{ayswizaw} \\
	\end{tabular}
\end{center}

\subsubsection{} Das Pluralpräfix kann \textit{nur} dann entfallen, wenn es Lenition auslöst. Der Plural von \N{prrnen} ist entweder \N{ayfrrnen} oder die kurze Pluralform \N{frrnen} (siehe aber \horenref{syn:adp:short-plural}).\footnote{Ausnahme: \N{'u} \E{Ding} nimmt nicht den kurzen Plural an, sondern tritt immer als \N{ayu} auf. \index{'u@\textbf{'u}!keine kurze Pluralform}} Der Dual und der Trial werden entfallen nie.
\index{Plural!Kurzform} \label{morph:short-plural}
\LanguageLog

\subsubsection{} Wenn ein Wort mit \N{e} oder \N{'e} beginnt, wird das resultierende \N{*ee} kontrahiert, sodass \N{me+} $+$ \N{'eveng} zu \N{meveng} wird. Siehe dazu auch \horenref{lands:phonotactics:nsc}.

\subsubsection{} Im Riff-Na'vi wird \N{pxe-} als \N{be-} ausgesprochen. \index{Riff-Na'vi!Trial-Präfix}
\NTeri{13/1/2023}{http://naviteri.org/2023/01/2653/}

\section{Das Pronomen}

\subsection{Die Personalpronomen}
Die Pronomen erhalten die gleichen Kasusendungen und Numeruspräfixe wie Substantive.

\begin{center}
	\begin{tabular}{rllll}
		Person      & Singular & Dual & Trial & Plural \\ 
		\hline
		1. exklusiv   & \N{\ACC{o}e}  & \N{m\ACC{o}e}  & \N{px\ACC{o}e}   & \N{ay\ACC{o}e} \\
		1. inklusiv   & —      & \N{o\ACC{e}ng} & \N{pxo\ACC{e}ng} & \N{ayo\ACC{e}ng}, \N{aw\ACC{nga}} \\
		2.         & \N{nga} & \N{me\ACC{nga}} & \N{pxe\ACC{nga}} & \N{ay\ACC{nga}} \\
		3. belebt & \N{po}  & \N{me\ACC{fo}} & \N{pxe\ACC{fo}}  & \N{ay\ACC{fo}, fo} \\
		3. unbelebt   & \N{\ACC{tsa}'u}, \N{tsaw} & \N{me\ACC{sa}'u} & \N{pxe\ACC{sa}'u} & \N{ay\ACC{sa}'u, \ACC{sa}'u} \\
		reflexiv & \N{sno} & — & — & — \\
		indefinit & \N{fko} & — & — & — \\
	\end{tabular}
\end{center}

\subsubsection{} Wenn die Grundform der ersten Person \N{oe} nicht am Wortende vorkommt, ändert sich alltagssprachlich ihre Aussprache zu \N{we}, wie in \N{oel} ausgesprochen als \N{wel}, \N{oeru} als \N{weru}. Diese Aussprache gilt jedoch nicht für die Dual- und Trialformen, \N{moe} und \N{pxoe}, die zu unzulässigen Konsonantenclustern am Wortanfang führen würden, wie etwa *\N{mwel}. Diese Aussprache wird in der Tabelle oben durch die Akzentmarkierung unter dem \N{e} angezeigt. \label{morph:pron:oe-we}

Wenn \N{oe} auf ein Wort folgt, das auf \N{-u} endet, wird \N{oe} wie \N{we} ausgesprochen, z. B. wird \N{'efu oe} wie \N{'efu we} ausgesprochen.
\NTeri{30/4/2021}{http://naviteri.org/2021/04/mipa-ayliu-mipa-sioeykting-new-words-new-explanations/}

\subsubsection{} Die Pronomen der ersten Person, welche nicht im Singular stehen, sind entweder exklusiv (d. h. sie schließen die angesprochene Person aus) oder inklusiv (d. h. sie schließen die angesprochene Person ein). Die inklusive Endung, \N{-ng}, stammt von \N{nga}, das vollständig wieder auftaucht, wenn eine Kasusendung hinzugefügt wird. Der Agens von \N{oeng} ist \N{oengal}, nicht \N{*oengìl}.

\subsubsection{} \N{Ayoeng} hat die Kurzform \N{aw\uline{nga}}. Beide können frei mit jeder beliebigen Kasusendung verwendet werden, wobei \N{awnga} häufiger vorkommt.\index{ayoeng@\textbf{ayoeng}}\index{awnga@\textbf{awnga}}

\subsubsection{} Es gibt eigene Pronomen der dritten Person für belebte und unbelebte Objekte. Tiere können mit dem belebten Pronomen \N{po} bezeichnet werden, nicht aber Insekten. Je wichtiger die Beziehung des Sprechers zum Tier ist, desto eher wird eine Form von \N{po} verwendet.\index{Belebtheit}
\LNForum{25/2/2017}{https://forum.learnnavi.org/index.php?msg=650313}

\subsubsection{} Das Pronomen für die dritte belebte Person \N{po} ist geschlechtsunspezifisch -- es kann im Deutschen ``er'' oder ``sie'' bedeuten. Es gibt jedoch geschlechtsspezifische Formen, \N{po\ACC{an}} \E{er} und \N{po\ACC{e}} \E{sie}, die regelmäßig dekliniert werden, obwohl sie keine Pluralformen haben. Siehe \horenref{syn:pron:gender} für ihren Gebrauch.
\label{morph:pron:gender}
\NTeri{17/10/2017}{http://naviteri.org/2017/09/zisikrr-amip-ayliu-amip-new-words-for-the-new-season/\#comment-27696}

\subsubsection{} \label{morph:pron:irreg-gen}
Mehrere Pronomen weisen unregelmäßige Genitive mit Vokaländerung auf:

\begin{center}
	\begin{tabular}{cc}
		Absolutiv & Genitiv \\
		\hline
		\N{fko} & \N{fkeyä} \\
		\N{nga} & \N{ngeyä} \\
		\N{po} & \N{peyä} \\
		\N{sno} & \N{sneyä} \\
		\N{tsa'u} & \N{tseyä} \\
		\N{ayla} & \N{ayleyä}
	\end{tabular}
\end{center}

\noindent Diese Vokaländerung kommt in allen Numeri vor, \N{feyä} $<$ \N{fo}, und in der inklusiven ersten Person, \N{awngeyä} $<$ \N{awnga}.
\index{nga@\textbf{nga}!Genitiv \textbf{ngeyä}}
\index{fko@\textbf{fko}!Genitiv \textbf{fkeyä}}
\index{po@\textbf{po}!Genitiv \textbf{peyä}}
\index{sno@\textbf{sno}!Genitiv \textbf{sneyä}}
\index{awnga@\textbf{awnga}!Genitiv \textbf{awngeyä}}

Von \N{po} abgeleitete Pronomen (\N{'awpo} \E{eine einzelne Person}, \N{frapo} \E{jede Person}, \N{lapo} \E{andere Person}, \N{fìpo} \E{diese Person}, \N{tsapo} \E{jene Person}) sind auch von der Vokaländerung betroffen, d. h. \N{frapeyä}, nicht \N{*frapoä}.

\subsubsection{} In der informellen sowie in knapper Militärsprache kann das abschließende \N{ä} der Genitivendung bei Pronomen entfallen, \N{ngey 'upxaret}.\label{morph:pron:gen-clipped} \index{Genitiv!Kurzform bei Pronomen}\index{Pronomen!Kurze Genitivform}

\subsubsection{} Die dritte unbelebte Person, \N{tsa'u}, ist einfach das Demonstrativpronomen ``dieses'' und steht im Genitiv als \N{tseyä}. In der informellen, schnellen Sprache kann es die Form \N{tsaw} annehmen, die mit Postpositionen gebraucht werden kann (\N{tsawfa}), aber keine Kasusmarkierung annehmen darf (es gibt kein \N{*tsawl}). Der Stamm \N{tsa-} kann jedoch mit den Kasusendungen (\N{tsal, tsar} usw.) oder mit einer Postposition (\N{tsafa}) verwendet werden, wiederum in schneller Sprache.
\label{morph:pron:tsa}\index{tsaw@\textbf{tsaw}}\index{tsa'u@\textbf{tsa'u}}
\LNWiki{6/5/2010}{https://wiki.learnnavi.org/Canon/2010/March-June\%23History_of_Tsaw}
\NTeri{3/8/2011}{http://naviteri.org/2011/08/new-vocabulary-clothing/comment-page-1/\#comment-917}

\subsubsection{} Das Reflexivpronomen \N{sno} kann keine Numerusmarkierung erhalten. \index{sno@\textbf{sno}!keine Numerusmarkierung}

\subsubsection{} Das belebte Indefinitpronomen der dritten Person ist \N{fko} (Genitiv \N{fkeyä}).
\LNWiki{17/5/2013}{https://wiki.learnnavi.org/Canon/2013\%23Double_Dative_and_more}

\subsection{Zeremonielle Pronomen / respektvolle Anrede}

\begin{center}
	\begin{tabular}{rllll}
		& Singular & Dual & Trial & Plural \\ 
		\hline
		1. exklusiv & \N{\ACC{o}he}  & \N{\ACC{mo}he}  & \N{\ACC{pxo}he}   & \N{ay\ACC{o}he} \\
		1. inklusiv & -         & \N{\ACC{o}heng} & \N{\ACC{pxo}heng} & \N{a\ACC{yo}heng} \\
		2.           & \N{nge\ACC{nga}} & \N{menge\ACC{nga}} & \N{pxenge\ACC{nga}} & \N{aynge\ACC{nga}} \\
		3. belebt   & \N{\ACC{po}ho} \\
		3. weiblich      & \N{po\ACC{he}} \\
		3. männlich      & \N{po\ACC{han}} 
	\end{tabular}
\end{center}\index{Pronomen!zeremoniell}\label{morph:hon-pron}

\QUAESTIO{Der Genitiv einer Form wie \N{poho} ist unsicher.}

\NTeri{28/2/2022}{http://naviteri.org/2022/02/lifyengteri-concerning-honorific-language/}

\subsubsection{} Für die inklusiven Formen der ersten Person sind die \N{oheng}-Formen der Standard. Die Verwendung separater Pronomen, \N{ohe ngengasì} mit dem Enklitikum\footnote{\href{https://de.wikipedia.org/wiki/Klitikon}{Vgl. Wikipedia: Klitikon}} \N{sì} \E{und}, ist ebenfalls möglich.
\LNForum{25/10/2022}{https://forum.learnnavi.org/index.php?msg=679687}

\subsection{Lahe}\index{lahe@\textbf{lahe}!Deklinierung}\label{morph:lahe:short}
Wenn es als Pronomen verwendet wird, hat \N{lahe} \E{andere/r/s} Plural-Kurzformen: 

\begin{center}
	\begin{tabular}{lll}
		& Lang & Kurz \\
		Absolutiv & \N{ay\ACC{la}he}     & \N{ay\ACC{la}} \\
		Agens   & \N{ay\ACC{la}hel}    & \N{ay\ACC{lal}} \\
		Patiens & \N{ay\ACC{la}het(i)} & \N{ay\ACC{la}t(i)} \\
		Dativ     & \N{ay\ACC{la}her(u)} & \N{ay\ACC{la}r(u)} \\
		Genitiv   & \N{ay\ACC{la}heyä}   & \N{ay\ACC{le}yä} \\
		Topik    & \N{ay\ACC{la}heri}   & \N{ay\ACC{la}ri}
	\end{tabular}
\end{center}

\noindent Man beachte, dass der Genitiv der Kurzform (\N{ayla}) den unregelmäßigen Vokalwechsel aufweist, der auch bei anderen Pronomen zu beobachten ist
(\horenref{morph:pron:irreg-gen}).
\NTeri{5/5/2023}{http://naviteri.org/2023/05/reef-navi-part-2-morphology-syntax-lexicon-and-more/}

\section{Nominalpräfixe}

\noindent Nominalpräfixe sind Präfixe, die vor Substantive gestellt werden können.\footnote{Anm. d. Ü.: Im englischen \E{Horen leNa'vi} wurde für dieses Kapitel die Überschrift ``Prenouns'' gewählt. Bei den besprochenen Morphemen handelt es sich aber auch in englischer linguistischer Terminologie nicht um ``prenouns'' (\href{https://en.wikipedia.org/wiki/Prenoun}{vgl. Wikipedia: Prenoun}); die deutsche Entsprechung ``Pränomen'' ist kein linguistischer Begriff (\href{https://www.duden.de/rechtschreibung/Praenomen}{vgl. Wikipedia: Pränomen}). Vielmehr handelt es sich hierbei schlicht um Präfixe für Substantive (\href{https://de.wikipedia.org/wiki/Präfix}{vgl. Wikipedia: Präfix}). Daher wurde für dieses Kapitel die Übersetzung ``Nominalpräfix'' bzw. einfach ``Präfix'' gewählt.} \index{Nominalpräfix}

\subsection{Fì-} Dieses Präfix wird für die nahe Deixis\footnote{\href{https://de.wikipedia.org/wiki/Deixis}{Vgl. Wikipedia: Deixis}} (Nahdeixis) verwendet, \E{diese/r/s}.
Wenn es von der Pluralsilbe \N{ay+} gefolgt wird, kontrahieren die beiden Morpheme umgangssprachlich im Allgemeinen zu \N{fay+}, \E{diese}. 
In präziser oder formeller Sprache kann jedoch \N{fìay+} verwendet werden, \Npawl{oel foru fìaylì'ut tolìng a krr, kxawm oe harmahängaw}. \label{morph:prenoun:fi}
\index{fiì-@\textbf{fì-}} \index{fay+@\textbf{fay+}}
\LNForum{27/7/2013}{https://forum.learnnavi.org/index.php?msg=590094}

\subsubsection{} Einige Substantive und Adjektive bilden zusammen mit \N{fì-} Adverbien, wie z. B. \N{fìtrr} \E{heute} and \N{fìtxan} \E{so (sehr)}.

\subsection{Tsa-} Dieses Präfix wird für die ferne Deixis (Ferndeixis) verwendet, \E{jene/r/s.} Wenn es von der Pluralsilbe \N{ay+} gefolgt wird, kontrahieren die beiden Morpheme zu \N{tsay+} \E{jene}.
\index{tsa-@\textbf{tsa-}} \index{tsay+@\textbf{tsay+}}

\subsection{-Pe+} \label{morph:pre:pe} Dieses Interrogativpräfix bedeutet \E{was, welche/r/s} wie in \N{pelì'u} \E{welches Wort?}. Es ist insofern ungewöhnlich, als es entweder ein Präfix (\N{pelì'u}) oder ein Suffix (\N{lì'upe}) sein kann. Wenn es als Präfix vorangestellt ist, wird das folgende Wort leniert. Wenn auf das Präfix die Pluralsilbe \N{ay+} folgt, kontrahieren die beiden Morpheme zu \N{pay+}.

\subsection{Fra-} Dieses Präfix bedeutet \E{alle, jede/r/s}. Wenn es von der Pluralsilbe \N{ay+} gefolgt wird, kontrahieren die beiden Morpheme zu \N{fray+}.
\index{fra-@\textbf{fra-}} \index{fray+@\textbf{fray+}}
\LNForum{27/7/2013}{https://forum.learnnavi.org/index.php?msg=590094}

\subsection{Fne-} Dieses Präfix bedeutet \E{Typ (von), Art (von)}.
\index{fne-@\textbf{fne-}}

\subsubsection{} Das Präfix ist mit dem Substantiv \N{fnel} verwandt, das ebenfalls \E{Typ, Art} bedeutet. Es kann mit einem Substantiv im Genitiv auftreten und hat dann dieselbe Bedeutung wie das Präfix. \N{Tsafnel syulangä} und \N{tsafnesyulang} bedeuten beide \E{diese Art von Blume}.
\index{fnel@\textbf{fnel}}

\subsection{Kontraktion} Wenn ein Präfix mit demselben Vokal endet, mit dem das folgende Wort beginnt, kontrahieren die beiden Vokale, wie in \N{tsatan} \E{jenes Licht} von \N{tsa-atan} (\horenref{lands:phonotactics:precontract}).
\index{Nominalpräfix!Kontraktion}\label{morph:prenoun:contraction}

\subsection{Kombinationen} Die Vorsilben können zu einem einzigen Wort kombiniert werden, und zwar in dieser Reihenfolge: \index{Nominalpräfix!Kombinationen}

\begin{center}
	\begin{tabular}{cccccc}
		\N{fì-} \\
		\N{tsa-} & \N{fra-} & Numerusmarkierung & \N{fne-} & Substantiv & \N{-pe} \\
		\N{pe+}
	\end{tabular}
\end{center}

\noindent Aus jeder Spalte darf nur ein Präfix verwendet werden, und die Fragesilbe wird natürlich nur einmal verwendet. \QUAESTIO{Die vollständigen Einzelheiten dieser Reihenfolge sind für \N{fra-} noch nicht bestätigt.}

\subsubsection{} Kurze Pluralformen (\horenref{morph:short-plural}) werden nicht mit den deiktischen Präfixen verwendet; \N{tsaytele} \E{diese Angelegenheiten}, niemals *\N{tsatele} (Singular \N{txele}). \index{Plural!Kurzform!ohne Nominalpräfixe}


\section{Pronominale und adverbiale Präfigierungen}

\noindent Demonstrativpronomen und bestimmte gebräuchliche Adverbien der Zeit, der Art und Weise und des Ortes sind schlicht Substantive, die mit Präfixen kombiniert werden. Das System ist jedoch nicht vollkommen regelmäßig.\footnote{Anm. d. Ü.: Im englischen \textit{Horen leNa'vi} werden diese Wörter als ``Korrelativa'' (im Sinne von ``Entsprechungen'') bezeichnet. Der Begriff meint in der Linguistik aber etwas anderes (\href{https://de.wikipedia.org/wiki/Korrelat_(Grammatik)}{vgl. Wikipedia: Korrelat}), sodass hier von dessen Verwendung abgesehen wurde.}
% This is what I get for attaching several words to the same footnote.
\addtocounter{footnote}{1}
\newcounter{coraccent}\setcounter{coraccent}{\value{footnote}}
\begin{center}\small
	\begin{tabular}{rllllll}% name person thing time place manner
		& Person & Sache & Aktion & Zeit & Ort & Art und Weise \\
		\hline
		\multirow{2}{*}{dies} & \N{\ACC{fì}po} & \N{fì\ACC{'u}} & \N{fì\ACC{kem}} & \N{set} & \N{fì\ACC{tseng}(e)} & \N{fì\ACC{fya}}  \\ 
		& \E{diese Person} & \E{dies (Sache)} & \E{dies (Aktion)} & \E{jetzt} &  \E{hier} & \E{so, auf diese Weise} \\
		\multirow{2}{*}{jenes} & \N{\ACC{tsa}tu} & \N{\ACC{tsa}'u} & \N{tsakem}\footnotemark[\value{coraccent}] & \N{tsa\ACC{krr}} & \N{tsatseng}\footnotemark[\value{coraccent}] & \N{\ACC{tsa}fya} \\
		& \E{jene Person} & \E{jenes (Sache)} & \E{jenes (Aktion)} & \E{dann} & \E{dort} & \E{auf jene Weise} \\
		\multirow{2}{*}{alle} & \N{\ACC{fra}po} & \N{\ACC{fra}'u} & --- & \N{\ACC{fra}krr} & \N{\ACC{fra}tseng} & \N{\ACC{fra}fya}  \\
		& \E{jede/r} & \E{alles} &  & \E{immer} & \E{überall} & \E{in jeder Weise} \\
		\multirow{2}{*}{kein} & \N{\ACC{kaw}tu} & \N{\ACC{ke}'u} & \N{\ACC{ke}kem} & \N{\ACC{kaw}krr} & \N{\ACC{kaw}tseng} & --- \\
		& \E{niemand} & \E{nichts (Sache)} & \E{nichts (Aktion)} & \E{nie} & \E{nirgends} \\
	\end{tabular}
\end{center}\label{morph:correlatives}
\footnotetext[\value{coraccent}]{Kann auf beiden Silben betont werden.}
\LNWiki{18/5/2011}{http://wiki.learnnavi.org/index.php/Canon/2011/April-December\%23Kawtseng.2C_tsapo_and_prefixes}
\NTeri{24/7/2011}{http://naviteri.org/2011/07/txantsana-ultxa-mi-siatll-great-meeting-in-seattle/comment-page-1/\%23comment-845} % kekem

\subsubsection{} \QUAESTIO{Die Plurale für diese sind ein bisschen merkwürdig. Obwohl \N{tsa'u} von \N{tsa-} und \N{'u} stammt, ist der Plural \N{(ay)sa'u}. Bestätigt, aber Details wären schön. Wie funktioniert es bei \N{tsapo}?}

\subsubsection{} Für die Formen von \N{tsa'u} siehe \horenref{morph:pron:tsa}.

\subsection{Fragen} Wie bei den Substantiven kann das Interrogativaffix \N{-pe+} entweder ein lenierendes Präfix oder ein Suffix sein.

\begin{center}
	\begin{tabular}{rl}
		wer? & \N{pe\ACC{su}}, \N{\ACC{tu}pe} \\
		welches (Sache)? & \N{pe\ACC{u}}, \N{\ACC{'u}pe} \\
		welches (Aktion)? & \N{pe\ACC{hem}} \N{\ACC{kem}pe} \\
		wann? & \N{pe\ACC{hrr}}, \N{\ACC{krr}pe} \\
	\end{tabular}
	\hskip2em
	\begin{tabular}{rll}
		wo? & \N{pe\ACC{seng}}, \N{\ACC{tseng}pe} \\
		wie? & \N{pe\ACC{fya}}, \N{\ACC{fya}pe} \\
		warum? & \N{pe\ACC{lun}}, \N{\ACC{lum}pe} \\
		welche Art und Weise? & \N{pe\ACC{fnel}}, \N{\ACC{fne}pe}\\
	\end{tabular}
\end{center}

\subsubsection{} Das Fragewort für Personen, \N{tupe} / \N{pesu} \E{wer}, hat enorm viele Formen hinsichtlich Geschlecht und Numerus:

\begin{center}
	\begin{tabular}{lccc}
		& Unspezifisch & Männlich & Weiblich \\
		\hline
		Singular & \N{pe\ACC{su}}, \N{\ACC{tu}pe} & 
		\N{pe\ACC{stan}}, \N{tu\ACC{tam}pe} &
		\N{pe\ACC{ste}}, \N{tu\ACC{te}pe} \\
		Dual     & \N{pem\ACC{su}}, \N{me\ACC{su}pe} & 
		\N{pem\ACC{stan}}, \N{me\ACC{stam}pe} &
		\N{pem\ACC{ste}}, \N{me\ACC{ste}pe} \\
		Trial    & \N{pep\ACC{su}}, \N{pxe\ACC{su}pe} & 
		\N{pep\ACC{stan}}, \N{pxe\ACC{stam}pe} &
		\N{pep\ACC{ste}}, \N{pxe\ACC{ste}pe} \\
		Plural   & \N{pay\ACC{su}}, \N{(ay)\ACC{su}pe} & 
		\N{pay\ACC{stan}}, \N{(ay)\ACC{stam}pe} &
		\N{pay\ACC{ste}}, \N{(ay)\ACC{ste}pe} \\
	\end{tabular}
\end{center}

\noindent Genitive wie \N{pesuä} sind den Substantiven ähnlich, d. h. es findet keine Form von Vokalveränderung statt, wie es bei \N{po} vs. \N{peyä} zu beobachten ist.

Die Dual-, Trial- und Pluralformen von \N{pehem} / \N{kempe} folgen einem ähnlichen Muster:

\begin{center}
	\begin{tabular}{lc}
		Singular & \N{pe\ACC{hem}}, \N{\ACC{kem}pe} \\
		Dual & \N{pem\ACC{hem}}, \N{me\ACC{hem}pe} \\
		Trial & \N{pep\ACC{hem}}, \N{pxe\ACC{hem}pe} \\
		Plural & \N{pay\ACC{hem}}, \N{(ay)\ACC{hem}pe} \\
	\end{tabular}
\end{center}

\noindent\NTeri{30/7/2011}{http://naviteri.org/2011/07/number-in-na’vi/}
\LNForum{19/12/2021}{https://forum.learnnavi.org/index.php?msg=677501}

\subsection{Fì'u und tsa'u bei der Bildung von Inhaltssätzen} \label{morph:fwa-tsawa}
Das Demonstrativpronomen \N{fì'u} und das unbelebte Pronomen \N{tsa'u} werden mit der Attributpartikel \N{a} verwendet, um Inhaltssätze (Subjekt- und Objektsätze) einzuleiten (\horenref{syn:clause-nom}). Wenn die attributive Partikel auf bestimmte Kasusformen des Pronomens folgt, kontrahieren sie:
% http://forum.learnnavi.org/language-updates/txelanit-hivawl/

\begin{center}
	\begin{tabular}{rcc}
		Kasus & \N{Fì'u}-Kontraktion & \N{Tsa'u}-Kontraktion \\
		\hline
		Absolutiv & \N{fwa} ($<$ \N{fì'u a}) & \N{\ACC{tsa}wa} ($<$ \N{tsa'u a}) \\
		Agens & \N{\ACC{fu}la} ($<$ \N{fì'ul a}) & \N{\ACC{tsa}la} ($<$ \N{tsa'ul a}) \\
		Patiens & \N{\ACC{fu}ta} ($<$ \N{fì'ut a}) & \N{\ACC{tsa}ta} ($<$ \N{tsa'ut a})\\
		Dativ & \N{\ACC{fu}ra} ($<$ \N{fì'ur a}) & \N{\ACC{tsa}ra} ($<$ \N{tsa'ur a}) \\
		Topik & \N{\ACC{fu}ria} ($<$ \N{fì'uri a}) & \N{\ACC{tsa}ria} ($<$ \N{tsa'uri a})\\
	\end{tabular}
\end{center}
\index{fwa@\textbf{fwa}}\index{tsawa@\textbf{tsawa}}
\index{fula@\textbf{fula}}
\index{futa@\textbf{futa}}\index{tsata@\textbf{tsata}}
\index{furia@\textbf{furia}}\index{tsaria@\textbf{tsaria}}
\LNWiki{18/6/2010}{http://wiki.learnnavi.org/index.php/Canon/2010/March-June\%23The_contrast_between_fwa.2Ftsawa.2C_furia.2Ftsaria}
\LNForum{21/4/2020}{https://forum.learnnavi.org/index.php?msg=670210}

Beim schnellen Sprechen kann \N{futa} wie \N{fta} und \N{tsata} als \N{tsta} ausgesprochen werden. 
\LNForum{25/2/2022}{https://forum.learnnavi.org/index.php?msg=682691}

\subsection{Andere Substantive für die Einleitung von Inhaltssätzen} Während \N{fì'u} und \N{tsa'u} Satzglieder die meisten Inhaltssätze einleiten können, bevorzugen Verben des Hörens, Sprechens und Fragens im Hauptsatz die Substantive \N{fmawn} \E{Nachricht}, \N{tì'eyng} \E{Antwort} und \N{faylì'u} \E{diese Worte}; sowie Verben des Befehlens \N{tson} \E{Pflicht, Verpflichtung}. Hierbei gibt es weniger Kontraktionen: \label{morph:fmawn} 

\begin{center}
	\begin{tabular}{rc}
		Kasus & Kontraktion \\
		\hline
		Absolutiv & \N{teynga} ($<$ \N{tì'eyng a}) \\
		Agens & \N{teyngla} ($<$ \N{tì'eyngìl a}) \\
		Patiens & \N{teyngta} ($<$ \N{tì'eyngit a})
	\end{tabular}
\end{center}
\index{fmawnta@\textbf{fmawnta}} \index{teynga@\textbf{teynga}}
\index{teyngla@\textbf{teyngla}} \index{teyngta@\textbf{teyngta}}
\index{fmawn@\textbf{fmawn}} \index{tiì'eyng@\textbf{tì'eyng}}
\index{fayluta@\textbf{fayluta}} \index{fayliì'u@\textbf{faylì'u}}
\index{tsonta@\textbf{tsonta}}

\noindent Kontraktionen gibt es nur im Patiens für \N{fmawn} und \N{faylì'u}, nämlich \N{fmawnta} ($<$ \N{fmawnit a}) und \N{fayluta} ($<$ \N{faylì'ut a}), sowie \N{tsonta} ($<$ \N{tsonit a}). Siehe \horenref{syn:quot:nominalized} für die Syntax.
\NTeri{2/10/2014}{http://naviteri.org/2014/10/tson-si-fpomron-obligation-and-mental-health/}


\section{Das Adjektiv}
\subsection{Attribution} Attributive Adjektive werden mit dem Substantiv durch das Affix \N{-a-} verbunden, das an das Adjektiv auf der Seite angehängt wird, die dem Substantiv am nächsten ist, wie in \N{yerik awin} oder \N{wina yerik} \E{ein schnelles Yerik}.\label{morph:adj-attr}
\index{-a-@\textbf{-a-}\index{Adjektiv!attributiv!Bildung}}

\subsubsection{} Ein mit \N{le-} abgeleitetes Adjektiv lässt in der Regel das vorangestellte (aber nicht nachgestellte) \N{a-} fallen, sodass entweder \N{ayftxozä lefpom} oder, seltener, \N{ayftxozä alefpom} erscheinen kann. Wenn das Adjektiv mit \N{le-} jedoch vor dem Substantiv steht, wird es immer mit dem Affix \N{-a} versehen, \N{lefpoma ayftxozä}.

\begin{center}
	\begin{tabular}{ll}
		\N{ayftxozä lefpom} & üblich \\
		\N{ayftxozä \uwave{a}lefpom} &  erlaubt \\
		\N{*lefpom ayftxozä} &  ungrammatisch \\
		\N{lefpom\uwave{a} ayftxozä} &  korrekt \\
	\end{tabular}
\end{center}

\section{Das Verb}
\subsection{Infixpositionen} Frommer beschreibt drei Positionen für Verbinfixe: die Vor-Position (Po-sition 0), die erste Position (Position 1) und die zweite Position (Position 2). Jeder Position sind Infixe eines bestimmten Typs zugeordnet (siehe unten).

\subsubsection{} Alle Infixe treten in der letzten und vorletzten Silbe des Verbstamms auf und werden vor dem Vokal, Diphthong oder Pseudovokal der jeweiligen Silbe eingefügt, wie in \N{kä} $>$ \N{k‹ìm›ä} und \N{taron} $>$ \N{t‹ol›ar‹ei›on}.

\subsubsection{} Wenn eine Silbe keinen Anfangskonsonanten hat, steht das Infix trotzdem vor dem Vokal, wie in \N{omum} $>$ \N{‹iv›omum} und \N{ftia}
$>$ \N{fti‹ats›a}.

\subsubsection{} Die Betonung bleibt auf dem Vokal, auf dem sie vor dem Hinzufügen von Infixen lag, \N{\ACC{haw}nu} $>$ \N{h‹il\ACC{v›aw}nu}.\footnote{Ausnahme: Zweisilbige Verben, die vokalisch anlauten und im Infinitiv auf der zweiten Silbe betont werden\linebreak (z. B. \N{o\ACC{mum}}), verschieben ihre Betonung in jeder flektierten oder abgeleiteten Form auf die erste Silbe des Verbstamms (\N{i\ACC{vo}mum}, \N{nìaw\ACC{no}mum}). Weitere Verben, auf die dies zutrifft, sind \N{a\ACC{ho}}, \N{i\ACC{nan}} und \N{u\ACC{e'}}. \index{inan@\textbf{inan}!Betonung}}

\subsubsection{} Normalerweise werden Infixe nur in ein Element eines zusammengesetzten Verbs eingefügt. Z. B. ist \N{yomtìng} \E{füttern} eine Verbindung aus \N{yom} \E{essen} und \N{tìng} \E{geben}. Das Perfekt davon ist nicht *\N{y\INF{ol}omtìng}, sondern \N{yomt\INF{ol}ìng}. Das verbale Element steht bei den meisten zusammengesetzten Verben an letzter Stelle und nimmt die Infixe auf. Bei einigen wenigen Verbindungen werden Infixe jedoch ins erste Element gesetzt. Diese Verben müssen aus dem Wörterbuch gelernt werden. \index{Verb!zusammengesetzt!Infixpositionen}

\subsubsection{} Eine kleine Anzahl von Verb+Verb-Verbindungen kann Infixe in beiden Elementen aufnehmen, wie z. B. \N{kan'ìn} \E{spezialisieren auf}, zusammengesetzt aus \N{kan} \E{zielen, beabsichtigen} und \N{'ìn} \E{beschäftigt sein}.
\Ultxa{2/10/2010}{http://wiki.learnnavi.org/index.php/Canon/2010/UltxaAyharyu\%C3\%A4\%23Transitivity_and_Infix_Positions}

\subsection{Vor-Position} Diese Infixe ändern die Transitivität. Sie werden vor dem Vokal der vorletzten Silbe eines Verbs eingefügt, oder bei einsilbigen Verben vor den Vokal der einzigen Silbe (Position 0).
\label{morph:pre-first}
\index{Reflexiv!Bildung}\index{Kausativ!Bildung}

\begin{center}
	\begin{tabular}{lr}
		Kausativ & \N{\INF{eyk}} \\
		Reflexiv & \N{\INF{äp}} \\
	\end{tabular}
\end{center}

\noindent\LNWiki{1/2/2010}{http://wiki.learnnavi.org/index.php/Canon\%23Reflexives_and_Naming} % reflexive
\LNWiki{15/2/2010}{http://wiki.learnnavi.org/index.php/Canon\%23Trials_.26_transitivity} % causative

\subsubsection{}
\index{si-Konstruktion@\textbf{si}-Konstruktion!Reflexiv!Aussprache}
In der Umgangssprache wird das reflexive Perfekt von \N{si}-Verben, \N{säpo\ACC{li}}, oft als \N{spo\ACC{li}} und der reflexive Subjunktiv \N{säpi\ACC{vi}} gewöhnlich als \N{spi\ACC{vi}} ausgesprochen.
\NTeri{30/9/2010}{http://naviteri.org/2010/09/quick-follow-up/}
\NTeri{3/8/2011}{http://naviteri.org/2011/08/new-vocabulary-clothing/}


\subsubsection{} \index{Reflexiv!eines Kausativs}
Das kausative Reflexiv, ``sich selbst (zu etwas) veranlassen'', wird mit \N{\INF{äp}\INF{eyk}} gebildet, d. h. \N{po täpeykerkup} \E{er veranlasst sich selbst dazu, zu sterben}. Die Transitivität des ursprünglichen Ausdrucks bestimmt die Kasusverwendung mit Verben, die \N{\INF{äp}\INF{eyk}} enthalten. Siehe \horenref{reflexive-of-causative} für den Gebrauch.

\subsection{Erste Position} Diese Infixe markieren Tempus, Aspekt und Modus und bilden Partizipien. Sie werden vor dem Vokal der vorletzten Silbe eines Verbs eingefügt (Position 1), oder bei einsilbigen Verben vor den Vokal der einzigen Silbe. Sie folgen immer unmittelbar auf Infixe in der Vor-Position. \label{morph:verb:first-position}

\begin{center}
	\begin{tabular}{r|ccc}
		& Nur Tempus & Perfektiv & Imperfektiv \\
		\hline
		Zukunft & \N{\INF{ay}, \INF{asy}} & \N{\INF{aly}} & \N{\INF{ary}} \\
		nahe Zukunft & \N{\INF{ìy}, \INF{ìsy}} & \N{\INF{ìly}} & \N{\INF{ìry}} \\
		ohne Tempusmarkierung    &  — & \N{\INF{ol}} & \N{\INF{er}} \\
		nahe Vergangenheit & \N{\INF{ìm}} & \N{\INF{ìlm}} & \N{\INF{ìrm}} \\
		Vergangenheit & \N{\INF{am}} & \N{\INF{alm}} & \N{\INF{arm}} \\
	\end{tabular}
\end{center}
\LanguageLog{}
\LNWiki{27/1/2010}{http://wiki.learnnavi.org/index.php/Canon\%23Extracts_from_various_emails}
\LNWiki{19/2/2010}{http://wiki.learnnavi.org/index.php/Canon\%23Evidential} %ìrm

\subsubsection{} Die Zukunftsinfixe mit \N{s}, \N{\INF{asy}} und \N{\INF{ìsy}}, markieren Absicht (\horenref{syn:verb:intenfut}).

\subsubsection{} Der Subjunktiv, \N{\INF{iv}}, hat eine begrenzte Anzahl von Kombinationen mit weniger Zeitabstufungen.
\index{Subjunktiv!Infix}

\begin{center}
	\begin{tabular}{r|ccc}
		& Nur Tempus & Perfektiv & Imperfektiv \\
		\hline
		Zukunft & \N{\INF{ìyev}, \INF{iyev}} & — & — \\
		ohne Tempusmarkierung & — & \N{\INF{ilv}} & \N{\INF{irv}} \\
		Vergangenheit & \N{\INF{imv}} & — & —
	\end{tabular}
\end{center}

\noindent\LNWiki{9/1/2010}{http://wiki.learnnavi.org/index.php/Canon\%23Extracts_from_various_emails}
\LNWiki{30/1/2010}{http://wiki.learnnavi.org/index.php/Canon\%23Vocabulary_and_.3Ciyev.3E}
\LNWiki{30/1/2010}{http://wiki.learnnavi.org/index.php/Canon\%23Fused_-iv-_Infixes}

\subsubsection{} Es gibt nur zwei Partizip-Infixe. Sie lassen sich nicht mit Zeit-, Aspekt- oder Modus-Infixen kombinieren. \index{Partizip!Bildung}

\begin{center}
	\begin{tabular}{lr}
		Aktiv & \N{\INF{us}} \\
		Passiv & \N{\INF{awn}} \\
	\end{tabular}
\end{center}

\noindent Da Partizipien Eigenschaften von Adjektiven annehmen, die nicht prädikativ verwendet werden können, treten sie immer mit dem attributiven Adjektiv-Affix \N{-a-} auf. (\horenref{morph:adj-attr}, \horenref{syn:part:attr}).
\LNWiki{13/3/2011}{http://wiki.learnnavi.org/index.php/Canon/2010/March-June\%23Participial_Infixes}


\subsection{Zweite Position} Diese Infixe, welche die Einstellung oder das Urteil des Sprechers anzeigen, stehen vor dem Vokal der letzten Silbe des Verbs (Position 2), oder bei einsilbigen Verben unmittelbar nach den Infixen der Vor-Position bzw. der ersten Position.
\index{Affektinfixe}\label{morph:verb:2nd-pos}

\begin{center}
	\begin{tabular}{rl}
		positive Einstellung & \N{\INF{ei}}, \N{\INF{eiy}} (\horenref{lands:eiy-epenth}) \\
		negative Einstellung & \N{\INF{äng}}, \N{\INF{eng}} (\horenref{lands:eng}) \\
		formal, zeremoniell & \N{\INF{uy}} \\
		schlussfolgernd, annehmend & \N{\INF{ats}} \\
	\end{tabular}
\end{center}

\noindent\LNWiki{19/2/2010}{http://wiki.learnnavi.org/index.php/Canon\%23Evidential} % <ats>

\subsection{Beispiele} Die oben genannten Regeln sind etwas abstrakt, daher gebe ich hier Beispiele möglicher Flexionen für einige Verbformen. Die Verben sind \N{eyk} \E{(an)führen} als Beispiel für ein einsilbiges Verb ohne Anfangskonsonanten, \N{fpak} \E{aufhalten, unterbrechen} als einsilbiges Verb mit anlautendem Konsonantencluster, \N{taron} \E{jagen} als das übliche zweisilbige Verb, das Frommer in Beispielen verwendet, und \N{yom·tìng} \E{füttern} als zusammengesetztes Verb, bei dem nur das letzte Element flektiert wird.

\begin{center}
	\begin{tabular}{lllll}
		& \N{eyk} & \N{fpak} & \N{\ACC{ta}ron} & \N{\ACC{yom}-tìng} \\
		\hline
		nahe Vergangenheit & \N{ì\ACC{meyk}} & \N{fpì\ACC{mak}} & \N{tì\ACC{ma}ron} & \N{\ACC{yom}tìmìng} \\
		reflexiv  & \N{ä\ACC{peyk}} & \N{fpä\ACC{pak}} & \N{tä\ACC{pa}ron} & \N{\ACC{yom}täpìng} \\
		reflexiv, nahe Vergangenheit & \N{äpì\ACC{meyk}} & \N{fpäpì\ACC{mak}} & \N{täpì\ACC{ma}ron} & \N{\ACC{yom}täpìmìng} \\
		zeremoniell & \N{u\ACC{yeyk}} & \N{fpu\ACC{yak}} & \N{\ACC{ta}ruyon} & \N{\ACC{yom}tuyìng} \\
		perfektiv, zeremoniell & \N{olu\ACC{yeyk}} & \N{fpolu\ACC{yak}} & \N{to\ACC{la}ruyon} & \N{\ACC{yom}toluyìng} \\
		reflexiv, perfektiv, zeremoniell & \N{äpolu\ACC{yeyk}} & \N{fpäpolu\ACC{yak}} & \N{täpo\ACC{la}ruyon} & \N{\ACC{yom}täpoluyìng} \\
	\end{tabular}
\end{center}

\noindent Die Bedeutungen einiger dieser Beispiele strapazieren den gesunden Menschenverstand bis zum Äußersten. Der Zweck ist nur, Infixpositionen in einer konsistenten Gruppe von Verbformen zu zeigen.
