\nchapter{Wortbildung}

\section{Derivationsaffixe (ableitende Affixe)}\index{Affix!Derivationsaffix}\label{lingop:affixes}
\noindent Na'vi hat eine Reihe von Affixen, die neue Lexeme bilden können. Einige ändern einfach die Wortklasse, z. B. verwandeln sie ein Substantiv in ein Adjektiv. Allerdings sind die Bedeutungen der abgeleiteten Formen nicht immer völlig vorhersehbar. Nur mithilfe eines Wörterbuchs kann man sich der Bedeutung eines abgeleiteten Wortes sicher sein (siehe aber \horenref{syn:nifyao} für Adverbien). \textit{Diese Affixe sollten, sofern nicht anders angegeben, nicht als frei produktiv betrachtet werden.}

Es gibt zwar starke Muster, wie die Betonung durch einige Ableitungsprozesse verändert wird, aber es gibt keine ausnahmslosen Regeln dafür. Auch hier kann man sich nur mithilfe des Wörterbuchs über die Betonung eines abgeleiteten Wortes sicher sein.

\subsection{Präfixe} Diese Ableitungspräfixe führen nur selten dazu, dass sich die Betonung von ihrer ursprünglichen Position entfernt, \N{ngay} \E{wahr} $>$ \N{tì\ACC{ngay}} \E{Wahrheit}. 

\subsubsection{} \N{Le-} bildet Adjektive aus Substantiven, wie in \N{le\ACC{hrr}ap} \E{gefährlich} aus \N{\ACC{hrr}ap} \E{Gefahr}.
\index{le-@\textbf{le-}}

\subsubsection{} \N{Nì-} bildet Adverbien aus Substantiven, Pronomen, Adjektiven und Verben, wie etwa in \N{nìNa'vi} \E{auf Na'vi-Art} aus \N{Na'vi}, \N{nìayfo} \E{wie sie} aus \N{ayfo} \E{sie}, \N{nì\ACC{ftu}e} \E{in einfacher Weise} aus \N{\ACC{ftu}e} \E{einfach}, und \N{nì\ACC{tam}} \E{genug} aus \N{tam} \E{genügen, ausreichen}. \index{niì-@\textbf{nì-}}

Dieses Präfix ist mit Adjektiven, Pronomen und Ordnungszahlen produktiv, aber nicht mit anderen Wortarten.
\NTeri{11/7/2010}{https://naviteri.org/2010/07/diminutives-conversational-expressions/}
\LNWiki{5/6/2013}{https://wiki.learnnavi.org/Canon/2013\%23Ordinals_.26_nume}
\LNWiki{27/7/2013}{https://wiki.learnnavi.org/index.php/Canon/2013\%23Na.27vi_details_from_Avatarmeet_2013}

\subsubsection{} \N{Sä-} bildet Substantive aus Verben und Adjektiven, wie in \N{sä\ACC{nu}me} \E{Unterweisung, Unterricht} aus \N{\ACC{nu}me}, und \N{sä\ACC{spxin}} \E{Krankheit} aus \N{spxin} \E{krank}. \index{saä-@\textbf{sä-}}

\subsubsection{} \N{Sä-} bildet auch Substantive, die eine bestimmte, konkrete Ausführung einer Handlung bezeichnen. Ein \N{sätsyìl} \E{Aufstieg} ist ein bestimmter Vollzug der Handlung des Kletterns, \N{tsyìl}. Wortstämme können sowohl mit \N{tì-} als auch mit \N{sä-} abgeleitet werden, wie in \N{'ipu} \E{humorvoll}. \N{Tì'ipu} ist der abstrakte Begriff des Humorvollen, d. h. des Humors im Allgemeinen. \N{Sä'ipu} hingegen bezeichnet einen konkreten Fall von Humor -- beispielsweise ein Witz. \NTeri{29/2/2012}{https://naviteri.org/2012/02/trr-asawnung-lefpom-happy-leap-day/}

\subsubsection{} \N{Tì-} bildet Substantive aus Verben, Adjektiven und gelegentlich anderen Substantiven, wie in \N{tì\ACC{ngay}} \E{Wahrheit} aus \N{ngay} \E{wahr}, \N{tìfti\ACC{a}} \E{Studium} aus \N{fti\ACC{a}} \E{studieren}, \N{tì\ACC{'awm}} \E{(das) Kampieren, (das) Lagern} aus \N{'awm} \E{Lager}. \index{tiì-@\textbf{tì-}}

\subsubsection{} Mit diesen Präfixen können Stammsilben einen Vokal verlieren, wenn der anlautende Konsonant auch ein gültiger auslautender Konsonant ist, \N{nìm\ACC{wey}pey} \E{auf geduldige Weise} $<$ \N{ma\ACC{wey}-pey} \E{sich gedulden, geduldig sein}.

\subsection{Negatives Präfix} Einige Wörter -- meist, aber nicht ausschließlich Adjektive -- werden gebildet, indem das Wort \N{ke} \E{nicht} als Präfix gebraucht wird.

\subsubsection{} Wenn \N{ke-} vor dem Adjektivpräfix \N{le-} steht, wird dieses auf \N{-l-} reduziert, wie in \N{kel\ACC{tsun}} \E{unmöglich} im Vergleich zu \N{le\ACC{tsun}slu} \E{möglich}, und \N{kel\ACC{fpom}tokx} \E{ungesund} aus \N{lefpom\ACC{tokx}} \E{gesund}.

\subsubsection{} Wenn \N{le-} vor \N{ke-} steht, wird das Negativpräfix auf \N{-k-} reduziert, wie in \N{lek\ACC{ye}'ung} \E{wahnsinnig, verrückt} aus \N{ke\ACC{ye}'ung} \E{Wahnsinn, Irrsinn}.

\subsubsection{} Das Präfix \N{ke-} kann mit Adjektivstämmen und Partizipien verwendet werden. In diesem Fall verschiebt sich die Betonung normalerweise zu \N{ke-}, wie in \N{\ACC{ke}teng} \E{unterschiedlich} aus \N{teng} \E{gleich} und \N{\ACC{ke}rusey} \E{tot} aus \N{ru\ACC{sey}} \E{lebendig}. Man beachte aber \N{ke\ACC{yawr}} \E{inkorrekt} aus \N{e\ACC{yawr}} \E{korrekt}.\label{lingop:prefix:ke}

\subsubsection{} Das Präfix \N{ke-} kann auch Substantive bilden und mit ihnen kombiniert werden, wie in \N{ke\ACC{ye}'ung} \E{Wahnsinn, Irrsinn}, und \N{\ACC{ke}tuwong} \E{Alien}. \QUAESTIO{Es gibt zu wenige Beispiele, um das Betonungsverhalten zu bestimmen.}

\subsection{Adverbiales ``a-''} Zwei Zustandsverben, \N{lìm} \E{fern sein} und \N{sim} \E{nah sein, näher sein} haben die adverbialen Formen \N{a\ACC{lìm}} \E{weit entfernt} und \N{a\ACC{sim}} \E{nahe, aus nächster Nähe}. Diese werden als ``versteinerte'' Abkürzungen von Formen wie \N{nìfya'o a lìm} (\horenref{syn:nifyao}) betrachtet. Sie sind lexikalisierte Elemente und haben keine Formen wie \N{*lìma} und \N{*sima}.
\index{-a-@\textbf{-a-}!mit Adverbien}\index{aliìm@\textbf{alìm}}\index{asim@\textbf{asim}}
\LNWiki{17/5/2010}{https://wiki.learnnavi.org/index.php/Canon/2010/March-June\%23Near.2C_Distant_and_Irregular_Adverbs}

\subsection{Präfix mit Infix} Es gibt eine einzige Ableitung, die eine Kombination aus einem Präfix und einem Infix verwendet.

\subsubsection{} \N{Tì- ‹us›} bildet das Gerundium (substantiviertes Verb).\footnote{In der neuhochdeutschen Grammatik wird die Bezeichnung ``Gerundium'' nicht verwendet, weil es in der modernen deutschen Sprache kein Gerundium im eigentlichen Sinne mehr gibt. \href{https://de.wikipedia.org/wiki/Gerundium}{Vgl. Wikipedia: Gerundium}, \href{https://de.wikipedia.org/wiki/Substantivierung}{Substantivierung}} Es ist voll produktiv für Verb-stämme und Zusammensetzungen (\N{si}-Verben \horenref{lingop:si-const} können nicht in ein Gerundium verwandelt werden). Dies ist vor allem dann sinnvoll, wenn eine einfache \N{tì-}-Ableitung bereits eine feste Bedeutung hat, wie in \N{rey} \E{leben}, \N{tì\ACC{rey}} \E{Leben}, aber \N{tìru\ACC{sey}} \E{das Leben, i. S. v. ``zu leben''}. Bei zusammengesetzten Verben steht \N{tì-} am Anfang des Wortes und \N{‹us›} wird in das verbale Element der Zusammensetzung eingefügt; \N{yomtìng} wird zu \N{tìyomtusìng}. Siehe auch
\horenref{syn:gerund}.
\index{Gerundium!Bildung}\label{lingop:gerund}
\index{si-Konstruktion@\textbf{si}-Konstruktion!kein Gerundium}
\LNForum{31/1/2013}{https://forum.learnnavi.org/index.php?msg=572997}

\subsection{Agensnominalisierung} Auch diese Suffixe, die aus einem Verb oder Substantiv das entsprechende Substantiv bilden, das die jeweilige Handlung ausführt (das Agens), führen nicht zu einer Betonungsverschiebung.

\subsubsection{} \index{-tu@\textbf{-tu}}
\N{-tu} bildet Substantive meist aus anderen Wortarten als Verben, wie in \N{\ACC{pam}tseotu} \E{Musiker} aus \N{\ACC{pam}tseo} \E{Musik}, \N{tsul\ACC{fä}tu} \E{Meister eines Handwerks oder einer Fähigkeit, Experte} aus \N{tsul\ACC{fä}} \E{Meisterschaft}. Wenn es an Verben angehängt ist, kann es sich entweder auf den Agens (den Ausführenden) oder den Patiens (den Empfänger) einer verbalen Handlung beziehen, wie z. B. \N{\ACC{frr}tu} \E{Gast} aus \N{\ACC{frr}fen} \E{besuchen} (Agens), \N{spe\ACC{'e}tu} \E{Gefangener} aus \N{spe\ACC{'e}} \E{fangen} (Patiens). Dieses Suffix ist nicht produktiv.
\NTeri{30/4/2021}{https://naviteri.org/2021/04/mipa-ayliu-mipa-sioeykting-new-words-new-explanations/}

\subsubsection{} \N{-yu} bildet Substantive aus Verben, um eine Person zu bezeichnen, die regelmäßig eine Tätigkeit oder Rolle ausübt, wie in \N{taronyu} \E{Jäger} aus \N{taron} \E{jagen}. Dieses Suffix ist voll produktiv, sowohl mit regelmäßigen Verben als auch mit \N{si}-Verben, wie z. B. \N{stiwisiyu} \E{Störenfried}, aus \N{stiwi si} \E{Unfug treiben}.
\NTeri{11/7/2010}{https://naviteri.org/2010/07/diminutives-conversational-expressions/}
\LNForum{30/10/2020}{https://forum.learnnavi.org/index.php?msg=673394}

\subsection{Diminutivsuffix} Das unbetonte Suffix \N{-tsyìp} kann frei verwendet werden, um Diminutive zu bilden, sowohl bei Substantiven als auch bei Pronomen. Personennamen können bei der Derivation mit diesem Suffix Silben verlieren, wie z. B. \N{Kamtsyìp} oder \N{Kamuntsyìp} für \E{kleiner Kamun}. Der Diminutiv hat drei Verwendungen.
\label{lingop:dimin}\index{Diminutiv}\index{tsyiìp@\textbf{-tsyìp}}
\NTeri{11/7/2010}{https://naviteri.org/2010/07/diminutives-conversational-expressions/}

\subsubsection{} Erstens kann die Diminutivform eine lexikalisierte Ableitung sein. Solche Wörter finden sich im Wörterbuch, wie z. B. \N{puktsyìp} \E{Broschüre, Pamphlet} aus \N{puk} \E{Buch}. Das Suffix ist hier semantisch so schwach, dass man das Adjektiv \N{tsawl} \E{groß} mit einem lexikalisierten Diminutiv ohne Widerspruch verwenden kann, wie in \N{tsawla utraltsyìp} \E{ein großer Busch}.

\subsubsection{} Zweitens kann der Diminutiv Zuneigung oder Zärtlichkeit ausdrücken, \Npawl{za'u fìtseng, ma 'itetsyìp} \E{komm her, kleine Tochter}. Diese Verwendung sollte nicht als Hinweis auf ein Alter verstanden werden. Die Tochter im Beispielsatz könnte eine Erwachsene sein.

\subsubsection{} Drittens kann der Diminutiv eine Herabsetzung oder Beleidigung ausdrücken, \Npawl{fìtaronyutsyìp ke tsun ke'ut stivä'nì} \E{dieser (wertlose) kleine Jäger kann nichts fangen}. Der abschätzige Ton kann auch auf sich selbst gerichtet sein, \Npawl{nga nìawnomum to \uwave{oetsyìp} lu txur nìtxan} \E{wie jeder weiß, bist du viel stärker als \uwave{meine Wenigkeit}}. Nur der Kontext unterscheidet die abschätzige von der liebevollen Verwendung des Diminutivs.

\subsection{-nay} Durch dieses Suffix wird ein neues Substantiv gebildet, das auf etwas Niedrigeres in einer Hierarchie (Größe, Rang, Leistung usw.) hinweist. Das Suffix erhält die Betonung, \N{karyu\ACC{nay}} \E{Lehrling} aus \N{karyu} \E{Lehrer}. Wenn das Substantiv bereits auf \N{-n} endet, verliert das Suffix das \N{-n-}, \N{'eyla\ACC{nay}} \E{Bekannte/r} aus \N{'eylan} \E{Freund}, \N{ikra\ACC{nay}} \E{Wald-Ikran} aus \N{ikran} \E{Ikran}. Es ist nicht produktiv.
\index{-nay@\textbf{-nay}}
\NTeri{28/2/2013}{https://naviteri.org/2013/02/vospxi-ayol-posti-apup-short-post-for-a-short-month/}

\subsection{Geschlechtssuffixe} Die Geschlechtssuffixe sind insofern ungewöhnlich, als sie nicht nur mit Substantiven, sondern auch mit den Pronomen der dritten Person verwendet werden\\ (\horenref{morph:pron:gender}).\label{lingop:suffix:gender}

\subsubsection{} Das Suffix \N{-an} bezeichnet männliche Personen, wie in \N{po\ACC{an}} \E{er} und \N{\ACC{'i}tan} \E{Sohn}.

\subsubsection{} Das Suffix \N{-e} bezeichnet weibliche Personen, wie in \N{po\ACC{e}} \E{sie} und \N{\ACC{'i}te} \E{Tochter}.

\subsubsection{} Die Wirkung dieser Suffixe auf die Betonung ist unvorhersehbar, \N{tu\ACC{tan}} \E{männliche Person} aus \N{\ACC{tu}te} \E{Person}, aber \N{mun\ACC{txa}tan} \E{Ehemann} aus \N{mun\ACC{txa}tu} \E{Ehepartner}.

\section{Reduplikation}
\noindent Die Reduplikation ist ein nicht-produktiver Ableitungsprozess. Dennoch gibt es einige gebräuchliche Wörter, die sie verwenden. \index{Reduplikation}

\subsection{Iteration} Bei Wörtern der Zeit zeigt die Reduplikation Wiederholung oder gewohnheitsmäßiges Auftreten an, \N{letrrtrr} \E{gewöhnlich}, i. S. v. ``täglich vorkommend'', \N{krro krro} \E{gelegentlich}.

\subsection{Verschiebung des Grades} Mit den Verben \N{'ul} \E{zunehmen} und \N{nän} \E{abnehmen} markieren reduplizierte Adverbien Veränderungen in extremem Ausmaß, \N{nì'ul'ul} \E{zunehmend, mehr und mehr}, \N{nìnänän}\footnote{Die Reduplikation ist partiell, da Konsonanten nicht verdoppelt werden können.} \E{weniger und weniger}.
\NTeri{29/2/2012}{https://naviteri.org/2012/02/trr-asawnung-lefpom-happy-leap-day/}

\section{Kompositionen}

\subsection{Kopf} Das dominante (d. h. die grammatischen Eigenschaften des Wortes bestimmende) Element einer Na'vi-Komposition kann an erster (Kopf-links) oder letzter (Kopf-rechts) Stelle der Komposition stehen.\footnote{Viele menschliche Sprachen sind restriktiver. Bei englischen Komposita z. B. steht das dominante Element (der ``Kopf'') generell zuletzt, wie in \textit{blueberry}, \textit{night-light}, \textit{blackboard}. Im Deutschen ist es ähnlich, wie in \textit{Krautsalat}, \textit{Tischplatte} oder \textit{weinrot}. Andererseits nutzt das Vietnamesische das Kopf-links-Prinzip für eigene Kompositionen und das Kopf-rechts-Prinzip für Kompositionen, die es aus dem chinesischen Vokabular entlehnt hat.} Es besteht jedoch eine starke Tendenz zum Kopf-rechts-Prinzip bei Komposita. Verbkompositionen folgen am ehesten dem Kopf-links-Prinzip.

\subsubsection{} Komposita gehören der Wortklasse ihres Kopfes an, also ist \N{txam\ACC{pay}} \E{Meer} ein Substantiv, weil \N{pay} \E{Wasser} ein Substantiv ist.

\subsubsection{} Wie die Wortstämme können auch die Kompositionen die Wortklasse durch Hinzufügen der oben aufgeführten Derivationsaffixe ändern, \N{lefpom\ACC{tokx}} \E{gesund} aus \N{fpom\ACC{tokx}} \E{Gesundheit}.

\subsection{Apokope} Wörter können Wortbestandteile verlieren (Apokope\footnote{\href{https://de.wikipedia.org/wiki/Apokope_(Sprachwissenschaft)}{Vgl. Wikipedia: Apokope}}), wenn sie in einer Komposition verwendet werden, wie in \N{\ACC{ven}zek} \E{Zeh} $<$ \N{\ACC{ve}nu} \E{Fuß} $+$ \N{\ACC{zek}wä} \E{Finger}, und \N{sìl\ACC{pey}} \E{hoffen} $<$ \N{sìltsan} \E{gut} $+$ \N{pey} \E{warten}.

\subsection{``Si''-Konstruktionen} Die übliche Art, ein Substantiv oder Adjektiv in ein Verb umzuwandeln, besteht darin, den unflektierten Wortstamm mit dem abstrakten Verb \N{si} zu kombinieren, das nur in diesen Konstruktionen vorkommt. Die Reihenfolge ist fest, Stamm $+$ \N{si}, wobei \N{si} alle Verbalaffixe erhält.\label{lingop:si-const}
\index{si-Konstruktion@\textbf{si}-Konstruktion}

\subsubsection{} Im Verb \N{irayo si} \E{danken} ist die Reihenfolge freier.
\LNWiki{12/5/2010}{https://wiki.learnnavi.org/index.php/Canon/2010/March-June\%23Word_Order_Issues}

\subsubsection{} Die normale Stamm $+$ \N{si}-Wortfolge wird bei der Negation aufgehoben, \N{oe pamrel ke si} \E{ich schreibe nicht} (\horenref{syn:neg:si-const}), \N{txopu rä'ä si} \E{habe keine Angst} (\horenref{syntax:prohibitions}).

\section{Häufige und bemerkenswerte Kompositionselemente}

\subsection{-fkeyk} Abgeleitet von dem Substantiv \N{tìfkeytok} \E{Zustand, Situation}, bildet dieses unbetonte Suffix einige Wörter mit speziellen, idiomatischen Bedeutungen, wie z. B. \N{\ACC{ya}fkeyk} \E{Wetter}. Dennoch ist es sehr produktiv, \Npawl{kilvanfkeyk lu fyape fìtrr?} \E{wie ist der Zustand des Flusses heute?}
\index{-fkeyk@\textbf{-fkeyk}}
\NTeri{1/4/2011}{https://naviteri.org/2011/04/yafkeykiri-plltxe-frapo-everyone-talks-about-the-weather/}

\subsection{Hì(')-} Von dem Adjektiv \N{hì'i} \E{klein} stammt das betonte Präfix \N{hì-} oder \N{hì'-}, das in einigen wenigen Wörtern zur Bildung von Diminutiven verwendet wird, aber nicht als produktiv angese-hen werden sollte (siehe \horenref{lingop:dimin}), wie in \N{\ACC{hì}'ang} \E{Insekt} ($<$ \N{hì'} + \N{ioang} \E{Tier}), \N{\ACC{hì}krr} \E{Moment, eine kurze Zeit} ($<$ \N{hì} + \N{krr} \E{Zeit}).
\index{hiì(')-@\textbf{hì(')-}}

\subsection{-ìva} Wenn das Substantiv \N{ìlva} \E{Flocke, Tropfen, Span, Splitter} in Kompositionen verwendet wird, fällt das \N{l} weg, \N{\ACC{txe}pìva} \E{Asche, Schlacke}, \N{\ACC{her}wìva} \E{Schneeflocke}.
\NTeri{1/4/2011}{https://naviteri.org/2011/04/yafkeykiri-plltxe-frapo-everyone-talks-about-the-weather/}
\index{-iva@\textbf{-ìva}}\index{ilva@\textbf{ìlva}}

\subsection{Munsna-} Das Wort \N{munsna} \E{Paar} wird als Präfix verwendet, um ein einzelnes Paar von etwas zu bezeichnen, insbesondere von Dingen, die von Natur aus paarweise vorkommen (mit Ausnahme von Körperteilen, wo der Dual üblich ist), wie z. B. \N{munsnahawnven} \E{ein Paar Schuhe}. Dieses Präfix wird nicht betont, \N{munsna\ACC{tu}te} \E{ein Paar von Leuten, ein Duo}.
\index{munsna-@\textbf{munsna-}}
\NTeri{19/7/2012}{https://naviteri.org/2012/07/fivospxiya-aylifyavi-amip-this-months-new-expressions-pt-2/}

\subsection{-nga'} Dieses Suffix, abgeleitet von dem Verb \N{nga'} \E{enthalten, beinhalten}, bildet Adjektive aus Substantiven und beschreibt etwas, das das Substantiv ``enthält'', wie in \N{\ACC{txum}nga'} \E{giftig}. Es ist viel seltener als \N{le-}. Es ist möglich, dass ein Substantiv sowohl \N{le-}- als auch \N{-nga'}-Ableitungen hat, \N{lepay} \E{wässrig, durchtränkt} vs. \N{\ACC{pay}nga'} \E{feucht, klamm}.
\index{-nga'@\textbf{-nga'}}
\NTeri{5/5/2011}{https://naviteri.org/2011/05/weather-part-2-and-a-bit-more-2/}

\subsection{-pin} Abgeleitet von dem Substantiv \N{'opin} \E{Farbe}, wird dieses unbetonte Suffix an Farbadjektive angehängt, um Farbsubstantive zu bilden, \N{\ACC{rim}pin} \E{die Farbe Gelb} von \N{rim} \E{gelb}. Ein auslautendes \N{-n} im Farbadjektiv wird durch phonetische Assimilation zu \N{-m}, \N{\ACC{e}ampin} von \N{\ACC{e}an}.\index{'opin@\textbf{'opin}}\index{-pin@\textbf{-pin}}

\subsection{Pxi-} Das Adjektiv \N{pxi} \E{scharf} wird als Präfix Zeitadverbien und -adpositionen vorange-stellt, um Unmittelbarkeit anzuzeigen. Das Präfix nimmt dabei keine Betonung an, \N{pxi\ACC{sre}} \E{direkt vor}, \N{pxi\ACC{set}} \E{jetzt gerade}.
\index{pxi-@\textbf{pxi-}}

\subsection{Sna-} Dieses Präfix ist eine verkürzte Form des Substantivs \N{sna'o} \E{Gruppe, Menge, Klumpen} und kann frei mit Lebewesen (außer Menschen) verwendet werden, um eine natürliche Gruppierung anzuzeigen, wie \N{snatalioang} \E{eine Herde von Sturmbestien}, \N{snautral} \E{ein Bestand an Bäumen}.
Das Präfix wird auch bei nicht belebten Dingen verwendet, um Wörter zu bilden, aber diese Verwendungsweise ist nicht produktiv, \N{snatxärem} \E{Skelett}.
\NTeri{31/3/2012}{https://naviteri.org/2012/03/spring-vocabulary-part-2/}
\index{sna-@\textbf{sna-}}

\subsection{-tseng} In der Umgangssprache kann das Substantiv \N{tseng} \E{Ort} als Suffix an Verben angehängt werden, um spontan Substantive zu bilden, die Orte bezeichnen, an denen bestimmte Handlungen stattfinden, wie etwa \N{yomtseng} für einen Ort, an dem gegessen wird. Diese Wörter werden sehr wahrscheinlich nicht lexikalisiert und ins Wörterbuch aufgenommen.
\NTeri{5/7/2023}{https://forum.learnnavi.org/language-updates/verb-tseng-compounds-in-informal-speech/}

In einigen ``offiziellen'' Komposita mit \N{tseng} können die anderen Bestandteile verschiedene phonetische Veränderungen aufweisen, wie z. B. \N{numtseng} \E{Schule}, in dem das Verb \N{nume} für das Kompositum modifiziert wurde.

\subsection{-tsim} Das Substantiv \N{tsim} \E{Quelle} kann als Suffix an Substantive angehängt werden, um die Quelle oder Ursache eines Zustandes anzuzeigen, wie z. B. \N{\ACC{sngum}tsim} \E{etwas Beunruhigendes, Ursache von Besorgnis} aus \N{sngum} \E{Sorge}, \N{ya\ACC{yayr}tsim} \E{etwas Verwirrendes, Ursache von Verwirrung} aus \N{yayayr} \E{Verwirrung}, \N{\ACC{ing}yentsim} \E{Mysterium, Rätsel} aus \N{ingyen} \E{Gefühl von Unverständnis, Rätselhaftigkeit, Nichtverstehen}. Man beachte, dass die Betonung nicht verändert wird und dieses Suffix nicht produktiv ist.
\NTeri{25/1/2013}{https://naviteri.org/2013/01/awvea-posti-zisita-amip-first-post-of-the-new-year/}
\index{-tsim@\textbf{-tsim}}

\subsection{Tsuk-} Abgeleitet von \N{tsun fko}, bildet dieses unbetonte Präfix Adjektive der Fähigkeit aus transitiven Verben, \N{tsuk\ACC{yom}} \E{essbar} aus \N{yom} \E{essen}. Die Verneinung gebraucht einfach das Präfix \N{ke-}, das auch hier keine Betonungsveränderung bewirkt, \N{ketsuk\ACC{tswa'}} \E{unvergesslich} aus \N{tswa'} \E{vergessen}.
\index{tsuk-@\textbf{tsuk-}}\index{ketsuk-@\textbf{ketsuk-}}
\NTeri{22/3/2011}{https://naviteri.org/2011/03/\%E2\%80\%9Creceptive-ability\%E2\%80\%9D-and-hesitation/}

\subsubsection{} Darüber hinaus können intransitive Verben mit \N{tsuk-} kombiniert werden, wobei die Beziehung zwischen dem Substantiv und dem resultierenden Adjektiv loser ist, \Npawl{fìtseng lu tsuktsurokx} \E{man kann sich hier ausruhen, dieser Ort ist ``erholsam''}, \Npawl{lu na'rìng tsukhahaw} \E{man kann im Wald schlafen}.

\subsection{-tswo} Dieses Suffix kann frei auf jedes Verb angewandt werden und bildet ein Substantiv, das die Fähigkeit bezeichnet, die Handlung des Verbs auszuführen, \N{wemtswo} \E{Fähigkeit zu kämpfen}, \N{roltswo} \E{Fähigkeit zu singen}. Dieses Suffix ist verwandt mit dem Wort \N{tsu'o} \E{Fähigkeit}.
\NTeri{31/3/2012}{https://naviteri.org/2012/03/spring-vocabulary-part-2/}
\index{-tswo@\textbf{-tswo}}

\subsubsection{} Das Suffix \N{-tswo} wird an das Substantiv- oder Adjektivelement von \N{si}-Verben angehängt, wie in \N{srungtswo} \E{Fähigkeit zu helfen} und \N{tstutswo} \E{Fähigkeit zu schließen}.

\subsection{-vi} Abgeleitet von dem Substantiv \N{'evi}, selbst eine verkürzte Form von \N{'eveng} \E{Kind}, wird das unbetonte Suffix \N{-vi} eher lose für die Abspaltung von etwas Größerem oder für den Teil eines größeren Ganzen verwendet, \N{\ACC{txep}vi} \E{Funke} ($<$ \N{txep} \E{Feuer}), \N{\ACC{lì'}fyavi} \E{Ausdruck, Sprachteil} ($<$ \N{lì'fya} \E{Sprache}). Es kann geringfügige Änderungen an dem Wort, an das es angehängt ist, verursachen: \N{sä\ACC{num}vi} \E{Lektion} von \N{sä\ACC{nu}me} \E{Unterweisung, Lehre}.
\index{-vi@\textbf{-vi}}
\LNWiki{14/3/2010}{https://wiki.learnnavi.org/index.php/Canon/2010/March-June\%23A_Collection}

\subsection{Kä- und za-} Die beiden Verben der Bewegung \N{kä} \E{gehen} und \N{za'u} \E{kommen} (reduziert auf \N{za-}) werden in einigen zusammengesetzten Verben als Präfixe verwendet, um die Bewegungsrichtung anzugeben, \N{kä\ACC{mak}to} \E{ausreiten}. Man beachte die Unterscheidung zwischen \N{kä\ACC{'ä}rìp} \E{drücken} und \N{za\ACC{'ä}rìp} \E{ziehen} von \N{\ACC{'ä}rìp} \E{etwas bewegen}.
\index{kaä-@\textbf{kä-}}\index{za-@\textbf{za-}}

\section{Zeit}
\noindent Adverbien der Zeit werden nach einem regelmäßigen Muster von Substantiven abgeleitet.

\subsection{Die momentane Zeit} Das Nominalpräfix \N{fì-} (\horenref{morph:prenoun:fi}) bildet ein Adverb für die aktuelle Zeiteinheit, \N{fìtrr} \E{heute} (``dieser Tag''), \N{fìrewon} \E{heute Morgen}. \index{fiì-@\textbf{fì-}!in Zeitadverbien}

\subsection{Die vorherige Zeit} Das betonte Suffix \N{-am} bildet ein Adverb für die vorherige Zeiteinheit, \N{trr\ACC{am}} \E{gestern}, \N{pxiswaw\ACC{am}} \E{vor einem Moment}.
\index{-am@\textbf{-am}}

\subsection{Die nachherige Zeit} Das betonte Suffix \N{-ay} bildet ein Adverb für die nächste Zeiteinheit, \N{trr\ACC{ay}} \E{morgen}, \N{ha'ngir\ACC{ay}} \E{morgen Nachmittag}.
\index{-ay@\textbf{-ay}}
