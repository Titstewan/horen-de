\nchapter{Pragmatik und Diskurs}

\noindent In früheren Kapiteln wurde über Laute, Wörter und Sätze in Na'vi gesprochen. Ein Großteil dieser Diskussion fand in Form von Regeln statt. Dieses Kapitel ist der Sprachebene gewidmet, die über der Syntax, d. h. der Satzebene, steht -- Konversation, Erzählung und die Kontexte, in denen Sprache stattfindet, was Linguisten unter dem Begriff ``Pragmatik'' zusammenfassen. Eindeutige Regeln sind hier schwieriger zu finden, sodass die Diskussion notwendigerweise einer etwas anderen Struktur folgt.


\section{Reihenfolge der Konstituenten}

\subsection{Freie Wortstellung} Na'vi wurde als Sprache mit freier Wortstellung beschrieben. Dies ist ein wenig irreführend, da dieser Ausdruck aus linguistischer Perspektive sehr spezifisch ist. Vielmehr hat Na'vi eine freie Stellung der Konstituenten.\footnote{Eine Konstituente ist ein etwas größerer ``Sprachbaustein'' als ein Wort, aber kleiner als ein Satz. Der Begriff beschreibt eine Gruppe von Wörtern, die als eine einzige grammatikalische Einheit fungieren. In dem Satz ``der große böse Wolf fraß Rotkäppchens Großmutter'' ist beispielsweise die Phrase ``der große böse Wolf'' eine Konstituente, die als Subjekt fungiert, das Verb ``fressen'' steht für sich allein und ``Rotkäppchens Großmutter'' konstituiert das direkte Objekt. Manchmal kann eine Konstituente ein einzelnes Wort sein (``er fraß sie'' -- jede Konstituente besteht aus einem Wort), und manchmal kann sie sehr viel komplexer sein (\href{https://de.wikipedia.org/wiki/Konstituente}{vgl. auch Wikipedia: Konstituente}).} Innerhalb von Konstituenten kann die Wortfolge recht eingeschränkt sein; man kann nicht einen Teil einer Konstituente in die Mitte einer anderen verschieben. Beispielsweise kann in dem Satz \N{ayoel tarmaron tsawla yerikit} \E{wir waren dabei, einen großen Hexapoden zu jagen} die Konstituente des direkten Objekts, \N{tsawla yerikit}, nicht aufgelöst werden; ein Satz wie \N{*tarmaron tsawla ayoel yerikit} oder \N{*ayoel tsawla tarmaron yerikit} ist ungrammatisch.\index{Wortstellung}

\subsubsection{} In komplexen Konstituenten ist es möglich, dass ein Genitivattribut von seinem Substantiv durch einen Relativsatz getrennt wird, 

\begin{interlin}
	\glll ngeyä teri faytele a aysänumeri \\
	nga-yä teri fì-ay-txele a ay-sänume-ri \\
	2\I{sg-gen} über dies-\I{pl}-Angelegenheit \I{rel} \I{pl}-Anweisung-\I{top}\\
	\trans{bezüglich deiner Anweisungen über diese Angelegenheiten}
\end{interlin}

\subsection{SOV, SVO, VSO} Viele menschliche Sprachen lassen sich anhand ihrer Standard-Wortfolge für Subjekt (\I{s}), Verb (\I{v}) und direktes Objekt (\I{o}) kategorisieren. Englisch ist überwiegend eine \I{svo}-Sprache, Japanisch ist \I{sov}. Sprachen mit freier Wortstellung lassen sich nicht leicht in dieses System einordnen, obwohl es bei einigen Sprachen durchaus erwähnenswerte Tendenzen gibt.\footnote{In früheren Versionen dieser Grammatik stand an dieser Stelle ein Überblick über von Paul Frommer in zwei längeren Na'vi-Texten verwendete Wortfolgen. Dieser war nützlich, um darzulegen, dass prinzipiell jede Wortfolge möglich ist; er konnte gleichwohl nicht erklären, \textit{weshalb} diese oder jene Wortfolge im konkreten Fall gewählt wurde. Seit Juli 2023 liegen genauere Informationen vor, weshalb die Übersicht entfernt wurde.} \NTeri{19/3/2011}{https://naviteri.org/2011/03/word-order-and-case-marking-with-modals/}

Sprachen mit freier Wortfolge werden manchmal auch mit dem Begriff der ``Diskurs-Konfigu-rationalität'' beschrieben. Diese recht technische Bezeichnung meint einfach, dass weniger die Grammatik als vielmehr Eigenschaften des Diskurses die Wahl der Wortfolge bestimmen. In konkreten Diskurs-Situationen passt der Sprecher die Informationsstruktur seiner Aussagen fortlaufend an, sodass der Empfänger nicht nur versteht, welche Informationen neu und relevant sind, sondern auch, wie sie mit bereits gegebenen Informationen zusammenpassen. Ein sehr vereinfachtes Schema eines solchen Diskurses sieht wie folgt aus:

\begin{quotation}
	\noindent Satz 1: [\dots\space neu\textsubscript{1}]\\
	Satz 2: [alt\textsubscript{2}\dots\space neu\textsubscript{2}]\\
	Satz 3: [alt\textsubscript{2}\dots\space neu\textsubscript{3}] \dots
\end{quotation}
Das Schema macht deutlich, dass jede neu in den Diskurs eingebrachte Information im folgenden Satz bereits alte Information ist. Das Deutsche nutzt diverse Konstruktionen, um alte Information zu kennzeichnen; darunter den definiten Artikel sowie Passivkonstruktionen.

Na'vi tendiert dazu, alte Information zu markieren, indem es sie an den Satzanfang verschiebt. Es folgt eine Sammlung von Beispielen aus dem ersten \textit{Avatar}-Film:
\begin{center}
	\begin{tabular}{llc}
		\N{pot lonu} & {\small\E{lass ihn gehen}} & \I{ov}\\
		\N{tsaswiräti lonu} & {\small\E{lass diese Kreatur gehen}} & \I{ov}\\
		\N{fìswiräti ngal pelun molunge fìtsenge?} & {\small\E{warum hast du diese Kreatur hierher gebracht?}} & \I{osv}\\
		\N{pot tsun oe tspivang nìftue} & {\small\E{ich könnte ihn leicht töten}} & \I{ovs}\\
		\N{ikranti makto} & {\small\E{reite den Ikran}} & \I{ov}\\
		\N{(ma sempul,) ngati oel kin} & {\small\E{(Vater,) ich brauche dich}} & \I{osv}\\
		\N{(ma 'ite,) tskoti munge} & {\small\E{(Tochter,) nimm den Bogen}} & \I{ov}\\
		\N{Omatikayaru tìhawnu sivi} & {\small\E{beschütze das Volk}} & \I{ov}\\
		\N{fra'ut fkol skera'a} & {\small\E{alles wird zerstört}} & \I{osv}\\
		\N{fkol pole'un fì'ut} & {\small\E{dies wurde entschieden}} & \I{svo}
	\end{tabular}
\end{center}

Man beachte, dass bestimmte Wörter bzw. Konstituenten im Diskurs bereits aufgrund ihrer Bedeutung alte Information vermitteln. Konkret sind dies Personalpronomen (wie in \N{pot lonu}), Personennamen, Demonstrativpronomen (\N{fì'u}, \N{tsa'u}), sowie Substantive mit Demonstrativpräfixen (wie in \N{tsaswiräti lonu}); daher stehen sie in den Beispielen am Satzanfang.\footnote{Substantive mit Possessivpronomen werden oft ebenfalls als alte Information angesehen. Ein Beispiel dafür aus Frommers Blog: \N{ayngeyä aysìralpengit ngop nì'o'!}}

Die Übersetzungen der Beispiele verwenden für Substantive am Satzanfang den definiten Artikel, wie in \N{tskoti munge} \E{nimm den Bogen}, weil das Deutsche auf diese Weise alte Information markiert. Darüber hinaus sind es allerdings nicht nur Substantive im Patiens (d. h. in der Konstituente des direkten Objekts), die an den Satzanfang verschoben werden -- indirekte Objekte kann dies ebenfalls betreffen, wie in \N{Omatikayaru tìhawnu sivi}. In gleicher Weise kann ein mit einer Adposition versehenes Pronomen als alte Information am Satzanfang stehen, \Nfilm{eo ngenga kllkxem ohe} \E{ich stehe vor Ihnen} (\I{pp v s}).

Frommer hat betont, dass alte Information am Satzanfang zwar üblich, aber eine stilistische Entscheidung ist; d. h. alte Information an anderen Positionen im Satz ist nicht ungrammatisch. Aussagen in Diskursen enthalten oft mehrere alte Informationen, sodass die relevanteste davon für den Satzanfang ausgewählt werden kann. Wenn ein Satz sowohl ein Personalpronomen als auch ein definites Substantiv (v. a. ein mit \N{fì-} bzw. \N{tsa-} markiertes) beinhaltet, so wird das Substan-tiv mit höherer Wahrscheinlichkeit als das Pronomen an den Satzanfang gestellt (z. B. \N{fìswiräti ngal pelun molunge fìtsenge} und \Nfilm{fìketuwongti oel stìyeftxaw}). \NTeri{9/7/2023}{https://forum.learnnavi.org/language-updates/a-word-order-tendency-old-information-to-the-front/}

\subsubsection{} Neben dem definiten Artikel ist das Passiv in vielen menschlichen Sprachen eine Strategie, die Informationsstruktur von Diskursbeiträgen zu organisieren. Im Satz ``die Nonne wurde von einem Auto überfahren'' wird mitgeteilt, dass die Nonne das bedeutendste Element der Aussage ist, während das Fahrzeug weniger relevant ist.\footnote{Im Deutschen kann der Agens einer Passivkonstruktion sogar wegfallen: ``Die Nonne wurde überfahren.''} Na'vi hat kein Passiv, aber Frommer hat darauf hingewiesen, dass die Wortfolge \I{osv} -- mit dem Patiens am Satzanfang -- denselben Effekt hat (siehe aber \N{fko}, \horenref{syn:prn:fko}).

\subsection{Fokus} \index{Fokus}\index{Wortstellung!Fokus}\index{Pointe}
Sprachen, die ihre freie Wortstellung nicht für die Markierung syntaktischer Beziehungen verwenden, können sie nutzen, um andere Aspekte wie Stil, Betonung und Fokus anzuzeigen.
Eine von Frommer formulierte Regel über Wortstellung lautet: ``Das Ende des Satzes ist der Punkt, an dem die `Pointe' steht''. Das bedeutet, dass eine Konstituente tendenziell ans Ende der Aussage gesetzt wird, wenn sie neu ist oder betont werden soll (d. h. im Fokus steht).

\begin{interlin}
	\glll Fayvrrtep fìtsenge lu \uwave{kxanì}. \\
	fì-ay-vrrtep fìtsenge lu kxanì \\
	dies-\I{pl}-Dämon hier sein verboten \\
	\trans{Diese Dämonen sind hier \uwave{verboten}.}
\end{interlin}

\begin{interlin}
	\glll Nì'ul kame \uwave{tskxe}. \\
	nì'ul kame tskxe \\
	mehr sieht Stein \\
	\trans{Ein \uwave{Stein} sieht mehr.}
\end{interlin}

\noindent Man beachte insbesondere, wie Frommer die folgenden Sätze übersetzt:

\begin{quotation}
	\noindent\Npawl{Fkxilet a tsawfa poe ioi säpalmi ngolop \uwave{Va'rul}.}\\
	\indent\E{\uwave{Va'ru} ist diejenige, die die Halskette, die sie getragen hat, hergestellt hat.}\\
	
	\noindent\Npawl{Ke sunu ngar \uwave{teylu} srak? Txotsafya, tìng oeru \uwave{pumit ngeyä!}}\\
	\indent\E{Du magst kein \uwave{Teylu}? Wenn das so ist, gib mir \uwave{deins}!}
\end{quotation}
% https://naviteri.org/2011/08/new-vocabulary-clothing/
% https://naviteri.org/2022/12/trr-anawm-polaheiem-the-great-day-has-arrived/

\noindent Im Fokus und damit am Satzende stehen auch kontrastive Informationen, d. h. Korrekturen von im Diskurs vorangegangenen Aussagen:

\begin{quotation}
	\noindent\Npawl{Spaw oel futa Mo'atìl tsole'a Neytirit.}\\
	\indent\E{Ich glaube, Mo'at sah Neytiri.}\\
	
	\noindent\Npawl{Kehe. Tsole'a Neytirit \uwave{Eytukanìl}.}\\
	\indent\E{Nein. Derjenige, der Neytiri sah, war \uwave{Eytukan}.}
\end{quotation}

\noindent Das Deutsche markiert fokussierte Elemente nicht über die Wortstellung, sondern primär über die Betonung des entsprechenden Wortes: \E{\uline{Eytukan} sah Neytiri}.
\NTeri{19/3/2011}{https://naviteri.org/2011/03/word-order-and-case-marking-with-modals/}

% Add a note about the keng example in:  ???   2018jan02
% https://naviteri.org/2012/02/trr-asawnung-lefpom-happy-leap-day/

\subsection{Auswirkungen der Wortstellungen} \index{Wortstellung!Auswirkungen} Änderungen in der Wortstellung können manchmal zu Änderungen in der Grammatik führen. \label{pragma:woe}

\subsubsection{} \label{pragma:word-order-effects:modals} Wenn ein Satz so geordnet ist, dass ein Modalverb und das von ihm kontrollierte transitive Vollverb nebeneinander stehen, und dass Subjekt und direktes Objekt ebenfalls nebeneinander stehen, kann die Kombination aus Modalverb und Vollverb als ein einziges transitives Verb reanalysiert werden, z. B. hat \Npawl{oel teylut new yivom} \E{ich möchte Teylu essen} die erwartete, korrekte Kasusverwendung mit dem Subjekt des Modalverbs im Agens und dem direkten Objekt im Patiens (\horenref{syn:modals}). In einigen Wortfolgen ist ein Subjekt im Agens allerdings weniger akzeptiert. In abnehmender Reihenfolge der Akzeptabilität:

\begin{center}
\begin{tabular}{lr}
\N{\uwave{Oel} teylut new yivom.} & weitgehend akzeptabel\footnotemark \\
\N{Teylut \uwave{oel} new yivom.} & zu etwa 50\% akzeptabel \\
\N{New yivom teylut \uwave{oel}.} & zu etwa 30\% akzeptabel \\
\N{*New yivom oel teylut.} & völlig inakzeptabel 
\end{tabular}
\end{center}
\footnotetext[\value{footnote}]{Laut Frommer's Blog ``[...] in allen außer den formellsten Situationen.''}
\NTeri{19/3/2011}{https://naviteri.org/2011/03/word-order-and-case-marking-with-modals/}

% \QUAESTIO{Thetic vs. categorical statements.}

\section{Thema und Kommentar}
\label{pragma:topic-comment}\index{Kasus!Topik}\index{Topik}

\noindent Die Thema-Kommentar-Konstruktion\footnote{Anm. d. Ü.: Die ``Thema-Kommentar-Konstruktion'' ist sehr ähnlich zum Konzept der Thema-Rhema-Gliederung (\href{https://de.wikipedia.org/wiki/Thema-Rhema-Gliederung}{vgl. Wikipedia}). Der Einfachheit halber habe ich mich hier für die Begriffe ``Thema'' und ``Kommentar'' entschieden.} ist konzeptionell einfach: Das ``Thema'' (auch ``Topik'') gibt an, worauf sich der Rest des Satzes bezieht, und der ``Kommentar'' macht eine Aussage zu diesem Thema. Viele menschliche Sprachen organisieren den Diskurs stark anhand der Thema-Kommentar-Struktur; das Deutsche gehört nur sehr eingeschänkt dazu. Das kann es schwierig machen, für Thema-Kommentar-Strukturen angemessene Übersetzungen zu finden, die sowohl die Bedeutung des Originals wiedergeben als auch die Informationsstruktur deutlich machen. In diesem Abschnitt werden daher oft Präpositionalphrasen mit ``bezüglich'' für die Beispiele verwendet; dies ist gleichwohl eine umständliche Behelfslösung, die nur der Klarheit dient.

\subsection{Topik-Kasus} Im Na'vi können nur Substantive, Nominalphrasen und Pronomen das Thema bilden. Diese werden mit dem Topik-Kasus (\N{-ri, -ìri}) gekennzeichnet. Komplexere Themen können mit nominalisierten Sätzen gebildet werden (\horenref{syn:clause-nom}).

\subsection{Rolle des Topik-Kasus} \label{pragma:topical-role}
Das Topik kann insbesondere für diejenigen, die damit nicht vertraut sind, verwirrend sein, weil fast jede syntaktische Rolle in einem Satz als Topik hervorgehoben werden kann. Eine idiomatische Verwendung ist die Kennzeichnung von unveräußerlichem Besitz (\horenref{syn:topical:poss}). Aber man kann das Topik auch dort verwenden, wo man im Deutschen einfach ein direktes Objekt verwenden würde:

\begin{quotation}
\noindent\Npawl{Fayupxare layu aysngä’iyufpi, fte \uwave{lì'fyari awngeyä} fo tsìyevun nìftue nìltsansì nivume.}

\indent\E{Diese Nachrichten sind für Anfänger gedacht, damit sie \uwave{unsere Sprache} leicht und gut lernen können.} 
\end{quotation}

\noindent Auch die Beziehung zwischen Thema und Kommentar muss nicht unbedingt strengen syntaktischen Rollen entsprechen:

\begin{quotation}
\noindent\Npawl{Ma oeyä eylan, \uwave{faysänumviri} rutxe fì’ut tslivam: \dots}\\
\indent\E{Meine Freunde, \uwave{bezüglich des Unterrichts}, bitte versteht dies: \dots}

\medskip
\noindent\Npawl{\uwave{Ayngeyä sìpawmìri} kop fmayi fìtsenge tivìng sì’eyngit.}\\
\indent\E{\uwave{Bezüglich eurer Fragen}, will (ich) ebenfalls versuchen, hier Antworten (auf sie) zu geben.}
\end{quotation}

\subsubsection{} Ein Topik kann einen komplexen Satz einleiten und dabei sogar vor einer einleitenden Konjunktion stehen,

\begin{quotation}
\noindent\Npawl{\uwave{Fori} mawkrra fa renten ioi säpoli holum.}\\
\indent\E{Nachdem sie ihre Brillen aufgesetzt hatten, gingen sie weg.}
\end{quotation}
% https://naviteri.org/2011/08/new-vocabulary-clothing/

\subsubsection{} Gleichermaßen kann ein Topik sich auf mehrere Kommentare beziehen, 

\begin{quotation}
\noindent\Npawl{\uwave{Poeri} uniltìrantokxit tarmok a krr, lam stum nìayfo, slä lu 'a'awa tìketeng -- natkenong, \uwave{tsyokxìri} ke lu zekwä atsìng ki amrr.}

\E{{\uwave{Was sie betrifft}, so war \uwave{sie} beinahe wie sie, als \uwave{sie} einen Avatar bewohnte, aber es gab ein paar Unterschiede -- beispielsweise, \uwave{was ihre Hand betrifft}, so hatte diese nicht nur vier, sondern fünf Finger.}}
\end{quotation}

\subsection{Benutzung des Topik}
Jede menschliche Sprache hat ihre eigenen Regeln und Tendenzen, wann das Topik verwendet werden sollte. Zum jetzigen Zeitpunkt ist es etwas schwierig, für das Na'vi Regeln dafür aufzustellen, aber aus dem, was wir bisher gesehen haben, lassen sich Tendenzen ableiten. Erstens hat Frommer bisher nicht annähernd so häufig Thema-Kommentar-Konstruktionen verwendet, wie sie im Chinesischen oder Japanischen verwendet werden (beides Topik-orientierte Sprachen, wenn auch jede auf ihre Weise). Zweitens verwendet Frommer den Topik-Kasus nicht, um neue Themen zur Diskussion zu stellen; vielmehr beziehen sich Topiken auf aktuelle Angelegenheiten oder auf solche, die sich aus dem Diskurs leicht erschließen lassen.

Im Deutschen wird der definite Artikel \E{der, die, das} verwendet, um Informationen zu kennzeichnen, die bereits in den Diskurs eingeführt wurden, sowie Informationen, die aus dem Gesprächskontext heraus angenommen oder abgeleitet werden können. Im Satz ``Ich wollte \E{Avatar} sehen, aber die Schlange war zu lang'' wird der definite Artikel mit \E{Schlange} nicht deshalb verwendet, weil zuvor über Schlangen gesprochen wurde, sondern weil das Schlangestehen etwas ist, das man gewohnt ist, wenn man einen beliebten Film sieht. In Kommentaren zu einem Blogbeitrag\footnote{\href{https://naviteri.org/2010/08/20/}{A Na'vi Alphabet}, 20. August 2010} schreibt Frommer: \index{Kasus!Topik!für den deutschen definiten Artikel}

\begin{quote}Wenn die Nachricht jedoch unbestimmt ist, funktioniert der Topik-Kasus nicht so gut, da Themen in der Regel bestimmt sind. So kann \N{'upxareri ngaru pamrel soli trram} sicherlich \E{ich habe dir gestern DIE Nachricht geschrieben} bedeuten. Kann es aber auch \E{ich habe dir gestern EINE Nachricht geschrieben} bedeuten? Da es im Na'vi keine Artikel per se gibt und Substantive entweder bestimmt oder unbestimmt sein können, ist das wohl möglich. Aber irgendetwas daran geht mir gegen den Strich.
\end{quote}

\noindent Es scheint am besten zu sein, wirklich unbestimmte Topiken zu vermeiden.

\section{Register}

\subsection{Formelles Register} Im Na'vi gibt es zwei Möglichkeiten, zeremonielle oder formelle Sprache zu kennzeichnen: mit speziellen Pronomen (\horenref{morph:hon-pron}) und mit dem Verbalinfix \N{\INF{uy}} (\horenref{morph:verb:2nd-pos}).\index{Register!formell}

\begin{interlin}
\glll Muntxatul ngengeyä tuyok pesenget? \\
   muntxatu-l ngenga-yä t\INF{uy}ok pe-tsenge-t \\
   Ehepartner-\I{agt} \I{2form}-\I{gen} sein.in\INF{\I{form}} welches-Ort-\I{acc}\\
\trans{Wo ist Ihr Ehepartner?}\Ipawl{}
\end{interlin}

\noindent Es ist nicht notwendig, das formelle Infix \N{\INF{uy}} immer mit den formellen Pronomen zu verwenden, noch ist es erforderlich, die formellen Pronomen mit allen Verwendungen von \N{\INF{uy}} zu benutzen. Im formellsten Register werden beide verwendet; etwas weniger formell ist es, wenn nur eines von beidem verwendet wird.
\NTeri{28/2/2022}{http://naviteri.org/2022/02/lifyengteri-concerning-honorific-language/}

\subsubsection{} Die formellen Pronomen können in enger Abfolge verwendet werden, \N{Sìfmetokit emzola'u \uwave{ohel}. Ätxäle si tsnì livu \uwave{oheru} Uniltaron} \E{Ich habe die Prüfungen bestanden. Ich bitte respektvoll um die Traumjagd.}

\subsubsection{} Wie bei den Zeit- und Aspektmarkern ist es nicht notwendig, das Infix \N{\INF{uy}} zu wiederholen, sobald ein formeller Kontext hergestellt wurde.

\subsubsection{} Auch die Ernsthaftigkeit oder Aufrichtigkeit einer Aussage kann durch die Markierung sowohl von Pronomen als auch Verb ausgedrückt werden:

\begin{interlin}
\glll Faysulfätuä tìkangkem \uwave{oheru} meuia \uwave{luyu} nìngay. \\
      fì-ay-tsulfätu-ä tìkangkem ohe-ru meuia l\INF{uy}u nìngay \\
      dies-\I{pl}-Experte-\I{gen} Arbeit \I{1sg.form}-\I{dat} Ehre sein\INF{\I{form}} wirklich \\
\trans{Die Arbeit dieser Experten ist wirklich eine Ehre für mich.} \Ipawl{}
\end{interlin}

\subsection{Poetisches Register}

\subsubsection{} In Prosa steht das Topik am Anfang des Satzes oder unmittelbar nach einer Konjunktion (\horenref{syn!topical!word-order}). In der Poesie kann es später im Satz auftreten, \Npawl{\uwave{pxan} livu txo nì'aw oe \uwave{ngari} / tsakrr nga Na'viru yomtìyìng} \E{nur wenn ich deiner würdig bin, wirst du das Volk ernähren}.

\subsubsection{} Wenn in Prosa eine Adposition vor dem Substantiv oder der Nominalphrase steht, muss auch jeder Genitiv nach der Adposition stehen, wie in \N{fa oeyä tsyokx} oder \N{fa tsyokx oeyä} \E{mit meiner Hand}. In der Poesie kann der Genitiv auch vor der Adposition stehen, \N{oeyä fa tsyokx}. 
\LNWiki{17/3/2012}{https://wiki.learnnavi.org/index.php/Canon/2012/January-June\%23A_poetic_license_and_a_note_on_adposition_position}

\subsubsection{} In der Alltagssprache muss ein Modalverb vor seinem Vollverb stehen (\horenref{syn:modal-syntax}); in poetischer oder zeremonieller Sprache kann das Modalverb auf das Vollverb folgen.
\NTeri{3/19/2011}{https://naviteri.org/2011/03/word-order-and-case-marking-with-modals/}

\subsection{Informelles Register} Das umgangssprachliche Register zeigt sich meist in einer vereinfachten Grammatik oder einem abgekürzten Ausdruck.
\index{Register!informell}

\subsubsection{} Verben der Kognition können einen Nebensatz ohne Konjunktion einleiten.

\begin{quotation}
\noindent\E{Ich glaube, es war ein Fehler, dass er gegangen ist.}\\
\noindent\Npawl{\uwave{Spängaw oel futa} fwa po kolä lu kxeyey.}\\
\noindent Umgangssprachlich: \Npawl{\uwave{Spaw oe}, fwa po kolä längu kxeyey.}
\end{quotation}
\NTeri{5/4/2011}{https://naviteri.org/2011/04/\%E2\%80\%99a\%E2\%80\%99awa-li\%E2\%80\%99fyavi-amip\%E2\%80\%94a-few-new-expressions/}

\subsubsection{} In der Umgangssprache wird das reflexive Perfekt der \N{si}-Verben, \N{säpo\ACC{li}}, oft als \N{spo\ACC{li}} ausgesprochen.
\NTeri{3/8/2011}{https://naviteri.org/2011/08/new-vocabulary-clothing/}

\subsubsection{} Die Konjunktion \N{tìk} (\horenref{syn:tìk}) kann verwendet werden, wenn ein zweites Ereignis eine unmittelbare Folge des ersten ist. Dies kann einige Verwendungen des konditionalen \N{txo} ersetzen, wie in \Npawl{tsatxumit näk tìk terkup} \E{wenn du dieses Gift trinkst, wirst du sofort sterben}.
\NTeri{31/12/2021}{https://naviteri.org/2021/12/zolau-niprrte-ma-3746-welcome-2022/}

\subsubsection{} \label{prag:colloq:omit}
In der Umgangssprache können \N{lu}, \N{tok} und \N{pum} entfallen, wobei dies nicht vorgeschrieben ist. Auch im Riff-Na'vi entfällt oft \N{lu}. \NTeri{5/5/2023}{https://naviteri.org/2021/12/zolau-niprrte-ma-3746-welcome-2022/}
\index{lu@\textbf{lu}!informeller Entfall}
\index{tok@\textbf{tok}!informeller Entfall}
\index{pum@\textbf{pum}!informeller Entfall}

\begin{quotation}
\noindent Eher formell: \Npawl{Nga lu pesu?} \E{Wer bist du?}\\
\noindent Eher informell: \Npawl{Nga pesu?} \E{Wer bist du?}\\

\noindent Eher formell: \Npawl{Pol tok pesenget?} \E{Wo ist er?}\\
\noindent Eher informell: \Npawl{Pol pesenget?} \E{Wo ist er?}\\

\noindent Eher formell: \Npawl{Fìtsko lu pum oeyä.} \E{Dieser Bogen ist meiner.}\\
\noindent Eher informell: \Npawl{Fìtsko lu oeyä.} \E{Dieser Bogen ist meiner.}
\end{quotation}

\noindent Da \N{tok} transitiv ist, sind im Satz \N{pol pesenget?} weiterhin die Endungen für Agens und Patiens erforderlich. Das Verb \N{'efu} \E{fühlen} darf nicht auf diese Weise wegfallen.
\NTeri{30/4/2021}{https://naviteri.org/2021/04/mipa-ayliu-mipa-sioeykting-new-words-new-explanations/}
\LNForum{25/10/2022}{https://forum.learnnavi.org/index.php?msg=679687}

\subsection{Jargon} Auf Wunsch der Community gestattete Paul Frommer einige von der Standardgrammatik abweichende Konstruktionen, die von den Na'vi in informellen Kontexten als ``Jargon'' verwendet werden können, oder um Spaß mit der Sprache zu haben.
\index{Register!Jargon}\index{Jargon}

\subsubsection{}Die Affektinfixe \N{\INF{ei}} und \N{\INF{äng}}, die normalerweise in Verben verwendet werden, können in \N{srane} \E{ja} und \N{kehe} \E{nein} eingefügt werden, anstatt Adverbien zu verwenden, um eine Stimmung auszudrücken. Nach dem bei Verben üblichen Muster werden diese Infixe in die zweite Silbe eingefügt (\horenref{morph:verb:2nd-pos}), \N{sran\INF{äng}e}, \N{keh\INF{ei}e}.
\LNForum{19/04/2020}{https://forum.learnnavi.org/index.php?msg=670176}

\subsubsection{}Personennamen können mit \N{si} ein zusammengesetztes Verb bilden, um \E{tun wie X} auszudrücken, was eine Erweiterung des gegenwärtigen Bildungsmusters dieser Verben darstellt.
\LNForum{19/04/2020}{https://forum.learnnavi.org/index.php?msg=670176}

\subsection{Verkürztes Register} In einer militärischen Umgebung, in der es darauf ankommt, Aussagen kurz zu fassen, können bestimmte Aspekte der Grammatik geändert werden oder entfallen.\index{Register!militärisch}\index{Register!verkürzt}

\subsubsection{} In Nominalphrasen können Partizipien mit ihrem Substantiv ohne das attributive Affix \N{-a-} verwendet werden (\horenref{syn:part:attr}), \N{tìkan tawnatep} \E{Ziel verloren} (aus dem Videospiel).
\LNWiki{21/5/2010}{https://wiki.learnnavi.org/index.php/Canon/2010/March-June\%23Losing_and_registers}

\subsubsection{} Einige Possessivpronomen verlieren das finale \N{-ä}, siehe \horenref{morph:pron:gen-clipped}. Dies kann auch beiläufig, in nicht-militärischen Situationen, unter Freunden oder guten Bekannten verwendet werden.
\LNWiki{21/5/2010}{https://wiki.learnnavi.org/index.php/Canon/2010/March-June\%23Losing_and_registers}
