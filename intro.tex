\nchapter{Einführung}

Wir haben bislang noch keine offizielle Grammatik der Sprache der Na'vi, die von Paul Frommer geschrieben und von Lightstorm Entertainment oder 20th Century Fox abgesegnet wäre. Zum Zeitpunkt der Abfassung dieses Werks\footnote{Erstmals geschrieben im Juli 2010. Immer noch der Fall im Juli 2024.} scheint es unwahrscheinlich, dass wir bald eine bekommen. In Anbetracht dessen habe ich beschlossen, eine von mir verfasste Zusammenfassung der Grammatik in ein längeres, ausführliches Dokument zu überführen.

Wie jene Grammatikzusammenfassung wird dieses Werk Na'vi nicht lehren. Vielmehr soll es eine knappe und präzise Referenz zum aktuellsten Wissensstand über die Sprache darstellen. Es basiert auf sämtlichen Analysen, die in Monaten und Jahren nach der Veröffentlichung des Films stattgefunden haben, sowie auch auf jeder Kommunikation von Frommer, die bestimmte Aspekte der Sprache näher erläutern.

Ich stütze mich stark auf die Korpus- und Kanon-Wiki-Seiten auf LearnNavi.org, ohne deren Informationen dieses Werk nicht möglich gewesen wäre. Frommers eigener Blog stellt ebenfalls ausführliches Material bereit.\footnote{Ende Juni 2010, \url{https://naviteri.org}}

\section{Geschichte der Entschlüsselung}

Für Neulinge in der Na'vi-Sprache ist es wichtig zu verstehen, wie es dazu gekommen ist, dass wir über die Sprache der Na'vi das wissen, was wir wissen.

Die ersten Hinweise zur Sprache tauchten in Interviews mit Frommer bereits im Dezember 2009 auf, bevor der Film veröffentlicht wurde. Na'vi verfügt über Ejektive. Es ist eine Ergativ-Akkusativ-Sprache. Wir hatten wenige Sätze.

Der große Durchbruch kam, als jemand, der die IMDB verließ, die Liste der Na'vi-Wörter\footnote{\url{https://kcbluesman.websitetoolbox.com/post?id=4013403}, Login benötigt} in einem eigenen Forum veröffentlicht hatte. Diese wurde aus dem \textit{Activist Survival Guide}\footnote{Wilhelm, Maria; Mathison, Dirk (2009). \textit{James Cameron's Avatar: A Confidential Report on the Biological and Social History of Pandora (An Activist Survival Guide),} It Books (HarperCollins).} transkribiert und erneut in einem Blog-Eintrag am 11. Dezember\footnote{\url{http://www.suburbandestiny.com/?p=611}} veröffentlicht. Alle aktuellen Wörterbücher basieren auf diesem ersten Beitrag. So hatten wir nun genügend Vokabeln, um die Sätze, die in Interviews mit Frommer auftauchten, zu analysieren.

Am 15. Dezember erschien in einem Interview mit dem UGO-Filmblog\footnote{\url{https://web.archive.org/web/20100818193039/http://www.ugo.com/movies/paul-frommer-interview}} erstmals die grundlegende Grußformel der Na'vi, \N{oel ngati kameie} \E{Ich sehe dich}. Hier sind uns auch zum ersten Mal die Endungen für den Agens- und den Patiens-Fall begegnet. Dank des Wörterbuches konnten wir \N{-l} für den Agens- und \N{-ti} für den Patiens-Fall ermitteln.

Unser nächster großer Durchbruch kam ein paar Tage später mit dem Gast-Blogbeitrag auf Language Log\footnote{\url{https://languagelog.ldc.upenn.edu/nll/?p=1977}} am 19. Dezember. Dies ist immer noch eine Grundlagenlektüre für jeden Na'vi-Lerner. Darin erfuhren wir viel über das phonetische System des
Na'vi, und es verriet uns genug über die Grammatik, um uns bei Analysen von Beispielen aus späteren Interviews zu helfen.

Selbst jetzt stammt ein großer Teil des Wissens, das wir haben, nicht daher, dass Frommer uns beispielsweise direkt den Genitiv erklärte, sondern daher, dass er in einem Interview lediglich sagte, dass es einen Genitiv gebe, woraufhin man diese Information nutzte, um die  Na’vi-Sprach-beispiele zu analysieren. Einige der anfänglichen Analysen waren jedoch unvollständig, was zu einiger Verwirrung geführt hatte, insbesondere hinsichtlich der Fallendungen. Die allerersten Beispiele für den Genitiv enthielten alle die Endung \N{-yä}. Erst später ergaben sich Hinweise auf die Existenz der
Endung \N{-ä}. Noch immer gibt es ältere Dokumentationen, die nur die Endung \N{-yä} für den Genitiv aufführen.

In den darauffolgenden Monaten lieferte Frommer selbst umfangreichere Beispiele des Na'vi, von denen jedes sehr detailliert analysiert wurde, um so viele Informationen zur Grammatik zu gewinnen wie nur möglich. Ferner hat Frommer auch einige direkte Fragen zur Sprache beantwortet. Oftmals bestätigte er dabei die Vermutungen, die sich aus unseren Analysen ergaben, korrigierte aber auch fehlerhafte Annahmen und gab uns manchmal auch neue Informationen.

Ich habe so weit wie möglich versucht, sicherzustellen, dass alle Inhalte dieser Grammatik von Frommer selbst bestätigt wurden oder, falls dies nicht möglich war, ausreichend Beispiele aus Frommers eigenem Na'vi einzufügen, um den jeweiligen grammatikalischen Aspekt zu verdeut-lichen. Dennoch ist dieses Werk notwendigerweise als vorläufig zu betrachten. Es ist Frommers Vorrecht, diese Sprache nach seinen Vorstellungen davon, was notwendig ist, anzupassen und zu aktualisieren, etwaige Missverständnisse zu korrigieren, die bisher seiner Aufmerksamkeit entgangen waren und grammatikalische Lücken zu schließen, sobald er sich ihrer annimmt. Darüber hinaus müssen wir davon ausgehen, dass zukünftige \textit{Avatar}-Filme die Sprache auf eine unerwartete Art und Weise verändern werden; nicht nur, um Camerons Anforderungen an seine Filme zu entsprechen, sondern auch aufgrund von unvermeidlichen Änderungen, denen eine konstruierte Sprache unterworfen ist, wenn die Schauspieler diese im Film schließlich sprechen.


\section{Notation und Konventionen}

\textbf{Text} auf Na'vi ist fett, die deutschen Übersetzungen sind kursiv geschrieben: \N{fìfya} \E{so, auf diese Weise, daher}.

Wenn ein Beispiel für das Na'vi direkt und unverändert aus Interviews, E-Mails oder From-mers Blog stammt, steht am Rand ein $\mathcal{F}$-Zeichen wie in \Npawl{kìyevame}. Ebenfalls auf diese Weise sind das \textit{Jagdlied} und das \textit{Weberlied} aus dem \textit{Activist Survival Guide} markiert. Beispiele aus den Filmen sind mit den Zeichen $\mathcal{A}$ und $\mathcal{A}^2$ markiert.

Da die meisten mit dem Digraphen-System vertrauter sind, verwendet dieses Werk die Digraphen \N{ts} und \N{ng} anstelle der wissenschaftlichen Orthographie, die von Frommer entwickelt worden ist (\horenref{lands:cg}).

In Frommers Dokumentation für die Schauspieler wurde die Betonung dadurch angezeigt, dass die betonte Silbe unterstrichen worden war. Diese Grammatik führt diese Praxis fort, wie in \N{\ACC{tu}te} \E{Person} vs. \N{tu\ACC{te}} \E{Frau}. Um Verwechslungen mit Frommers Konvention bei der Markierung von Betonung zu vermeiden, benutzt dieses Dokument eine gewellte \uwave{Unterstreichung}, um die Aufmerksamkeit auf bestimmte Teile von Wörtern oder Sätzen zu lenken.

Zudem finden die üblichen Konventionen linguistischer Schreibung hier Anwendung. Hypo-thetische Beispiele oder solche, die einen wie auch immer gearteten Fehler enthalten, sind mit einem vorangestellten Stern markiert, wie in *\N{m'resh'tuyu}. Präfixe sind mit einem Bindestrich am Ende markiert, wie in \N{fì-}; Präfixe, die Lenisierung (\horenref{lands:lenition}) auslösen, sind mit einem Pluszeichen markiert, wie in \N{ay+}. Suffixe sind mit einem vorangestellten Bindestrich markiert, \N{-it}, und Infixe mit spitzen Klammern, \N{\INF{ol}}, dargestellt. In eckigen Klammern sind Transkriptionen im In-ternationalen Phonetischen Alphabet angegeben, [fɪ.ˈfja].

Wird eines der vier Lieder, die Frommer für den Film übersetzt hat, zitiert, so sind die einzel-nen Verse durch einen einzelnen Schrägstrich getrennt, wie bei \N{Rerol tengkrr kerä	/ Ìlä fya’o avol}.

Beginnend mit September 2011 werden für neues Material Links zu Zitaten über grammatika-lische Einzelheiten mit angegeben. Sie stehen am Ende eines Abschnitts und sehen wie folgt aus: \NTeri{11/7/2010}{https://naviteri.org/2010/07/diminutives-conversational-expressions/}. Zu beachten ist, dass die Datumsangaben dem europäischen Modell folgen, also Tag/Monat/Jahr. ``NT'' steht für Frommers Blog, inklusive seiner Antworten im Kommentarbe-reich, ``Wiki'' steht für die LN.org-Wiki, ``Forum'' ist das LN.org-Forum und ``Ultxa'' für das Treffen im Oktober 2010. Für einige Bereiche gibt es noch Lücken in den Zitaten, die nach und nach geschlossen werden.

Text \QUAESTIO{in Magenta} zeigt Thematiken auf, die meines Erachtens wichtige Fragen zur Sprache darstellen, für die bisher jedoch keine Informationen vorliegen. Einige bedürfen lediglich einer Bestätigung durch Frommer, während andere wesentlich genauere Überlegungen und
Arbeit seinerseits erfordern. Diese Grammatik strebt an, eines Tages frei von magentafarbenem Text zu sein.

\subsection{Lesen der Zwischenzeilen}
Diese Grammatik ging ursprünglich davon aus, dass jeder, der sie liest, bereits über Grundkenntnisse in Na'vi verfügt. Das ist heutzutage (2024) keine berechtigte Annahme mehr, daher habe ich begonnen, einige der Beispiele in Form von Zwischen-zeilen ausführlicher zu gestalten. Mit ihnen kann alles, was in einem Beispiel vor sich geht -- Grammatik, Vokabular, Morphologie -- auch jemandem erklärt werden, der mit der Na'vi-Sprache nicht vertraut ist. Das Format ist wie folgt:

\begin{interlin}
	\glll Oel ngati kameie. \\
	oe-l nga-ti kam‹ei›e \\
	\I{1sg-agt} \I{2sg-pat} sehen\INF{\I{pos.aff}} \\
	\trans{Ich sehe dich.}
\end{interlin}

\noindent Die erste Zeile enthält den ursprünglichen Na'vi-Text. Die zweite Zeile zeigt die Präfixe, Suffixe und Infixe. In der dritten Zeile wird jeder Teil erklärt, mit Nummern für die Pronomen (\I{1sg} = Pronomen der ersten Person Singular) und Abkürzungen für verschiedene Teile der Na'vi-Grammatik. In der letzten Zeile schließlich steht die herkömmliche Übersetzung. 

Die Abkürzungen sind etwas gewöhnungsbedürftig, halten die Beispiele aber kurz und sind daher langfristig leichter und verständlicher zu lesen.

\begin{multicols}{3}
	\noindent\I{s}: Subjekt, unmarkierte Form \\
	\I{agt}: Agens, \N{-(ì)l}-Form \\
	\I{pat}: Patiens, \N{-(i)t}-Form \\
	\I{dat}: Dativ, \N{-(u)r}-Form \\
	\I{gen}: Genitiv, \N{-(y)ä}-Form \\
	\I{top}: Topik-Fall, \N{-(ì)ri}-Form \\
	\I{voc}: Vokativpartikel \N{ma} \\
	\I{lig}: Adjektivligatur \N{-a-} \\
	\I{dim}: Diminutiv (\N{-tsyìp}) \\
	\I{rel}: Relativpartikel \N{a} \\
	\I{pfv}: Perfektiv \\
	\I{ipfv}: Imperfektiv \\
	\I{pst}: Vergangenheit \\
	\I{rem.pst}: ferne Vergangenheit \\
	\I{fut}: Zukunft \\
	\I{rem.fut}: ferne Zukunft \\
	\I{subj}: Subjunktiv \\
	\I{pos.aff}: positive Einstellung \\
	\I{neg.aff}: negative Einstellung \\
	\I{cerem}: Zeremoniell \\
	\I{infer}: Inferenz (Vermutung) \\
	\I{act.pcpl}: Partizip Aktiv \\
	\I{pass.pcpl}: Partizip Passiv \\
	\I{caus}: Kausativ \\
	\I{refl}: Reflexiv \\
	\I{quot}: Zitat \N{san...sìk}
\end{multicols}

\noindent Einige Na'vi-Affixe sind Infixe - sie werden in die Mitte eines Wortes gesetzt. Da das Deutsche dies nicht zulässt, stehen die Infixe in der grammatikalischen Erklärungszeile (Zeile 3) auf der rechten oder linken Seite des Wortes. Im obigen Beispiel steht das Infix für die positive Einstellung rechts vom Wort \E{sehen}, mit den kleinen Zeichen ‹ und › anstelle der für Präfixe und Suffixe verwendeten Bindestriche.

Die Umstellung alter Beispiele auf die Verwendung von Zwischenzeilen begann im Sommer 2022 und wird einige Zeit in Anspruch nehmen.

\section{The Way of Water (Der Weg des Wassers)}
Ein neuer Dialekt des Na'vi erscheint im zweiten Film \textit{Avatar: The Way of Water} (2022). Aus praktischen Gründen ist der Hauptdialekt dieser Grammatik das Wald-Na'vi (\N{Lì'fya Na'ringä}), die Sprache aus dem ersten Film und die Einzige, zu der Na'vi-Material seit über einem Jahrzehnt vorliegt. In \textit{The Way of Water} ist der neue Dialekt das Riff-Na'vi (\N{Lì'fya Wionä}).

Die Hauptunterschiede zwischen den beiden Dialekten betreffen die Lautung (Phonologie), die Formenlehre (Morphologie), den Satzbau (Syntax) und den Wortschatz (Lexik). Zu den phono-logischen Unterschieden gehören eine andere Aussprache der Ejektive, eine andere Behandlung des glottalen Plosivs zwischen Vokalen und eine Vokalverschmelzung im Wald-Na'vi, die im Riff-Na'vi nicht vollzogen wurde, das daher u und ù als separate Vokalphoneme beibehält. Ein Beispiel für einen lexikalischen Unterschied ist, dass Riff-Na'vi das Wort \N{syawm} \E{wissen} bevorzugt, wäh-rend Wald-Na'vi \N{omum} nutzt.

Weil wir so viel mehr Material über Wald- als über Riff-Na'vi haben, bezieht sich diese Gram-matik in den Dialekt-Anmerkungen hauptsächlich auf das Wald-Na'vi.\footnote{Dies ist keinesfalls so zu verstehen, dass Wald-Na'vi eine Art Standardsprache ist, oder korrekter wäre als alle anderen Dialekte.} Anmerkungen zu Riff-Na'vi kommen in der Grammatik vor, aber ich habe sie gründlich indiziert, um das Auffinden von Details zum Riff-Na'vi zu erleichtern.

\section{Anmerkung des Übersetzers}
Wie bei jeder Übersetzungsarbeit ist es nicht möglich, alle Aspekte und Feinheiten von einer in die andere Sprache zu übertragen. Ich habe mich, so gut wie es mir möglich war, am Original orientiert, und mich dabei ebenfalls von der ersten deutschen Übersetzung von \textit{Horen leNa'vi} (damals übersetzt von Olaf Martens) inspirieren lassen. Auch habe ich mir erlaubt, ein paar zusätzliche Fußnoten zu bestimmten Fachausdrücken oder schlicht als weiterführende Information (z. B. grammatikalische Besonderheiten oder Ergänzungen) hinzuzufügen. Dank geht an Tobi (Tsmuggah) dafür, dass er einen erheblichen Beitrag zur Korrektur dieser Übersetzung geleistet hat.

Verbesserungen oder Vorschläge sind natürlich immer willkommen! Diese können mir via Forum oder auf Discord (Beitrag im entsprechenden Bereich oder private Nachricht) mitgeteilt werden. Die Quelldateien sind auf Github verfügbar.

\vfill
Dank geht an die Mitglieder von LearnNavi.org 'Eylan Alfalulukanä, Taronyu und Ftiafpi dafür, dass sie die Entwürfe dieses Werks gegengelesen und Vorschläge gemacht haben. Ich habe nicht immer ihren Rat befolgt, daher liegt die Verantwortung für etwaige Schwächen bei mir.

Dank geht ferner an alle, die dieses Werk kommentiert und Korrekturen eingebracht haben, seitdem es erstmalig veröffentlicht wurde, sowie an alle, die Paul über die Jahre Fragen gestellt und seine Antworten veröffentlicht haben, damit alle von ihnen lernen können.

Schließlich vielen Dank an Paul Frommer, der weiterhin, seit über einem Jahrzehnt, unsere Grammatik- und Vokabelfragen beantwortet, soweit es ihm möglich ist.

\bigskip