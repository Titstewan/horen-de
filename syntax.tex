\nchapter{Satzbau}

\section{Transitivität und Ergativität}

\subsection{Transitivität} Na'vi kennzeichnet das Subjekt von transitiven und intransitiven Verben unterschiedlich. Um einen Na'vi-Satz mit einem Verb zu bilden, muss man verstehen, wie Transitivität funktioniert, d. h. für Na'vi ist ein tieferes Verständnis von Transitivität erforderlich als für das Erlernen vieler menschlicher Sprachen.\footnote{Im Gegensatz zu Sprechern des Englischen haben es Sprecher des Deutschen hier einfacher, da sich ein transitives Verb anhand des notwendigen Akkusativobjektes (das direkte Objekt) erkennen lässt. Im Englischen ist die Unterscheidung, ob ein Verb transitiv oder intransitiv ist, aufgrund fehlender eindeutiger Fälle ungleich schwieriger, was zusätzlich dadurch verkompliziert wird, dass es sehr oft nicht das Verb ist, das transitiv oder intransitiv ist, sondern die gesamte Satzkonstruktion. So ist ``I hunt'' (ich jage) intransitiv und ``I hunt the prey'' (ich jage die Beute) transitiv. Für das Na'vi ist es ebenfalls am besten, sich Transitivität als ein satz- statt als ein rein wortbasiertes Phänomen vorzustellen.}\index{Transitivität}

\subsubsection{} Viele Verbalkomposita entstehen dadurch, dass ein unflektiertes Substantiv, Adjektiv oder gelegentlich eine Interjektion mit dem Verbstamm \N{si} \E{tun, machen} gekoppelt wird, der nur in diesen Verbindungen verwendet wird, \N{irayo si} \E{danken}, \N{kavuk si} \E{verraten}. Diese Verben sind immer intransitiv und verwenden den Dativ für ihr Objekt
(\horenref{syn:case:dative-si}).

\subsubsection{} Verben mit dem Reflexivinfix \N{\INF{äp}} sind immer intransitiv, und  Verben mit dem Kausativ-infix \N{\INF{eyk}} sind immer transitiv.

\subsubsection{Reflexiv des Kausativ}
\label{reflexive-of-causative} \index{Reflexiv!eines Kausativs}
Verben, die sowohl im Kausativ als auch im Reflexiv stehen, werden mit der Infixgruppe \N{\INF{äp}\INF{eyk}} markiert. Die Transitivität dieser Formen ist komplexer, abhängig von der Transitivität des ursprünglichen Ausdrucks:

\begin{itemize*}
	\item Ist der ursprüngliche Ausdruck intransitiv, dominiert \N{\INF{äp}} und das Subjekt steht im Absolutiv.
	\item Ist der ursprüngliche Ausdruck transitiv, dominiert \N{\INF{eyk}} und das Subjekt steht im Agens.
\end{itemize*} 

\noindent Beispielsweise:

\begin{quotation}
	\noindent\Npawl{Oe täpeykaron.} \E{Ich bringe mich selbst dazu, zu jagen.}\\
	\noindent\Npawl{Oel täpeykaron yerikit.} \E{Ich bringe mich selbst dazu, einen Yerik zu jagen.}
\end{quotation}

\noindent Dieses Muster entspricht dem Verb ohne jegliche Änderung seiner Transitivität; \N{oe taron} verhält sich zu \N{oel taron yerikit} wie \N{oe täpeykaron} zu \N{oel täpeykaron yerikit}.
\LNForum{20/8/2021}{https://forum.learnnavi.org/index.php?msg=676894}

\subsection{Dreigeteiltheit} Na'vi markiert Substantive und Pronomen unterschiedlich, wenn sie das Subjekt eines intransitiven Satzes, das Subjekt eines transitiven Satzes oder das direkte Objekt eines transitiven Satzes sind (\horenref{syn:cases}).

\subsubsection{} Obwohl das deutsche Konzept des ``Subjekts'' eines Verbs im Na'vi in Abhängigkeit von der Transitivität der Verbalphrase zweigeteilt ist, gilt diese Unterteilung nicht für Partizipien. Es gibt ein adjektivisches Partizip Passiv und ein adjektivisches Partizip Aktiv, die für Subjekte sowohl im Absolutiv als auch im Agens verwendet werden (\horenref{morph:pre-first}).


\section{Nominalphrasen und Adjektive}

\subsection{Numerus} \QUAESTIO{Sind Kollektiva im Dual und Trial gegenüber dem Plural distributiv oder immer obligatorisch?}

\subsubsection{} Bei der Verwendung mit einem attributiven Zahlwort erhält das Substantiv keine Numerusmarkierung, \N{mrra zìsìt} \E{fünf Jahre}.
\index{Plural!unmarkiert mit Zahlwörtern}\LNWiki{18/6/2010}{https://wiki.learnnavi.org/index.php/Canon/2010/March-June\%23Numbers_take_nouns_in_the_SINGULAR.}

\subsubsection{} Die Mengenadjektive -- \N{'a'aw} \E{einige, mehrere}, \N{hol} \E{wenige}, \N{pxay} \E{viele}, \N{polpxay, holpxaype} \E{wie viele?} -- stehen in Adjektivphrasen ebenfalls mit Substantiven im Singular, \Npawl{lu poru \uwave{'a'awa 'eylan}} \E{er hat \uwave{mehrere Freunde}}.
\index{'a'aw@\textbf{'a'aw}}\index{hol@\textbf{hol}}\index{pxay@\textbf{pxay}}\index{polpxay@\textbf{polpxay}}\index{holpxaype@\textbf{holpxaype}}\NTeri{16/7/2010}{https://naviteri.org/2010/07/vocabulary-update/}

\subsubsection{} In der Umgangssprache kann die Anzahl mit dem Adjektiv \N{pxay} \E{viele} markiert werden: \Npawl{lu awngar \uwave{aytele apxay} a teri sa'u pivlltxe} \E{wir haben \uwave{viele Angelegenheiten} zu besprechen}.
\index{pxay@\textbf{pxay}!mit Substantiv im Plural}\\
\LNForum{16/7/2010}{https://forum.learnnavi.org/index.php?msg=123484}

\subsubsection{} Bei den Kopulaverben (\N{lu} und \N{slu}) lautet die Grundregel der Numerusmarkierung, dass bei Bezugnahme auf dieselbe Einheit der Numerus nur einmal pro Satz markiert werden darf.\label{syn:noun:concord}

\begin{quotation}
	\noindent\Npawl{Menga lu karyu.} \E{Ihr zwei seid Lehrer}.\\
	\noindent\Npawl{Fo lu karyu.} \E{Sie sind Lehrer}.\\
	\noindent\Npawl{Menga lu oeyä 'eylan.} \E{Ihr zwei seid meine Freunde}.
\end{quotation}
\index{lu@\textbf{lu}!Numeruskongruenz}\index{slu@\textbf{slu}!Numeruskongruenz}
\index{Plural!mit \textbf{lu} und \textbf{slu}}

\noindent In den ersten beiden Sätzen hat \N{karyu} keine Numerusmarkierung, da die Pronomen bereits markiert sind, und dasselbe gilt für \N{'eylan} im dritten Satz. Siehe aber \horenref{syn:pron:q-number} für das Interrogativpronomen \N{tupe}.
\NTeri{30/7/2011}{https://naviteri.org/2011/07/number-in-na\%E2\%80\%99vi/}

\subsubsection{} Bei allgemeinen Aussagen über eine Gruppe oder Klasse werden Substantive im Singular verwendet, \index{allgemeine Aussagen} 

\begin{interlin}
	\glll Nantangìl yom yerikit. \\
	nantang-ìl yom yerik-it \\
	Natterwolf-\I{agt} fressen Hexapede-\I{pat} \\
	\trans{Natterwölfe fressen Hexapeden.} \Ipawl{}
\end{interlin}

\NTeri{30/7/2011}{https://naviteri.org/2011/07/number-in-na\%E2\%80\%99vi/}

\subsection{Indefinit} Das Adjektiv \N{lahe} \E{weitere/r/s, andere/r/s} bedeutet \E{andere/r/s, i. S. v. `unterschiedlich'}, wenn es zusammen mit unbestimmten Substantiven und dem Suffix \N{-o} verwendet wird,

\begin{interlin}
	\glll Lu law \uwave{'uo} \uwave{alahe}, ma eylan. \\
	lu law 'u-o a-lahe, ma eylan \\
	sein gewiss Sache-irgendein \I{lig}-anderes, \I{voc} Freunde \\
	\trans{\uwave{Etwas anderes} ist gewiss, meine Freunde.} \Ipawl{}
\end{interlin}
\index{Indefinites Substantiv}\index{lahe@\textbf{lahe}!mit indefinitem Substantiv}

\subsection{Indefinite der freien Wahl} Na'vi verwendet das Adjektiv \N{ketsran} \E{egal (was), was (auch) immer} mit generischen Substantiven, um freie Wählbarkeit auszudrücken. Es kann wie ein Adjektiv verwendet werden, indem ein attributives \N{-a-} angehängt wird (Beispiele \ref{ketsran:ex01} und \ref{ketsran:ex02}), oder es kann eine Konjunktion sein (Beispiele \ref{ketsran:ex03} und \ref{ketsran:ex04}). Der Satz mit \N{ketsran} steht häufig, wenn auch nicht immer, im Subjunktiv:
\index{Indefinite!freie Wahl}\index{ketsran@\textbf{ketsran}}

\begin{interlin} \label{ketsran:ex01}
	\glll 'U aketsran tsun tivam. \\
	'u a-ketsran tsun t\INF{iv}am \\
	Sache \I{lig}-egal-was kann ausreichen\INF{\I{subj}} \\
	\trans{Irgendetwas wird in Ordnung sein.} \Ipawl{}
\end{interlin}

\begin{interlin} \label{ketsran:ex02}
	\glll Pukit aketsran ivinan. \\
	puk-it a-ketsran \INF{iv}inan \\
	Buch-\I{pat} \I{lig}-egal-was lesen\INF{\I{subj}} \\
	\trans{Lies irgendein Buch.} \Ipawl{}
\end{interlin}

\begin{interlin} \label{ketsran:ex03}
	\glll Ketsran tute nivew hivum, poru plltxe san rutxe 'ivì'awn. \\
	ketsran tute n\INF{iv}ew h\INF{iv}um, po-ru plltxe san rutxe '\INF{iv}ì'awn\\
	egal-was Person wollen\INF{\I{subj}} verlassen\INF{\I{subj}}, \I{3sg-dat} tell \I{quot} bitte bleiben\INF{\I{subj}} \\
	\trans{Egal wer gehen möchte, sag ihm, dass er bitte bleiben soll.} \Ipawl{}
\end{interlin}

\begin{interlin} \label{ketsran:ex04}
	\glll Ketsran tutel 'ivem, tsafnetsngan lu ftxìvä'. \\
	ketsran tute-l '\INF{iv}em, tsa-fne-tsngan lu ftxìvä' \\
	egal-was Person-\I{agt} kochen\INF{\I{subj}}, diese-Art-Fleisch sein eklig \\
	\trans{Diese Art von Fleisch ist eklig, egal wer es kocht.} \Ipawl{}
\end{interlin}
\noindent\NTeri{3/31/2013}{https://naviteri.org/2013/03/whoever-whatever-whenever/}

\subsection{Apposition} Substantive in einer Apposition\footnote{Ein Substantiv steht in einer Apposition, wenn es unmittelbar neben einem anderen Substantiv oder Pronomen steht und dieses näher beschreibt oder definiert (\href{https://de.wikipedia.org/wiki/Apposition}{vgl. Wikipedia: Apposition}). Wenn die Nominalphrase in der Apposition aus mehr als nur einem Substantiv besteht, wird sie im Deutschen durch Kommata abgetrennt.} zu anderen Substantiven stehen im Absolutiv, \Npawl{aylì'ufa \uwave{awngeyä 'eylanä a'ewan Markusì}} \E{in den Worten \uwave{unseres jungen Freundes, Markus}}. Dafür wird auch die Konjunktion \N{alu} verwendet (\horenref{syn:conj:alu}).\footnote{Die reine Apposition ist frühes Na’vi. Die Konjunktion \N{alu} zu verwenden dürfte sich im Hinblick auf den künftigen Gebrauch als sinnvoller erweisen.} \index{Apposition}

\subsubsection{Titel} Titel fungieren als Substantivmodifikatoren und werden daher nicht dekliniert, wenn sie mit Eigennamen verwendet werden. Der Dativ von \N{Karyu Pawl} ``Lehrer Paul'' ist \N{Karyu Pawlur}. \index{Kasus!mit Titeln}
% https://forum.learnnavi.org/language-updates/definitive-answers-on-compound-nouns/
% Followup and details: https://forum.learnnavi.org/language-updates/compound-nounsfinal-decision!/

\subsection{Adjektivische Attribution} Attributive Adjektive werden dem Substantiv, das sie beschreiben, mit dem Affix \N{-a-} zugeordnet (siehe \horenref{morph:adj-attr}), \Npawl{sìlpey oe, layu oeru ye'rìn \uwave{sìltsana fmawn}} \E{ich hoffe, ich werde bald gute Nachrichten haben}, \Npawl{lora aylì'u, lora aysäfpìl} \E{schöne Worte und schöne Gedanken}. \index{Adjektiv!attributiv}\label{syn:adj:attr}

\subsubsection{} Unabhängig von der Reihenfolge von Substantiv und Adjektiv werden die Kasusendungen immer an das Substantiv angehängt, niemals an das Adjektiv. Ebenso wird eine enklitische Adposition immer an das Substantiv angehängt
(\horenref{syn:adp:position}).

\subsubsection{} Wenn ein Adverb mit einem attributiven Adjektiv verwendet wird, darf es nicht zwischen dem Adjektiv und seinem Substantiv stehen, d. h. \Npawl{sìkenong ahìno nìhawng} \E{sehr detaillierte Beispiele} oder \N{nìhawng hìnoa sìkenong}, niemals etwas wie \N{*hìno nìhawnga sìkenong}.
\index{Adjektiv!attributiv!mit Adverb}

\subsubsection{} Wenn zwei Adjektive ein Substantiv beschreiben, tendiert Frommer dazu, sie in der Reihenfolge Adj. -- Subst. -- Adj. anzuordnen, \Npawl{nìawnomum tolel oel ta ayhapxìtu lì'fyaolo'ä \uwave{pxaya sìpawmit atxantsan}} \E{wie ihr wisst, habe ich \uwave{viele ausgezeichnete Fragen} von Mitgliedern der Sprachgemeinschaft erhalten}.

\subsubsection{} Bei mehr als zwei Adjektiven, oder wenn eine andere Reihenfolge als die oben angegebene (Adj. -- Subst. -- Adj.) verwendet werden soll, können die Adjektive mit \N{lu} in einen Relativsatz gesetzt werden:

\begin{interlin}
	\glll yayo a lu lor sì hì'i \\
	yayo a lu lor sì hì'i \\
	Vogel \I{rel} sein schön und klein \\
	\trans{ein kleiner, schöner Vogel (wörtl. ein Vogel, der klein und schön ist)} \Ipawl{}
\end{interlin}

\noindent Allerdings ist \N{yayo alor sì hì'i} (ohne \N{lu}) zulässig, wenn auch nicht bevorzugt. Dies ist viel wahrscheinlicher und akzeptabler, wenn ein drittes Adjektiv im Satz vorkommt:

\begin{interlin}
	\glll mrra yayo atsawl sì layon \\
	mrr-a yayo a-tsawl sì layon \\
	fünf-\I{lig} Vogel \I{lig}-groß und schwarz \\
	\trans{fünf große, schwarze Vögel} \Ipawl{}
\end{interlin}

\noindent Ohne \N{mrr-a} wäre \N{tsawla yayo alayon} vorzuziehen.
\Ultxa{2/10/2010}{https://wiki.learnnavi.org/index.php/Canon/2010/UltxaAyharyu\%C3\%A4\%23Multiple_Attributives}
\NTeri{28/2/2021}{https://naviteri.org/2021/02/aysipawm-si-aysieyng-questions-and-answers/\#comment-32798}

\subsubsection{} Ein einzelnes Adjektiv kann auf beiden Seiten des Substantivs wiederholt werden, um Intensität zu kennzeichnen. Das zweite Adjektiv erhält den Satzakzent, \Npawl{lu po lora tuté \uwave{alor}} \E{sie ist eine äußerst schöne Frau}.
\NTeri{2/28/2013}{https://naviteri.org/2013/02/vospxi-ayol-posti-apup-short-post-for-a-short-month/}

\subsubsection{} Bei der Wiederholung eines Substantivs mit verschiedenen Adjektiven (``der große\linebreak Hund, der kleine Hund, der kläffende Hund'', usw.) wird der Platzhalter\footnote{Frommer nennt ihn ein ``dummy noun'', betrachtet ihn also als Substantiv; man kann ihn aber durchaus auch als eine Art Pronomen verstehen.} \N{pum} anstelle des wiederholten Substantivs verwendet, \Npawl{lam set fwa Sawtute akawng holum, \uwave{pum asìltsan} 'ì'awn} \E{es scheint, dass jetzt, da die bösen Himmelsmenschen gegangen sind, \uwave{die guten übrig bleiben}}.\footnote{A. d. Ü.: Der Platzhalter lässt sich im Deutschen nicht nachbilden -- im Gegensatz zum Englischen (das über das Pronomen \E{one} verfügt) bleibt das Adjektiv im Nebensatz alleine stehen und wird nicht substantiviert.} Wenn es aber einen allgemeineren Begriff gibt, ist es eleganter, diesen zu verwenden. Als Antwort auf \Npawl{pol\-pxay\-a ta\-ron\-yu kelku si tsatsraymì?} \E{wie viele Jäger leben in diesem Dorf?} ist demnach die elegantere Antwort \N{tute amevol} \E{sechzehn Leute}, obwohl \N{pum amevol} durchaus akzeptabel ist.\label{syn:pum:adj} \index{pum@\textbf{pum}!mit attributiven Adjektiven}\LNForum{30/5/2020}{https://forum.learnnavi.org/index.php?msg=670718}

\subsubsection{} Das substantivische Element in den meisten \N{si}-Verben kann ein attributives Adjektiv haben, \N{wina uvan si} \E{ein schnelles Spiel spielen}.
\index{si-Konstruktion@\textbf{si}-Konstruktion!mit attributivem Adjektiv}
\LNForum{6/12/2013}{https://forum.learnnavi.org/index.php?msg=599196}

\subsection{Prädikation} Substantivische und adjektivische Prädikative verwenden beide die Konstruktion mit dem Verb \N{lu} \E{sein}, wie in \Npawl{\uwave{reltseotu atxantsan lu} nga} \E{du bist ein ausgezeichneter Künstler}, \Npawl{fìsyulang lu rim} \E{diese Blume ist gelb}.
\index{Adjektiv!Prädikation}\index{Substantiv!Prädikation}\label{syn:predicates}

\subsubsection{} Andere Verben, die ein Prädikativum haben können, sind \N{slu} \E{werden} und \N{'efu} \E{fühlen}, wie etwa in \N{ngenga slìyu Na'viyä hapxì} \E{du wirst Teil des Volkes werden}, \N{oe 'efu ohakx} \E{ich bin (fühle mich) hungrig}.
\index{slu@\textbf{slu}!Syntax mit Prädikativ}\index{'efu@\textbf{'efu}!Syntax mit Prädikativ}

\subsubsection{} Wenn es bei \N{slu} \E{werden} mehrdeutig ist, welche Konstituente das Subjekt und welche das Prädikativum ist, kann das Prädikativum mit der Adposition \N{ne} versehen werden, wie in \N{taronyu slu \uwave{ne tsamsiyu}} \E{der Jäger wird \uwave{ein Krieger}}.
\label{syn:predicate:slu-ne}\index{ne@\textbf{ne}! mit \textbf{slu}}\index{slu@\textbf{slu}!Syntax mit \textbf{ne}}\LNWiki{2/10/2010}{https://wiki.learnnavi.org/Canon/2010/UltxaAyharyuä\#Word_Order_with_Slu}

\subsubsection{} \N{Sleyku}, der Kausativ von \N{slu} \E{werden}, nimmt ebenfalls ein adjektivisches Prädikativum an, \Npawl{fula tsayun oeng pivängkxo ye'rìn ulte ngari oel mokrit stayawm, \uwave{oeti nitram sleyku nìtxan}} \E{\uwave{es macht mich sehr glücklich}, dass wir beide bald plaudern können und dass ich deine Stimme hören werde}. \QUAESTIO{Wie sieht es mit \N{'eykefu} aus?}
\index{sleyku@\textbf{sleyku}!Syntax mit Prädikativ}

\subsection{Vergleich} Komparative und Superlative von Adjektiven (\E{groß, größer, am größten}) werden mit der Partikel \N{to} markiert, die wie eine Adposition vor dem Substantiv stehen kann, mit dem verglichen wird, oder als Suffix an dieses angehängt wird (\horenref{lands:stress:enclisis}),
\index{to@\textbf{to}}\index{Adjektiv!Komparativ}
\index{Vergleich!von Adjektiven}

\begin{quotation}
	\noindent\N{Oe \uwave{to nga} lu koak.} \E{Ich bin älter \uwave{als du}}.\\
	\noindent\N{Oe \uwave{ngato} lu koak.} \E{Ich bin älter \uwave{als du}}.
\end{quotation}

\subsubsection{} Der Superlativ wird mit \N{\ACC{fra}to} \E{als alle} (wörtl. \E{``im Vergleich zu allen''}) gebildet, \Npawl{fìsyulang arim lu hì'i frato} \E{diese gelbe Blume ist die kleinste (von allen)}. \index{frato@\textbf{frato}}
\index{Adjektiv!Superlativ}

\subsubsection{} Vergleiche, die auf Gemeinsamkeiten basieren, ``so groß wie ein Baum'', werden mit dem Idiom `\N{nìftxan} -- Adj. -- \N{na} -- Substantiv oder Pronomen' formuliert, wie in \Npawl{oe lu nìftxan sìltsan na nga} \E{ich bin genauso gut wie du.} Wenn der Vergleichspunkt ein Pronomen oder ein definites Substantiv ist, das bereits Teil des Diskurses ist, kann der Topik-Fall verwendet werden, \Npawl{ngari lu oe nìftxan sìltsan}. Diese Konstruktion kann auch mit Adverben verwendet werden. \label{syntax:adj-eql-comp}
\index{Adjektiv!Gemeinsamkeit}\index{nìftxan@\textbf{nìftxan}}\index{Kasus!Topik!Vergleichspunkt}\index{na@\textbf{na}!Vergleichspunkt}\LNWiki{1/12/2010}{https://wiki.learnnavi.org/index.php/Canon/2010/October-December\#As_ADJ.2FADV_as_N.2FPRN}
%% CITE: https://wiki.learnnavi.org/index.php/Canon/2010/October-December#As_ADJ.2FADV_as_N.2FPRN


\subsection{Direkte Anrede}
\index{Vokativ}\index{direkte Anrede}\index{ma@\textbf{ma}}
Spricht man eine Person direkt an, so wird die Vokativpartikel \N{ma} unmittelbar dem entsprechenden Substantiv, Personennamen oder der Nominalphrase vorangestellt: \Npawl{oel ayngati kameie, ma oeyä eylan} \E{ich sehe euch, meine Freunde,} \Nfilm{ma Tsu'tey, kempe si nga?} \E{Tsu'tey, was tust du?}.

\subsubsection{} Bei Entscheidungsfragen folgt der Vokativ auf das finale \N{srak}, wie in \N{ngaru lu fpom srak, ma Txewì?} \E{geht es dir gut, Txewì?}. Und obwohl der Vokativ meist am Anfang oder Ende eines Satzes steht, kann er auch innerhalb eines Satzes vorkommen, \N{nga, ma Neytiri, plltxe nìltsan} \E{du, Neytiri, sprichst gut}.
\LNWiki{26/2/2018}{https://wiki.learnnavi.org/Canon/2018}

\subsubsection{} Wenn mehrere Personen angesprochen werden, wird \N{ma} nicht wiederholt, \N{ma smukan sì smuke} \E{Brüder und Schwestern}.

\subsubsection{}
\index{Vokativ!bei Tieren}\index{direkte Anrede!bei Tieren}
\index{ma@\textbf{ma}!bei Tieren}
Der Vokativ ist obligatorisch, wenn man mit Menschen (und Eywa) spricht, aber optional, wenn man zu Tieren spricht.
\LNForum{6/4/2010}{https://forum.learnnavi.org/index.php?msg=131570}

\subsubsection{} Kollektiva können das Suffix \N{-ya} annehmen, wie in \Nfilm{mawey, Na'viya, mawey} \E{ruhig, Leute, ruhig}.
\index{-ya@\textbf{-ya}!Vokativ}


\section{Pronomen}

%\subsection{Animacy}

\subsection{Geschlecht} Die geschlechtsspezifischen Pronomen der dritten Person, \N{poan} und \N{poe}, werden nur dann verwendet, wenn dies dazu beiträgt, Mehrdeutigkeiten im Diskurs zu vermeiden. Sprecher des Deutschen und anderer westeuropäischer Sprachen sollten darauf achten, sie nicht zu oft zu verwenden. \label{syn:pron:gender}

\subsection{Numerus} Die Formen des Interrogativpronomens \N{tupe} verhalten sich abweichend von den in \horenref{syn:noun:concord} besprochenen Regeln zur Numeruskongruenz. \label{syn:pron:q-number} Hier kann das Pronomen eine Numerusmarkierung erhalten, auch wenn das Substantiv ebenfalls markiert wurde. Man beachte die Antworten auf diese Fragen:

\begin{quotation}
	\noindent\Npawl{Tsaysamsiyu lu \uwave{tupe}?} \E{Wer sind jene Krieger?}\\
	\noindent\N{(Fo) lu 'eylan Tsu’teyä.} \E{Sie sind Tsu'teys Freunde.}\\
	
	\noindent\N{Tsaysamsiyu lu \uwave{supe}?} \E{Wer sind jene Krieger?}\\
	\noindent\N{(Fo) lu Kamun, Ralu, Ìstaw, sì Ateyo.} \E{Sie sind Kamun, Ralu, Ìstaw und Ateyo.}
\end{quotation}

\noindent Die Pluralformen fragen nach der Identität der einzelnen Personen, während der Singular nach einem Merkmal der gesamten Gruppe fragt.
% https://naviteri.org/2011/07/number-in-na’vi/

\subsection{Ähnlichkeit} Pronomen können das adverbiale Präfix \N{nì-} annehmen, wodurch Formen wie \N{nìnga} \E{wie du} entstehen. Diese Formen werden verwendet, um eine spezifische Art des Handelns zu bezeichnen, \Npawl{plltxe po nìayoeng} \E{sie spricht wie wir}. Um zu beschreiben, wie jemand wahrgenommen wird, werden die Adpositionen \N{na} oder \N{pxel} bevorzugt.
\LNForum{16/8/2016}{https://forum.learnnavi.org/language-updates/eapressions-of-'like-we'/}

\subsection{Fko} Das Indefinitronomen \N{fko} entspricht im Grunde dem deutschen ``man'', wie in \E{man sagt solche Dinge nicht}. \Npawl{Tsat ke tsun fko yivom} \E{das kann man nicht essen}, \Npawl{tsun fko sivar hänit fte payoangit stivä'nì} \E{man kann ein Netz benutzen, um einen Fisch zu fangen}.
\index{fko@\textbf{fko}}

\subsubsection{} \N{Fko} wird auch dort verwendet, wo das Deutsche bei allgemeinen Aussagen ein ``sie'' (bzw. alternativ ein ``man'') verwenden würde, wie in \Npawl{\uwave{plltxe fko} san ngaru lu mowan Txilte ulte poru nga} \E{\uwave{sie sagen / man sagt}, dass du Txilte magst und umgekehrt}.\footnote{Frommers Übersetzung lautet: \E{Ich höre, du magst Txilte und umgekehrt}.}

\subsubsection{} \N{Fko} kann mit dem Passiv übersetzt werden, wenn der Handelnde\footnote{Der Handelnde bzw. das Agens eines Verbs im Passiv ist die Person oder Sache, die in deutschen Passivkonstruktionen meist nach der Präposition ``von'' steht, wie in \E{ich wurde \uwave{vom Auto} angefahren}.} belebt ist, wie in der Redewendung \N{oeru syaw fko Wìlyìm} \E{mein Name ist William / ich werde William genannt}, \Npawl{tsalì'uri fko pamrel si fyape?} \E{wie wird dieses Wort geschrieben / wie schreibt man dieses Wort?}
\label{syn:prn:fko}
\index{Passiv!mit \textbf{fko}}\index{fko@\textbf{fko}!für deutsches Passiv}

\begin{interlin}
	\glll Fra'ut fkol skera'a. \\
	fra'u-t fko-l sk\INF{er}a'a \\
	alles-\I{pat} ``man''-\I{agt} zerstören\INF{\I{ipfv}} \\
	\trans{Alles wird zerstört.}
\end{interlin}

\subsection{Sno} \index{sno@\textbf{sno}}
Das Reflexivpronomen \N{sno} bezieht sich auf das Subjekt oder den Handelnden des Satzes, in dem es steht. Im Genitiv (\N{sneyä}) kann es mit \E{sein/ihr/deren eigene/r/s} übersetzt werden. Es wird verwendet, um die Doppeldeutigkeit von Sätzen wie ``er hat sein Essen zubereitet'' zu klären -- hier kann es sonst unklar sein, ob sich ``sein'' auf die Person bezieht, die das Essen zubereitet, oder auf jemand anderen:

\begin{quotation}
	\noindent\N{Pol 'olem peyä wutsot.} \E{Er hat sein (jemandes anderen) Essen zubereitet.}\\
	\noindent\N{Pol 'olem sneyä wutsot.} \E{Er hat sein (eigenes) Essen zubereitet.}
\end{quotation}

\noindent \N{Sno} kann sich auf ein Topik beziehen, das als Subjekt fungiert:

\begin{quotation}
	\noindent\N{Skxawngìri zìmup ulte sneyä tsko kxakx.} \E{Der Idiot fiel und zerbrach seinen Bogen.}
\end{quotation}

%\noindent If the topical is not the subject of the clause, \N{sno}
%will refer instead to the subject:
%
%\begin{quotation}
%\noindent\N{Skxawngìri sa'nok zìmup ulte sneyä tsko kxakx.}\\
%\indent\E{That idiot's mother fell and broke her (own) bow}.
%\end{quotation}
%\LNForum{27/11/2020}{https://forum.learnnavi.org/language-updates/sno-with-topical/}

\QUAESTIO{Es ist nach wie vor unklar, wie man mit \N{sneyä} usw. umgeht, wenn sowohl ein Topik als auch ein Subjekt im Spiel sind, sowie darüber, ob sich \N{sneyä} auf Dinge in einem Nebensatz beziehen kann, wie in \E{er denkt, dass \uwave{sein} Vater...}}

Man beachte, dass das Reflexivpronomen dem Substantiv, auf das es sich bezieht, zuvorkommen kann, wie in \Npawl{sìpawmìri sneyä aynumeyuä karyu 'eyng} \E{der Lehrer antwortet auf die Fragen seiner Schüler}. Hier steht \N{sneyä} vor \N{karyu}.
% https://naviteri.org/2020/05/tipusawm-tiuseyng-si-okvur-a-eltur-titxen-si-asking-answering-and-an-interesting-story/

\N{Sno} bezieht sich ausschließlich auf die dritte Person.
\LNWiki{23/1/2018}{https://wiki.learnnavi.org/Canon/2018}

\subsection{Lahe} Das Adjektiv \N{lahe} \E{andere/r/s, weitere/r/s} kann auch als Pronomen verwendet werden, \Nfilm{fìpoti oel tspìyang, fte tìkenong liyevu \uwave{aylaru}} \E{ich werde ihn als Lektion für die \uwave{Anderen} töten} (siehe \horenref{morph:lahe:short} für die Formen).
\index{lahe@\textbf{lahe}!als Pronomen}

\subsection{Kontrastierende Demonstrativa}\index{Demonstrativum!kontrastierend}
Um kontrastierende Elemente zu fokussieren, werden Formen der Präfixe \N{fì-} und \N{tsa-} mit Formen der unabhängigen Demonstrativa \N{fì'u} und \N{tsa'u} verbunden, die mit \N{alu} verwendet werden: \pagebreak
\index{alu@\textbf{alu}!mit kontrastierenden Demonstrativa}

\begin{quotation}
	\noindent\Npawl{Fìfkxen alu FÌ'u lu ftxìlor; tsafkxen\footnote{\normalfont Alternativ \N{pum}.} alu TSA'u ngati tspang.}\\
	\noindent\E{DIESES Gemüse ist lecker; JENES bringt dich um.}\\
	
	\noindent\Npawl{Fìkaryu alu fìpo lu tsulfätu; tsakaryu alu tsapo lu skxawng.}\\
	\noindent\E{Dieser Lehrer ist ein Meister; jener ist ein Dummkopf.}
\end{quotation}

\noindent Zusätzlich wird der Kontrast in dieser Konstruktion durch eine Betonung der unabhängigen Formen \N{\underline{fì}'u} und \N{\underline{tsa}'u} ausgedrückt.
\NTeri{31/12/2011}{https://naviteri.org/2011/12/one-more-for-2011/}

\section{Verwendung der Kasus}
\label{syn:cases}
\subsection{Absolutiv}\footnote{Anm. d. Ü.: Frommer bezeichnet diesen Fall als ``Subjektiv''; die deutsche linguistische Terminologie, der die vorliegende Übersetzung folgt, verwendet den Begriff ``Absolutiv'' (vgl. die Anmerkung zu \horenref{morph:decl}).} Der unmarkierte Kasus wird für das Subjekt intransitiver Sätze, das Prädikativ in Prädikatskonstruktionen (\horenref{syn:predicates}) sowie mit Adpositionen verwendet. \index{Kasus!Absolutiv}\index{Subjekt}

\subsubsection{} Bei Verben der Bewegung kann, wenn das Ziel unmittelbar nach dem Verb steht, die Adposition \N{ne} optional weggelassen werden, sodass ein unmarkiertes Substantiv übrig bleibt, \Npawl{za'u \uwave{fìtseng}, ma 'itetsyìp} \E{komm \uwave{hierher}, kleine Tochter}. \label{syn:subjective:ne}
\index{ne@\textbf{ne}!Entfall bei Verben der Bewegung}

\subsubsection{} Der Absolutiv wird auch in Ausrufen verwendet, wenn ein Substantiv oder eine Nominalphrase für sich selbst als Äußerung verwendet wird, \Npawl{lora aylì'u, lora aysäfpìl} \E{schöne Worte und schöne Gedanken}, \Npawl{aylì'u apawnlltxe nìltsan} \E{gut gesprochene Worte!}
\index{Kasus!Absolutiv!in Ausrufen}

\subsubsection{} Ein Zeitwort mit dem Indefinit \N{-o} wird im Absolutiv verwendet, um eine Zeitspanne anzugeben: \Nfilm{\uwave{zìsìto amrr} ftolia ohe} \E{ich studierte \uwave{fünf Jahre lang}}, \Npawl{herwì zereiup \uwave{fìtrro nìwotx}} \E{es hat \uwave{den ganzen Tag} geschneit}.
\index{-o@\textbf{-o}!für Zeitspannen}
\LNWiki{1/12/2010}{https://wiki.learnnavi.org/index.php/Canon/2010/October-December\#Duration_and_Loan_Word.2C_.22Jesus.22}

\subsection{Agens} Der Agens-Fall wird für das Subjekt transitiver Sätze verwendet, \N{\uwave{oel} ngati kameie} \E{ich sehe dich}.
\index{Kasus!Agens}\index{Subjekt}

\subsection{Patiens} Der Patiens-Fall wird für das direkte Objekt transitiver Sätze verwendet, \Npawl{\uwave{tì'eyngit} oel tolel a krr} \textit{wenn ich eine Antwort erhalte}.
\index{Kasus!Patiens}\index{direktes Objekt}

\subsection{Dativ} Der Dativ wird für das indirekte Objekt ditransitiver Verben verwendet, \Npawl{sìltsana fmawn a tsun oe \uwave{ayngaru} tivìng} \E{gute Nachrichten, die ich \uwave{euch} geben kann}.
\index{Kasus!Dativ}\index{indirektes Objekt}

\subsubsection{} Das Objekt eines \N{si}-Verbs steht im Dativ, \N{oe irayo si ngaru} \E{ich danke dir.} \label{syn:case:dative-si}

\subsubsection{} Das Ziel für den Kausativ eines transitiven Verbs kann im Dativ stehen, \N{oel \uwave{ngaru} tseykìye'a tsat} \E{ich werde dafür sorgen, dass \uwave{du} es siehst} (siehe \horenref{syn:trans-causative}). \index{Kasus!Dativ!mit Kausativ}

\subsubsection{} Das Verb \N{lu} bildet mit dem Dativ ein besitzanzeigendes Idiom, wo das Deutsche das Verb ``haben'' verwendet, \N{lu oeru ikran} \E{ich habe einen Ikran}. In dieser Konstruktion steht das Verb üblicherweise am Anfang des Satzes. \index{Kasus!Dativ!mit \textbf{lu}}
\index{lu@\textbf{lu}!mit Dativ}\LNWiki{28/1/2010}{https://wiki.learnnavi.org/index.php/Canon\%23Dative_.2B_copula_possessive}

\subsubsection{} Der Dativ des Interesses kann den Geltungsbereich eines Adjektivs auf die Beurteilung \QUAESTIO{oder den Nutzen} einer bestimmten Person beschränken, \N{fì'u oeru prrte' lu} \E{das ist mir angenehm}, \N{tìpängkxo ayoengeyä mowan lu oeru nìngay} \E{unser Gespräch ist wirklich angenehm für mich}.

\subsubsection{} Bei Verben des Sprechens, einschließlich eines Wortes wie \N{pawm} \E{fragen}, steht die angesprochene Person im Dativ, \N{oel poru polawm fì'ut} \E{ich habe ihn dies gefragt}.
\index{Kasus!Dativ!mit Verben des Sprechens}

\subsection{Genitiv} Der Genitiv markiert Besitz, wie in \N{oeyä 'eylan} \E{mein Freund}.\index{Kasus!Genitiv} Siehe aber unten für unveräußerlichen Besitz (\horenref{syn:topical:poss}).\index{Besitz}

\subsubsection{} Der Genitiv kann prädikativ verwendet werden, wie in \N{fìtseng lu awngeyä} \E{dieser Ort ist unser}. Der Platzhalter \N{pum} \E{die besessene Sache} wird jedoch häufiger verwendet, \Npawl{kelku ngeyä lu tsawl; \uwave{pum oeyä} lu hì’i} \E{dein Haus ist groß; \uwave{meines} ist klein}. \label{syn:pum:genitive}
\index{pum@\textbf{pum}!mit Genitiv}

\subsubsection{} Der \textit{genitivus partitivus} bezeichnet das größere Ganze, von dem etwas eine Teilmenge ist, \Nfilm{Na'\-vi\-yä luyu hapxì} \E{du bist ein Teil des Volkes}. Dies wird auch bei Brüchen verwendet, \Npawl{Tsu'teyìl tolìng oer mawlit \uwave{smarä}} \E{Tsu'tey gab mir die Hälfte \uwave{der Beute}}. 
\index{Kasus!Genitiv!Teilmenge}\label{syn:partitive-gen}

\subsubsection{} Der Genitiv wird gelegentlich von der dazugehörigen Nominalphrase getrennt, \Nfilm{Na'\-vi\-yä luyu hapxì} \E{du bist ein Teil des Volkes}.\index{Kasus!Genitiv!Trennung vom Substantiv}

\subsubsection{} Der Genitiv wird auch als Attribut deverbaler Substantive verwendet, wie in \Npawl{tì\-ftia ki\-fke\-yä} \E{Studium der natürlichen Welt}.

\subsection{Topik} Der Topik-Fall markiert das Thema in einer Thema-Kommentar-Konstruktion (siehe \horenref{pragma:topic-comment} für eine ausführlichere Erläuterung dazu).
Der Topik hat auch einige weitere feste Verwendungszwecke.\index{Kasus!Topik}

\subsubsection{}\index{Kasus!Topik!Syntax} \label{syn!topical!word-order}
\index{Kasus!Topik!mit \textbf{srake}}
In Prosa steht eine Nominalphrase im Topik-Fall so früh wie möglich im Satz: an erster Stelle in einem Hauptsatz, aber im Nebensatz oft nach der einleitenden Konjunktion. Wenn die Fragepartikel \N{srake} verwendet wird, kann das Topik entweder davor oder danach stehen.

\begin{interlin}
	\glll Srake ngari re'o tìsraw si? \\
	srake nga-ri re'o tìsraw si \\
	\I{q} 2\I{sg}-\I{top} Kopf Schmerz tun \\
	\trans{Tut dir der Kopf weh?}
\end{interlin}

\begin{interlin}
	\glll Ngari srake re'o tìsraw si? \\
	nga-ri srake re'o tìsraw si \\
	2\I{sg}-\I{top} \I{q} Kopf Schmerz tun \\
	\trans{Tut dir der Kopf weh?}
\end{interlin}

\noindent Auch wenn ein komplexer Satz mit einer Konjunktion eingeleitet wird, kann das Topik dieser vorausgehen,

\begin{interlin}
	\glll Fori mawkrra fa renten ioi säpoli holum. \\
	fo-ri mawkrra fa renten ioi s\INF{äp}\INF{ol}i h\INF{ol}um \\
	2\I{pl}-\I{top} nachdem mittels Brille Schmuck \INF{\I{refl}}tun\INF{\I{pfv}} weggehen\INF{\I{pfv}}\\
	\trans{Nachdem sie ihre Brillen aufgesetzt hatten, gingen sie weg.}
\end{interlin}
\LNWiki{8/10/2011}{https://wiki.learnnavi.org/index.php/Canon/2011/April-December\%23Topical_Position}
\LNForum{6/5/2022}{https://forum.learnnavi.org/language-updates/which-comes-first-srake-or-the-topical/}

\paragraph{} \index{Riff-Na'vi!Position des Topik}
\index{Kasus!Topik!Position im Riff-Na'vi}
Im Riff-Na'vi kann das Topik auch am Ende eines Satzes stehen, \N{irayo \uwave{fìstxeliri alor}} \E{danke für dieses schöne Geschenk}. Im Wald-Na'vi verlässt das Topik nur in Gedichten und Liedern die Anfangsposition. \Omaticon

\subsubsection{} Einige idiomatische Ausdrücke weisen besondere Verwendungen des Topik auf. Beispielsweise wird der Topik-Kasus oft mit dem \N{si}-Verb \N{irayo si} \E{danken} verwendet, um die Sache anzuzeigen, für die man sich bedankt, \Npawl{\uwave{tìmweypeyri ayngeyä} seiyi irayo nìngay} \E{ich danke euch wirklich \uwave{für eure Geduld}}. Diese Verben müssen aus dem Wörterbuch gelernt werden.

\subsubsection{}
\index{Besitz!unveräußerlich}\label{syn:topical:poss}
\index{Kasus!Topik!unveräußerlicher Besitz}
\index{Topik!unveräußerlicher Besitz}
Das Topik kann zur Kennzeichnung von unveräußerlichem Besitz\footnote{\href{https://de.wikipedia.org/wiki/Alienabilität}{Vgl. Wikipedia: Alienabilität}} verwendet werden. Zum unveräußerlichen Besitz zählen diejenigen Dinge, die einer Person von Natur aus gehören und die theoretisch nicht weggegeben oder genommen werden können (außer durch Beschädigung). In den meisten Sprachen, die unveräußerlichen Besitz kennen, gibt es für Körperteile und manchmal für Blutsverwandte häufig besondere grammatikalische Regeln. Im Na'vi wird der Topik-Fall für Verwandte üblicherweise nicht verwendet, aber die nachfolgenden Beispiele zeigen, dass Körperteile, der eigene Geist und das eigene Leben als unveräußerlich gelten.

\begin{quotation}
	\noindent\Npawl{\uwave{Oeri} nì'i'a \uwave{tsyokx} zoslolu.} \E{\uwave{Meine Hand} ist endlich geheilt.}\\
	\noindent\Npawl{\uwave{Oeri} tìngayìl \uwave{txe'lanit} tivakuk.} \E{Lass die Wahrheit \uwave{mein Herz} treffen.}\\
	\noindent\Nfilm{\uwave{Oeri} ta peyä fahew akewong \uwave{ontu} teya längu.}\\
	\indent\E{\uwave{Meine Nase} ist erfüllt von seinem Aliengestank.}\\
	\noindent\Nfilm{\uwave{Ngari} hu Eywa salew \uwave{tirea}, \uwave{tokx} 'ì'awn slu Na’viyä hapxì.}\\
	\indent\E{\uwave{Dein Geist} geht mit Eywa, \uwave{dein Körper} bleibt, um Teil des Volkes zu werden.}\\
	\noindent\Npawl{\uwave{Ngari tswintsyìp} sevin nìtxan lu nang!} \E{Was du für ein \uwave{schönes Zöpfchen} hast! }\\
	\noindent\Npawl{Tseiun oe pivlltxe san \uwave{oeri} lu \uwave{tìrey} sìltsan nìngay sìk.}\\
	\indent\E{Ich kann glücklicherweise sagen, dass \uwave{mein Leben} wirklich gut ist.}
\end{quotation}

\noindent Bei den meisten Beispielen ist zu beachten, dass das besessene Substantiv nicht unmittelbar neben dem Topik stehen muss. 
\NTeri{11/7/2010}{https://naviteri.org/2010/07/diminutives-conversational-expressions/}

Familienmitglieder werden im Na'vi grammatikalisch nicht als unveräußerlich betrachtet und stehen daher normalerweise nicht im besitzanzeigenden Topik. In einer Konversation, in der ein Themenwechsel erforderlich ist, kann es jedoch danach aussehen, etwa in einem Gesprächsausschnitt wie diesem:

\begin{quotation}
	\noindent A: \Npawl{Oeyä sempul lu kanu sì tstunwi.}\\
	\indent\E{Mein Vater ist klug und freundlich.}
	
	\noindent B: \Npawl{Oeri sempul längu skxawng.}\\
	\indent\E{\uwave{Mein} Vater ist ein Idiot.}
\end{quotation}

\noindent Hier wird das Topik nicht für unveräußerlichen Besitz benutzt, sondern um den Fokus auf die Situation von Sprecher B zu lenken.
\LNForum{13/5/2024}{https://forum.learnnavi.org/index.php?msg=698456}

\subsubsection{} Der Topik kann auch verwendet werden, um den Bezugspunkt eines auf Gemeinsamkeiten basierenden Vergleichs zu bezeichnen (siehe \horenref{syntax:adj-eql-comp}).

\subsection{Subjekt-Auslassung} \index{Subjekt!Auslassung}\label{subject-dropping}
Im Deutschen und in vielen anderen europäischen Sprachen kann das Subjekt wegfallen, solange es gleich bleibt:

\begin{quotation}
	\noindent Ich ging nach Hause und \_ ging ins Bett.
\end{quotation}

\noindent Anstelle des Unterstrichs wird dasselbe Subjekt, \E{ich}, verstanden.

Na'vi hat mindestens drei Fälle, die im weitesten Sinne als ``Subjekt'' verstanden werden können: Der Absolutiv, der Agens-Fall und manchmal sogar der Dativ können sich einem subjekt-ähnlichen Verhalten annähern, etwa in \N{fì'u sunu \uwave{oer}} \E{\uwave{ich} mag dies}.

Im Na'vi kann alles, was subjektähnlich ist, entfallen, solange der Kontext klar ist.

\begin{quotation}
	\noindent\N{Oeru lu upxare ulte \_ zene tivätxaw.} \E{Ich habe Nachrichten und \_ muss zurückkehren.}\\
	\noindent\N{Teylu sunu oer. Yom pxìm!} \E{Ich mag Teylu. (Ich) esse (es) oft!}
\end{quotation}

\begin{quotation}
	\noindent\Npawl{Fayupxaremì \uwave{oe} payängkxo teri horen lì’fyayä leNa’vi fpi sute a tsun srekrr tsat sivar. Ayngeyä sìpawmìri kop fmayi fìtsenge tivìng sì’eyngit.}
	
	\medskip
	\noindent\E{In diesen Mitteilungen werde \uwave{ich} für Leute, die sie bereits anwenden können, über die Regeln der Sprache der Na'vi sprechen. \uwave{(Ich)} werde auch versuchen, hier Antworten auf eure Fragen zu geben.}
\end{quotation}

\noindent Wenn eine Klarstellung erforderlich ist, wird einfach das entsprechende Pronomen verwendet. Beispielsweise könnte man im ersten Beispiel oben auch \N{oe zene tivätxaw} sagen. Man beachte auch im zweiten Beispiel, dass das ``Subjekt'' im ersten Satz im Dativ steht, \N{oer}, während im zweiten Satz das ausgelassene Subjekt im Agens stehen würde, \N{oel}.
\LNForum{25/10/2022}{https://forum.learnnavi.org/index.php?msg=679687}


\section{Adpositionen}\index{Adpositionen}
\noindent Na'vi-Adpositionen können mit Substantiven, Pronomen und Adverbien des Ortes und der Zeit verwendet werden. Über die Verwendungsmöglichkeiten und Bedeutungen einzelner Adpositionen gibt das Wörterbuch auf \href{https://learnnavi.org/navi-vocabulary/}{LearnNavi.org} oder Stefan Müllers \href{https://forum.learnnavi.org/index.php?msg=613941}{Annotated Dictionary} Aufschluss.

\subsection{Position} Adpositionen können an zwei Stellen stehen: Erstens können sie vor der gesamten Nominalphrase stehen, die sie modifizieren und werden dann als separate Wörter geschrieben, \Npawl{\uwave{ta} peyä fahew akewong} \E{\uwave{mit (von)} seinem Aliengestank}, \Nfilm{ngari \uwave{hu Eywa} salew tirea} \E{dein Geist geht \uwave{mit Eywa}}. Zweitens können sie enklitisch stehen, in diesem Fall werden sie immer an das Wort angehängt, \Npawl{fìtrr\uwave{mì} letsranten} \E{an diesem wichtigen Tag}, \Npawl{aylì'u\uwave{fa} awngeyä 'eylanä a'ewan} \E{in den Worten unseres jungen Freundes}, \Npawl{lala tsarel\uwave{mì} arusikx} \E{in diesem alten Film}.
\index{Adpositionen!Position}\label{syn:adp:position}

\subsection{Lenition}\index{Plural!Kurzform} Einige Adpositionen verursachen Lenition im folgenden Wort. In Wörterbüchern werden diese in der Regel als \E{adp.+} angegeben, wobei das Pluszeichen wie üblich die Lenisierung anzeigt.

\subsubsection{} Enklitische Adpositionen verursachen bei dem Wort, an das sie angehängt sind, keine Lenition, \N{mì hilvan} \E{in einem Fluss}, aber \N{kilvanmì}. Die Kombination \N{hilvanmì} kann nur \E{in Flüssen} bedeuten. Enklitische Adpositionen verursachen auch keine Lenition bei einem folgenden Wort, \N{fo kilvanmì kllkxem} \E{sie stehen in einem Fluss}, nicht \N{fo kilvanmì *hllkxem}.

Allerdings wird ein Wort, das unmittelbar auf eine nicht enklitische Adposition folgt, leniert. Es muss dabei nicht unbedingt das Wort sein, das von der Adposition modifiziert wird, \N{mì hivea trr} \E{am siebten Tag} (nicht \N{*mì kivea srr}).
\LNWiki{24/8/2010}{https://wiki.learnnavi.org/index.php/Canon/2010/July-September\%23Fmawno}

\subsubsection{} Da die Lenition allein auch für die kurze Pluralform (\horenref{morph:short-plural}) verwendet wird, kann es je nach Diskurskontext zu Unklarheiten beim Numerus kommen. Um ihn eindeutig zu bestimmen, wird die vollständige Pluralvorsilbe \N{ay+} verwendet; die lenierte Form ohne \N{ay+} sollte als Singular interpretiert werden.
\index{Plural!Kurzform!mit lenierenden Adpositionen}\label{syn:adp:short-plural}\NTeri{1/7/2010}{https://naviteri.org/2010/07/thoughts-on-ambiguity/}


%\subsection{Äo} \E{Below}.  \N{Äo Utral Aymokriyä} \E{under the Tree
	%of Voices}.
%\index{aäo@\textbf{äo}}\label{syn:adp:äo}
%
%\subsection{Eo} \E{Before, in front of} (place).  May be used
%metaphorically, \Nfilm{eo ayoeng lu txana tìkawng} \E{a great evil is
	%upon us,} \Npawl{tokx eo tokx} \E{face to face, in person}.
%\index{eo@\textbf{eo}}\label{syn:adp:eo}
%
%\subsection{Fa} \E{With, by means of, using.}  Do not confuse with
%\N{hu} (\horenref{syn:adp:hu}).
%\index{fa@\textbf{fa}}\label{syn:adp:fa}
%
%\subsubsection{} \N{Fa} may introduce words about to be quoted,
%\Npawl{\uwave{aylì'ufa} awngeyä 'eylanä a'ewan} \E{\uwave{in the
		%words} of our young friend}.
%
%\subsubsection{} \N{Fa} is also one way to express the causee when a
%transitive verb takes the causative (see \horenref{syn:trans-causative}).
%
%\subsection{Few} \E{across, (towards) the opposite side}.  Do not
%confuse with \N{ka} (\horenref{syn:adp:ka}).  \Npawl{Po spä few payfya
	%fte smarit sivutx} \E{he jumped across the stream to track his prey.}
%\index{few@\textbf{few}}\label{syn:adp:few}
%
%\subsection{Fkip} \E{Up among}
%\index{fkip@\textbf{fkip}}\label{syn:adp:fkip}
%
%\subsection{Fpi (+)} \E{For the benefit or sake of}.  Refers to
%people, \Npawl{fayupxare layu aysngä\-’i\-yufpi} \E{these messages will be
	%for beginners}, or inanimates, \Npawl{'uo a fpi rey'eng Eywa\-'e\-veng\-mì
	%’Rrtamì tsranten nìtxan awngaru nìwotx} \E{something that matters a
	%lot to all of us for the sake of The Balance of Life on both Pandora
	%and Earth}.
%\index{fpi@\textbf{fpi}}\label{syn:adp:fpi}
%
%\subsection{Ftu} \E{From} (direction).  This is used mostly with
%volitional verbs of motion, such as \N{kä}, \N{rikx}, etc.
%\Nfilm{ftu fìtseng zene hivum} \E{we have to get out of here}.
%\index{ftu@\textbf{ftu}}\label{syn:adp:ftu} 
%
%See also \N{ta} below.
%
%\subsection{Ftumfa} \E{Out of, from inside}. \Npawl{riti tswolayon
	%ftumfa slär} \E{the stingbat flew out of the cave;} \Npawl{reypay
	%skxirftumfa herum} \E{blood is coming out of (from inside of) the
	%wound}.
%\index{ftumfa@\textbf{ftumfa}}\label{syn:adp:ftumfa}
%\NTeri{25/4/2013}{https://naviteri.org/2013/04/wheres-the-bathroom-and-other-useful-things/}
%
%\subsection{Hu} \E{With}.  Of accompaniment only --- do not confuse
%with \N{fa} (\horenref{syn:adp:fa}).  \N{Tsun oe ngahu pivängkxo a
	%fì’u oeru prrte’ lu} \E{it is a pleasure to be able to chat with you.}
%\index{hu@\textbf{hu}}\label{syn:adp:hu}
%
%\subsection{Io} \E{Above}.  \Npawl{Kllkxayem fìtìkangkem oeyä rofa — ke
	%io — pum feyä} \E{this work of mine will stand beside — not above —
	%theirs}.
%\index{io@\textbf{io}}\label{syn:adp:io}
%
%\subsection{Ìlä (+)} \E{By, via, following}.  \Npawl{Rerol tengkrr kerä
	%/ Ìlä fya'o avol / Ne kxam\-tseng} \E{(we) sing while going via the
	%eight paths to the center}; \Npawl{ayfo solop ìlä hilvan fa uran}
%\E{they traveled along (up, down) the river by boat}.
%\index{ilä@\textbf{ìlä}}\label{syn:adp:ìlä}
%
%\subsubsection{} It also means \E{according to,} \Npawl{ìlä Feyral,
	%muntxa soli Ralu sì Newey nìwan mesrram} \E{according to Peyral,
	%Ralu and Newey were secretly married the day before yesterday.}
%\NTeri{5/7/2012}{https://naviteri.org/2012/07/meetings-waterfalls-and-more/}
%
%\subsection{Ka} \E{Across, covering}.  Do not confuse with
%\N{few} (\horenref{syn:adp:few}).
%\index{ka@\textbf{ka}}\label{syn:adp:ka}
%
%\subsection{Kam} \E{Ago}.  \Npawl{Tskot sngolä'i po sivar 'a'awa
	%trrkam} (or \N{kam trr a'a'aw}) \E{he started to use the bow several
	%days ago}.
%\index{kam@\textbf{kam}}\label{syn:adp:kam}
%\NTeri{24/9/2011}{https://naviteri.org/2011/09/miscellaneous-vocabulary/}
%
%\subsection{Kay} \E{From now (in the future)}.  \Npawl{Zaya'u Sawtute
	%fte awngati skiva'a kay zìsìt apxey} (or \N{pxeya zìsìtkay}) \E{the
	%Sky People will come to destroy us three years from now!}
%\index{kay@\textbf{kay}}\label{syn:adp:kay}
%\NTeri{24/9/2011}{https://naviteri.org/2011/09/miscellaneous-vocabulary/}
%
%\subsection{Kip} \E{Among}.  \Nfilm{Tivìran po ayoekip} \E{let her walk
	%among us}.
%\index{kip@\textbf{kip}}\label{syn:adp:kip}
%
%\subsection{Krrka} \E{During}. \Npawl{Krrka tsawlultxa
	%Uniltìrantokxolo’ä} \E{during the Avatar Community Meet-up}.
%\index{krrka@\textbf{krrka}}\label{syn:adp:krrka}
%
%\subsection{Kxamlä} \E{through (via the middle of)}.
%\Npawl{Palukanit tsole'a, yerik lopx hifwo kxamlä zeswa},
%\E{spotting a thanator, the hexapede panicked and escaped through the
	%grass.}
%\index{kxamlä@\textbf{kxamlä}}\label{syn:adp:kxamlä}
%
%\subsection{Lisre (+)} see \N{li}, \horenref{syn:li:sre}.
%
%\subsection{Lok} \E{Close to}.
%\index{lok@\textbf{lok}}\label{syn:adp:lok}
%
%\subsection{Luke} \E{Without}.  \Npawl{Luke pay, ke tsun ayoe tìreyti
	%fmival} \E{without water we cannot sustain life}. This may also be
%used with nominalized phrases (see \horenref{syn:rel:nom-adp}).
%\index{luke@\textbf{luke}}\label{syn:adp:luke}
%
%\subsection{Maw, Pximaw} \E{After} (time).  \Npawl{Maw hìkrr ayoe
	%tìyätxaw} \E{we will return after a short time}.
%\index{maw@\textbf{maw}}\label{syn:adp:maw}
%\index{pximaw@\textbf{pximaw}}\label{syn:adp:pximaw}
%
%\subsection{Mì (+)} \E{In, on}.  This indicates being in a location.  Motion
%inward or into is \N{nemfa}.
%\index{miì@\textbf{mì}}\label{syn:adp:mì}
%
%\subsubsection{} \N{Mì} describes location in or on the body,
%\N{aylì'u na ayskxe \uwave{mì te'lan}} \E{the words (are) like stones
	%\uwave{in my heart}} (from the film script), \N{\uwave{mì tal} ngeyä
	%prrnenä a sanhì lor nìtxan lu nang} \E{what pretty stars your baby
	%has \uwave{on his back}}.  It also describes location in (or ``on'')
%a planet, \Npawl{lì'fyari leNa'vi \uwave{'Rrtamì}, vay set 'almong a
	%fra'u zera'u ta ngrrpongu} \E{everything that has gone on with
	%(blossomed regarding) Na’vi until now \uwave{on Earth} has come from a
	%grassroots movement}.
%
%\subsubsection{} It can also be used in expressions for times that
%have duration, \Npawl{fìtrr\uwave{mì} letsranten} \E{on this important
	%day}.  An expression such as \N{*mì kxamtrr} \E{*at noon}, however,
%is not allowed, since that refers to a particular point in time.
%
%\noindent\LNForum{7/8/2014}{https://forum.learnnavi.org/language-updates/concerning-time-words-and-fimeu/}
%
%\subsubsection{} \QUAESTIO{How to explain this: \Npawl{law lu oeru fwa
		%nga \uwave{mì reltseo} nolume nìtxan!}  Restriction of scope, like
	%\N{mì sìrey}?}
%
%\subsubsection{} Other idioms with \N{mì}: \N{tì'efumì oeyä} \E{in my
	%opionion}. 
%
%\subsubsection{} Though the writing doesn't change, when followed by
%the plural prefix \N{ay+} the vowel \N{ì} is dropped.  \N{Mì ayhilvan}
%is pronounced as though *\N{mayhilvan} (\horenref{lands:elision-i}).
%
%\subsection{Mìkam} \E{Between}
%\index{miìkam@\textbf{mìkam}}\label{syn:adp:mìkam}
%
%\subsection{Mungwrr} \E{Except}
%\index{mungwrr@\textbf{mungwrr}}\label{syn:adp:mungwrr}
%
%\subsection{Na} \E{Like, as}.  \N{Aylì'u na ayskxe mì te'lan} \E{the
	%words are like stones in (my) heart}.
%\index{na@\textbf{na}}\label{syn:adp:na}
%
%\subsubsection{} \N{Na} is used to specify shades of colors,
%\Npawl{fìsyulang lu \uwave{ean na ta’leng}} or \Npawl{fìsyulang lu
	%\uwave{ta’lengna ean}} \E{this flower is (Na'vi-)skin-blue}. See
%\horenref{syn:attr:na} for attributive color phrases with \N{na}.
%
%\subsubsection{} \N{Na} is used to mark the point of comparison in
%comparisons of equality (see \horenref{syntax:adj-eql-comp}).
%
%\subsection{Ne} \E{To, towards} (direction).  This marks the
%destination in verbs of motion.  \Npawl{Terìran ayoe \uwave{ayngane}}
%\E{we are walking \uwave{your way}}.  Sometimes \N{ne} can be omitted
%(see \horenref{syn:subjective:ne}).
%
%\subsubsection{} Idioms with \N{ne}: \Npawl{ke \uwave{zasyup} lì'Ona
	%\uwave{ne} kxutu a mìfa fu a wrrpa} \E{The l'Ona will not
	%\uwave{perish to} the enemy within or the enemy without;}
%\Npawl{zola’u nìprrte’ ne pìlok Na’\-vi\-te\-ri} \E{welcome to the
	%Na'viteri blog}.
%\NTeri{29/3/2010}{https://wiki.learnnavi.org/index.php/Canon/2010/March-June\%23Intentional_Future_Details}
%
%\subsubsection{} \N{Ne} may be used to disambiguate the predicate of
%the verb \N{slu} \E{become} (\horenref{syn:predicate:slu-ne}).
%\index{ne@\textbf{ne}}\label{syn:adp:ne}
%
%\subsection{Nemfa} \E{Into}.  See also \N{mì} (\horenref{syn:adp:mì}).
%\index{nemfa@\textbf{nemfa}}\label{syn:adp:nemfa}
%
%\subsection{Nuä (+)} \E{beyond (at a distance)}.  Note the contrast
%with \N{few}, \Npawl{fo kelku si few 'ora} \E{they live across the
	%lake} (on the other side) vs.\ \Npawl{fo kelku si nuä ora} \E{they
	%live beyond the lake} (at a great distance and out of sight).
%\index{nuaä@\textbf{nuä}}\label{syn:adp:nuä}
%\NTeri{15/8/2011}{https://naviteri.org/2011/08/new-vocabulary-clothing/comment-page-1/\%23comment-986}
%
%\subsection{Pxaw} \E{Around}.  \N{Po pxaw txep srew} \E{he danced
	%around the fire}.
%\index{pxaw@\textbf{pxaw}}\label{syn:adp:pxaw}
%
%\subsection{Pxel} \E{Like, as.} \Npawl{Fwa sute pxel nga tsun oeyä
	%hì'ia tìngopit sivar fte pivlltxe nìlor fìtxan oeru teya si} \E{that
	%people like you are able to use my little creation to speak so
	%beautifully fills me with joy}.
%\index{pxel@\textbf{pxel}}\label{syn:adp:pxel}
%
%\subsection{Ro (+)} \E{At (locative only)}
%\index{ro@\textbf{ro}}\label{syn:adp:ro}
%
%\subsection{Rofa} \E{Beside, alongside}.  \Npawl{Kllkxayem fìtìkangkem
	%oeyä rofa — ke io — pum feyä} \E{this work of mine will stand beside —
	%not above — theirs;} \Npawl{maw sätswayon ayol ayoe kllpolä mì tayo a
	%lu rofa kilvan} \E{after a short flight we landed in a field beside
	%the river}.
%\index{rofa@\textbf{rofa}}\label{syn:adp:rofa}
%
%\subsection{Sìn} \E{On, onto}.  \Npawl{Aywayl yìm kifkeyä / 'Ìheyut avomrr
	%/ Sìn tireafya’o avol} \E{the songs bind the thirteen spirals of the
	%world onto the eight spirit paths}.
%\index{siìn@\textbf{sìn}}\label{syn:adp:sìn}
%
%\subsection{Sko (+)} \E{As, in the capacity of, in the role of}. 
%\Npawl{Sko Sahìk ke tsun oe mìftxele tsngivawvìk} \E{as Tsahik, I
	%cannot weep over this matter}.
%\index{sko@\textbf{sko}}\label{syn:adp:sko}
%\NTeri{31/3/2012}{https://naviteri.org/2012/03/spring-vocabulary-part-2/}
%
%\subsection{Sre (+), Pxisre (+)}  \E{Before (time)}
%\index{sre@\textbf{sre}}\label{syn:adp:sre}
%\index{pxisre@\textbf{pxisre}}\label{syn:adp:pxisre}
%
%\subsection{Ta} \E{From} (various uses). \Npawl{Oeri ta peyä fahew
	%akewong ontu teya längu} \E{my nose is full of (``from'') his alien
	%smell}.
%\index{ta@\textbf{ta}}\label{syn:adp:ta}
%
%\subsubsection{} \N{Ta} indicates land of origin, \Npawl{Markusì ta
	%Ngalwey} \E{Marcus from Galway}.
%
%\subsubsection{} Of time, \N{ta} means \E{since}, \Npawl{trr’ongta
	%txon’ongvay po tolìran} \E{he walked from dawn until dusk.}
%(Also see \N{takrra}, \horenref{syn:attr:takrra}.)
%
%\subsubsection{} Frommer often uses \N{ta Pawl} \E{from Paul} at the
%end of his email and blog posts.
%
%\subsubsection{} With transitive verbs \N{ta} is more likely to be
%used to indication motion than \N{ftu}, as in \Npawl{pot 'aku
	%fìtsengta} \E{get him out of here!}
%\NTeri{15/8/2011}{https://naviteri.org/2011/08/new-vocabulary-clothing/comment-page-1/\%23comment-994}
%
%
%\subsection{Takip} \E{From among}
%\index{takip@\textbf{takip}}\label{syn:adp:takip}
%
%\subsection{Tafkip} \E{From up among}
%\index{tafkip@\textbf{tafkip}}\label{syn:adp:tafkip}
%
%\subsection{Teri} \E{About, concerning}.  \Npawl{Fayupxaremì oe
	%payängkxo teri horen lì'fya\-yä leNa'vi} \E{in these messages I will
	%chat about the rules of the Na'vi language}.
%\index{teri@\textbf{teri}}\label{syn:adp:teri}
%
%\subsection{Uo} \E{Behind}
%\index{uo@\textbf{uo}}\label{syn:adp:uo}
%
%\subsection{Vay} \E{Up to, until}.  This may be used of both time and
%space, \Npawl{tsakrrvay, ayngeyä tìmwey\-pey\-ri irayo seiyi oe} \E{until
	%that time, I thank (you) for your patience}. \QUAESTIO{There's a line
	%from the video game with a local use.}
%\index{vay@\textbf{vay}}\label{syn:adp:vay}
%
%\subsubsection{} The phrase \N{vay set ke} means \E{not yet}.
%\index{vay@\textbf{vay}!\textbf{vay set ke}}
%
%\subsection{Wä (+)} \E{Against} (as in ``fight against'').
%\Npawl{Peyä tsatìpe'un a sweylu txo wivem ayoeng Omatikayawä lu fe'}
%\E{his decision that we should fight against the Omaticaya was a bad one}.
%\index{waä@\textbf{wä}}\label{syn:adp:wä}
%%https://naviteri.org/2011/07/txantsana-ultxa-mi-siatll-great-meeting-in-seattle/
%
%\subsection{Yoa} \E{In exchange for,} primarily used with verbs of
%trade and exchange, \Npawl{oel tolìng ngaru tsnganit yoa fkxen} \E{I
	%gave you meat in exchange for vegetables}. May be used with
%nominalized clauses, \Npawl{käsrolìn oel nikroit Peyralur yoa fwa po
	%rol oer} \E{I loaned Peyral a hair ornament in exchange for her
	%singing to me}.
%\index{yoa@\textbf{yoa}}\label{syn:adp:yoa}
%\NTeri{28/2/2014}{https://naviteri.org/2014/02/barter-and-exchange/}

\section{Adverben}
\index{Adverben}

\subsection{Maß und Menge} Adverbien des Maßes und der Menge folgen sehr oft dem Element, das sie modifizieren: \Npawl{'Rrtamì tsranten \uwave{nìtxan} awngaru \uwave{nìwotx}} \E{das bedeutet uns \uwave{allen} auf der Erde \uwave{sehr viel}}.

\subsubsection{} Ein sehr häufiges Muster bei prädikativen Adjektiven ist Adj. -- \N{lu} -- Adv., \Npawl{ngeyä lì'fya leNa'vi \uwave{txantsan lu nìngay}} \E{dein Na'vi ist wirklich ausgezeichnet}. In umgekehrter Reihenfolge begegnet in einem Dialog des ersten Films einmal das Muster Adv. -- \N{lu} -- Adj., \Nfilm{peyä menarisì \uwave{nìhawng lu hì'i}} \E{und seine Augen sind zu klein}.

\subsection{Mit Gerundien} Der verbale Ursprung des Gerundiums erlaubt, dass es auch ein Adverb annehmen kann,\footnote{Anm. d. Ü.: Dies funktioniert für englische Übersetzungen hervorragend, da das Englische selbst über ein Gerundium verfügt. Ins Deutsche lässt sich das Gerundium nicht so einfach übertragen und wird oftmals als Substantivierung dargestellt.} \Npawl{koren a'awve \uwave{tìruseyä 'awsiteng}} \E{die erste Regel \uwave{des Zusammenlebens}}. \index{Gerundium!mit Adverb}\label{syn:adverbs:gerund}

\subsection{Korrelativa} Die Verben \N{'ul} \E{zunehmen} und \N{nän} \E{abnehmen} werden idiomatisch als korrelative Adverbien verwendet, \N{'ul... 'ul} \E{je mehr ... desto mehr} und \N{nän... nän} \E{je weniger... desto weniger}.\index{Vergleich!mit Korrelativa}

\begin{quotation}
	\noindent\Npawl{'Ul tskxekeng si, 'ul fnan.}\\
	\noindent\E{Je mehr du übst, desto besser wirst du.}\\
	
	\noindent\Npawl{'Ul tute, 'ul tìngäzìk.}\\
	\noindent\E{Je mehr Leute, desto mehr Probleme.}\\
	
	\noindent\Npawl{Nän ftia, nän lu skxom a emza'u.}\\
	\noindent\E{Je weniger du lernst, desto schlechter sind deine Chancen, zu bestehen.}\\
	
	\noindent\Npawl{Nän yom kxamtrr, 'ul 'efu ohakx kaym.}\\
	\noindent\E{Je weniger du mittags isst, desto hungriger bist du am Abend.}
\end{quotation}
\noindent\NTeri{29/2/2012}{https://naviteri.org/2012/02/trr-asawnung-lefpom-happy-leap-day/}

\subsection{Fìtxan und nìftxan} Die beiden Adverbien \N{fìtxan} und \N{nìftxan} \E{so (sehr)} werden zusammen mit der Konjunktion \N{kuma} (\horenref{syn:attr:kuma}) für Konsekutivsätze verwendet,

\begin{quotation}
	\noindent\Npawl{Lu poe sevin nìftxan (\textrm{or} fìtxan) kuma yawne slolu oer.}\\
	\indent\E{Sie war so schön, dass ich mich in sie verliebte.}
\end{quotation}

\noindent In diesen Konstruktionen muss  \N{akum / kuma} direkt neben \N{fìtxan / nìftxan} stehen.
\NTeri{19/6/2012}{https://naviteri.org/2012/06/spring-vocabulary-part-3/}

\subsection{Keng} Das Adverb \N{keng} \E{selbst, sogar} wird verwendet, um unerwartete Information hervorzuheben, \N{yom teylut \uwave{keng oel}} \E{\uwave{sogar ich} esse Teylu.}
\index{keng@\textbf{keng}}
\LNWiki{31/12/2010}{https://wiki.learnnavi.org/index.php/Canon/2010/October-December\%23Keng}

\subsection{Li} Die Hauptbedeutung von \N{li} ist \E{schon, bereits}, \Npawl{tìkangkem li hasey lu} \E{die Arbeit ist bereits fertig.}
\index{li@\textbf{li}} 
\NTeri{20/2/2011}{https://naviteri.org/2011/02/new-vocabulary-part-2/}

\subsubsection{``bisher nicht''} Die Negation, \N{ke li}, bedeutet ``noch nicht, bisher nicht'' und verwendet doppelte Verneinung (\horenref{syn:neg:pleon}), \N{fo ke li ke polähem} \E{sie sind noch nicht angekommen}.
\index{bisher nicht}\index{ke@\textbf{ke}!\textbf{ke li}}\NTeri{4/9/2011}{https://naviteri.org/2011/09/\%E2\%80\%9Cby-the-way-what-are-you-reading\%E2\%80\%9D/comment-page-1/\%23comment-1092}

\subsubsection{Befehle} Bei Imperativen zeigt \N{li} besondere Dringlichkeit an: \Npawl{Ngal mi fìtsengit terok srak? Li kä!} \E{Du bist noch hier? Geh schon!} Mit \N{ko} (\horenref{syn:particle:ko}), \N{li ko} (betont auf \N{li}), bedeutet es ``nun, dann mal los'', oder ``lasst uns loslegen''.

\subsubsection{Zögerlichkeit} In Antworten vermittelt es ein etwas zögerliches ``Ja'', ähnlich wie das deutsche ``sozusagen'' oder ``nun ja'':

\begin{quotation}
	\noindent A: \Npawl{Nga mllte srak?} \E{Stimmst du zu?}\\
	\noindent B: \N{Li, slä\dots} \E{Nun ja, ich denke schon, aber\dots}.
\end{quotation}

\noindent Die Negation des \N{ke li} bedeutet in etwa ``nicht wirklich''. 

\subsubsection{Mit ``sre''} Wenn \N{li} mit der Adposition \N{sre} verwendet wird, bedeuten sie ``bis spätestens, aber nicht danach'', \Npawl{kem si li trraysre} \E{erledige das bis spätestens morgen}. \label{syn:li:sre} Wenn \N{sre} vor dem Substantiv steht, verbindet es sich mit \N{li} zu \N{\ACC{li}sre}, das wie \N{sre} lenisiert, \Npawl{kem si lisre srray} \E{erledige das bis spätestens morgen}.
\index{lisre@\textbf{lisre}}


\subsection{Nìwotx} Das Adverb \N{nìwotx} \E{in toto, im Ganzen, ganz} wird häufig mit Substantiven und Pronomen im Plural verwendet, um einen kollektiven Sinn zu vermitteln,

\begin{interlin}
	\glll Aysutel nìwotx new tìlayrot. \\
	ay-tute-l nìwotx new tìlayro-t \\
	\I{pl}-Person-\I{agt} in.toto wollen Freiheit-\I{pat} \\
	\trans{Alle Personen wollen Freiheit.} \Ipawl{}
\end{interlin}

\noindent\Npawl{Ayeylanur oeyä sì eylanur lì'fyayä leNa'vi \uwave{nìwotx}} \E{an \uwave{alle} meine Freunde und Freunde der Sprache der Na'vi}, \Npawl{tìfyawìntxuri oeyä perey aynga \uwave{nìwotx}} \E{ihr wartet \uwave{allesamt} auf meine Anleitung}.
\index{nìwotx@\textbf{nìwotx}}

\subsubsection{``Beide''} Mit dem Dual bedeutet \N{nìwotx} \E{beide}:

\begin{interlin}
	\glll Mefo nìwotx yolom. \\
	me-po nìwotx y\INF{ol}om \\
	\I{dual}-\I{3p} in.toto \I{\INF{pfv}}essen \\
	\trans{Sie haben beide gegessen.}
\end{interlin}
\index{nìwotx@\textbf{nìwotx}!mit Dual}\index{beide}
\NTeri{15/8/2011}{https://naviteri.org/2011/08/new-vocabulary-clothing/}

\subsection{Nìfya'o} Eine Nominalphrase, die \N{fya'o} beinhaltet, kann frei verwendet werden, um Adverbien der Art und Weise zu bilden.
Bei dieser Konstruktion wird die gesamte Nominalphrase adverbialisiert, nicht nur das Wort, das mit \N{nì-} präfigiert ist, \N{nì-[fya'o letrrtrr]} \E{auf gewöhnliche Art und Weise}, \Npawl{poe poltxe nìfya'o alaw} \E{sie sprach deutlich}.
\index{nìfya'o@\textbf{nìfya'o}}\label{syn:nifyao}

\subsubsection{} \N{Nìfya'o} kann auch Attribute annehmen, \Npawl{nìfya'o a hek} \E{auf seltsame Art und Weise}.

\subsubsection{} \QUAESTIO{Satzadverbien vs. \N{nìfya'o}-Formen?}

\subsection{``Kop'' und ``nìteng''} Sowohl \N{kop} als auch \N{nìteng} entsprechen dem deutschen Adverb \E{auch, ebenfalls}. \N{Kop} meint aber eher \E{darüber hinaus, zusätzlich}, während \N{nìteng} \E{in gleicher Weise} bedeutet. Man vergleiche \N{oel poleng kop poru tsa'ut} \E{darüber hinaus habe ich ihm das gesagt} mit \N{oel poleng nìteng poru tsa'ut} \E{das habe ich ihm ebenfalls gesagt}.
\index{kop@\textbf{kop}}
\index{nìteng@\textbf{nìteng}}

\subsubsection{} Sie können auch zusammen verwendet werden, \Npawl{furia nga lu nitram, lu oe kop nitram nìteng} \E{da du glücklich bist, bin auch ich ebenso glücklich}.
% https://naviteri.org/2011/05/weather-part-2-and-a-bit-more-2/comment-page-1/#comment-779

\subsection{Sunkesun} Das Adverb \N{\ACC{sun}kesun} \E{ob es dir gefällt oder nicht} ist eine verkürzte Form von \N{sunu ke sunu} mit dem Standardadressaten in der zweiten Person, \Npawl{sunkesun po slayu olo'eyktan} \E{ob es dir gefällt oder nicht, er wird Stammesanführer werden.}
\index{sunkesun@\textbf{sunkesun}}
\NTeri{6/6/2019}{https://naviteri.org/2019/06/50a-liu-amip-40-new-words/}

\subsubsection{} Man beachte, dass \N{sunkesun} nur direkt an den Hörer gerichtet werden kann, ansonsten muss eine \N{ftxey... fuke}-Konstruktion verwendet werden, \Npawl{pol vìyewng ayevengit fìha'ngir, ftxey sunu fuke} \E{er wird sich heute Nachmittag um die Kinder kümmern, ob er will oder nicht.}


\section{Aspekt und Zeitform}

\subsection{Die Rolle des Kontexts} Na'vi-Verben sind häufig nicht mit Tempus oder Aspekt markiert, sodass ein Verb ohne Infixe oder höchstens mit dem Subjunktivinfix \N{\INF{iv}} übrig bleibt. Ohne neue Information, wie z. B. ein Zeitadverb oder ein Umbruch im Diskurs, setzt ein unmarkiertes Verb die Zeitform und / oder den Aspekt des Verbs im vorherigen Satz fort.\index{Verb!unmarkiert}

\subsubsection{} Obwohl ein Nebensatz vor dem Hauptsatz stehen kann, bezieht er seinen Kontext hinsichtlich Tempus und Aspekt aus dem Hauptsatz, \Npawl{oel foru fìaylì'ut \uwave{tolìng} a krr, kxawm oe \uwave{harmahängaw}} \E{als ich ihm diese Worte \uwave{gesagt habe}, muss ich \uwave{geschlafen haben}}, \Npawl{tì'eyngit oel \uwave{tolel} a krr, ayngaru \uwave{payeng}} \E{wenn ich eine Antwort \uwave{erhalten habe}, \uwave{werde} ich es dich \uwave{wissen lassen}.}

\subsection{Das unmarkierte Verb} Die hinsichtlich Tempus und Aspekt unmarkierte Verbform hat zwei zusätzliche Funktionen: Erstens kann sie das Präsens anzeigen, \Npawl{ayngaru seiyi irayo} \E{ich danke euch}; zweitens wird sie für gewohnheitsmäßige oder allgemeine Aussagen verwendet, \Npawl{nga za'u fìtseng pxìm srak?} \E{kommst du oft hierher?}, \N{lu fo lehrrap} \E{sie sind gefährlich}.
\index{Verb!Präsens}\index{Verb!unmarkiert}

\subsection{Aspekt} Im Allgemeinen wird im Na'vi der Aspekt gegenüber der Zeitform bevorzugt markiert.\footnote{Die Verbkategorie ``Aspekt'' kann für Sprecher des Deutschen und vieler anderer europäischer Sprachen ungewohnt sein, da diese in ihren Verbformen Tempus und Aspekt oft vermischen, was es schwierig macht, die beiden Begriffe zu unterscheiden. Während das Tempus den Zeitpunkt einer Handlung relativ zum Zeitpunkt der Aussage verortet (d. h. in der Vergangenheit, Gegenwart oder Zukunft), drückt der Aspekt aus, wie sich die Handlung relativ zum betrachteten Zeitraum erstreckt (ob dieser in der Vergangenheit, Gegenwart oder Zukunft liegt, ist dabei irrelevant), und gibt dabei z. B. an, ob die Handlung im betrachteten Zeitraum abgeschlossen wird oder nicht, ob sie einmalig, wiederholt oder gewöhnlich stattfindet, etc. (\href{https://de.wikipedia.org/wiki/Aspekt_(Linguistik)}{vgl. Wikipedia: Aspekt}). Ein deutsches Beispiel zur Verdeutlichung:
	\begin{quotation}
		\noindent  1. Ich ging in den Laden.  (Perfekt)\\
		2.  Als ich in den Laden ging (Imperfekt), sah ich etwas ganz Erstaunliches. (Perfekt)
	\end{quotation}
	In beiden Sätzen liegt die Handlung des Einkaufens in der Vergangenheit, aber im zweiten Satz wird sie im Verlauf, d. h. unabgeschlossen (imperfektiv) geschildert und bildet damit den Hintergrund für die folgende Aussage, die abgeschlossen (perfektiv) ist.} Es ist hilfreich, sich das Perfekt als eine Momentaufnahme eines Ereignisses vorzustellen, während das Imperfekt den Hintergrund etabliert, \Npawl{tengkrr palulukan moene kxll \uwave{sarmi}, \uwave{poltxe} Neytiril aylì'ut a frakrr 'ok seyä layu oer} \E{als der Thanator auf uns beide \uwave{zustürmte}, \uwave{sagte} Neytiri etwas, an das ich mich immer erinnern werde.}

\subsection{Gleichzeitigkeit} Da das Imperfekt einen fortlaufenden Zustand beschreibt, kann es in komplexen Sätzen verwendet werden, um die Gleichzeitigkeit von Handlungen anzuzeigen, \Npawl{fìtxon yom tengkrr \uwave{teruvon}} \E{diese Nacht essen wir, während wir uns anlehnen}.
\index{Imperfekt!für Gleichzeitigkeit}
\LNWiki{14/3/2010}{https://wiki.learnnavi.org/index.php/Canon/2010/March-June\%23A_Collection}

\subsection{Vorzeitigkeit} In komplexen Sätzen kann das Perfekt in einem Nebensatz den Abschluss einer Handlung vor dem Ereignis im Hauptsatz anzeigen, \index{Perfekt!für Vorzeitigkeit}

\begin{quotation}
	\noindent\Npawl{Tì'eyngit oel \uwave{tolel} a krr, ayngaru payeng}.\\
	\indent\E{Wenn ich eine Antwort \uwave{erhalten habe}, werde ich es dich wissen lassen}.\\
	
	\noindent\Npawl{Fori mawkrra fa renten \uwave{ioi säpoli} holum}.\\
	\indent\E{Nachdem sie ihre Brillen \uwave{aufgesetzt hatten}, brachen sie auf.}
\end{quotation}

\subsection{Punktuelles Perfekt} Das Perfekt wird in mehreren verbalen Ausdrücken verwendet, um anzuzeigen, dass das Ereignis in einem Augenblick (punktuell) eingetreten ist, \N{tslolam} \E{verstanden, kapiert}, \N{rolun} \E{gefunden!}. Frommer sagt, \N{tolel}, \E{(ich) hab's!}, stehe für einen ``Geistesblitz''.
\index{Perfekt!für Punktualität}

\subsection{Tempus} Die Zeitform verortet im Na'vi, wie in menschlichen Sprachen, ein Ereignis in der Zeit.

\QUAESTIO{Es gibt zu wenige Beispiele für komplexe Sätze, um sich über relative Tempusverhältnisse in Nebensätzen sicher zu sein.}

\subsection{Nahe Temporaldeixis} Die Zeitformen der nahen Deixis markieren Ereignisse in der ``nahen'' Vergangenheit oder Zukunft, wobei ``Nähe'' kein absoluter Maßstab ist, sondern vom Kontext und der Perspektive des Sprechers bestimmt wird.

\subsubsection{``Bald''} Das Adverb \N{ye'rìn} \E{bald} kann zwar bei Verben mit dem Futurinfix \N{\INF{ay}} stehen; bei Verben, die mit der nahen Zukunft, \N{\INF{ìy}}, markiert sind, ist es aber redundant.
\LNForum{25/10/2022}{https://forum.learnnavi.org/index.php?msg=679687}
\index{Zukunft!nah!mit \textbf{ye'rìn}}
\index{ye'rìn@\textbf{ye'rìn}!mit naher Zukunft}

\subsection{Absicht} Die beiden Infixe, die eine auf die Zukunft ausgerichtete Absicht\footnote{\href{https://de.wikipedia.org/wiki/Energikus}{Vgl. Wikipedia: Energikus}} markieren -- \N{\INF{ìsy}} und \N{\INF{asy}} -- zeigen eher die Entschlossenheit des Sprechers an, einen Zustand herbeizuführen, als eine Vorhersage über die Zukunft, \Npawl{ayoe ke \uwave{wasyem}} \E{wir werden nicht kämpfen}, \Npawl{tafral ke \uwave{lìsyek} oel ngeyä keye'ungit} \E{daher werde ich deinem Irrsinn keine Beachtung schenken}.\label{syn:verb:intenfut}
\index{Zukunft!für Absicht}

\section{Subjunktiv}
\index{Subjunktiv}
\noindent Der Subjunktiv wird im Na'vi sehr häufig gebraucht. Außerhalb seiner Verwendung in unabhängigen Sätzen ist er in bestimmten grammatikalischen Konstruktionen zwingend erforderlich, ohne dabei notwendigerweise auf einen \textit{modus irrealis} hinzudeuten.

\subsection{Optativ} Der Subjunktiv wird verwendet, um einen Wunsch zu äußern, \Npawl{oeyä swizaw nìngay tivakuk} \E{lass meinen Pfeil sein Ziel sicher treffen}. \index{Subjunktiv!Optativ}

\subsection{Nìrangal} Unerfüllbare Wünsche werden mit dem Adverb \N{nìrangal} und dem Subjunktiv Imperfekt für einen Wunsch in der Gegenwart bzw. dem Subjunktiv Perfekt für einen Wunsch in der Vergangenheit formuliert. Dem entspricht im Deutschen ``wenn doch nur'' oder ``ich wünschte, (dass)...'', \N{nìrangal \uwave{lirvu} oeyä frrnenur lora sanhì} \E{ich wünschte, meine Kinder hätten schöne Sterne}, \N{nìrangal oel \uwave{tslilvam} nì'ul} \E{wenn ich doch nur mehr verstanden hätte}.
\index{nìrangal@\textbf{nìrangal}}
\LNWiki{14/3/2010}{https://wiki.learnnavi.org/index.php/Canon/2010/March-June\%23A_Collection}

\subsection{Modalverbkonstruktionen} \index{Modalverb}
Ein Vollverb, das zu einem Modalverb -- wie \N{zene} \E{müssen}, \N{tsun} \E{können}, usw. -- gehört, steht im Subjunktiv, wie in \N{ayngari zene hivum} \E{ihr müsst gehen}, \Npawl{oe new nìtxan ayngaru fyawivìntxu} \E{ich möchte euch sehr gerne führen}, \Npawl{fmawn a tsun oe ayngaru tivìng} \E{Neuigkeiten, die ich euch geben kann}. \label{syn:modals}

\subsubsection{} \label{syn:modal-syntax} Das vom Modalverb kontrollierte Vollverb kann nur die Infixe erhalten, die der Vor-Position zugeordnet sind (d. h. Reflexiv und Kausativ, \horenref{morph:pre-first}), und das Infix für den Subjunktiv. Es erhält keine Zeit-, Aspekt- oder Affektinfixe. Mit diesen wird stattdessen das Modalverb markiert, \N{oe namew tsive'a} \E{ich wollte sehen}, niemals \N{*oe new tsimve'a}.
\LNForum{14/10/2010}{https://forum.learnnavi.org/index.php?msg=332239}

\begin{quotation}
	\noindent\Npawl{Pori mesyokx rìkxi, ha ke \uwave{tsayun} yerikit \uwave{tivakuk}.}\\
	\indent\E{Seine Hände zittern, deshalb wird er den Hexapeden nicht treffen können.}\\
	\noindent\Npawl{Furia \uwave{tsolun} oe ngahu \uwave{pivängkxo}, oeru prrte' lu nìngay.}\\
	\indent\E{Es war mir eine Freude, mit dir zu sprechen}.\\
	\noindent\Npawl{Fteria oel lì'fyati leNa'vi, slä mi ke \uwave{tsängun pivlltxe} na hufwe.}\\
	\indent\E{Ich lerne Na’vi, aber ich fürchte, ich kann es noch immer nicht fließend sprechen.}
\end{quotation}

Außer in der Poesie oder zeremonieller Sprache steht das Modalverb immer unmittelbar vor dem kontrollierten Vollverb.
\NTeri{3/19/2011}{https://naviteri.org/2011/03/word-order-and-case-marking-with-modals/}

\subsubsection{} Bekannte Modalverben und Verben mit modaler Struktur:\footnote{\QUAESTIO{Weitere Kandidaten: \N{flä} \E{gelingen},
		\N{hawl} \E{vorbereiten}.}}
% This is what I get for attaching several words to the same footnote.
\addtocounter{footnote}{1}
\newcounter{modalsnote}\setcounter{modalsnote}{\value{footnote}}
\begin{center}
	\begin{tabular}{ll}
		\N{fmi} & versuchen \\
		\N{ftang} & aufhören, stoppen \\
		\N{kan} & vorhaben\footnotemark[\value{modalsnote}] \\
		\N{kom} & wagen\\
		\N{may'} & ausprobieren, testen \\
		\N{new} & wollen\footnotemark[\value{modalsnote}] \\
		\N{nulnew} & bevorzugen \\
	\end{tabular}
	\hskip 3em
	\begin{tabular}{ll}
		\N{sngä'i} & beginnen, starten \\
		\N{sto} & verweigern\footnotemark[\value{modalsnote}] \\
		\N{tsun} & können, in der Lage sein \\
		\N{var} & fortsetzen \index{var@\textbf{var}} \\
		\N{zene} & müssen \\
		\N{zenke} & nicht dürfen \\
	\end{tabular}
\end{center}
\footnotetext[\value{modalsnote}]{Siehe auch \horenref{syn:modal:new}.}
\index{fmi@\textbf{fmi}!Modalverb}\index{ftang@\textbf{ftang}!Modalverb}
\index{new@\textbf{new}!Modalverb}\index{kan@\textbf{kan}!Modalverb}
\index{snaä'i@\textbf{sngä'i}!Modalverb}\index{sto@\textbf{sto}!Modalverb}
\index{tsun@\textbf{tsun}!Modalverb}
\index{var@\textbf{var}!Modalverb}\index{zene@\textbf{zene}!Modalverb}
\index{zenke@\textbf{zenke}!Modalverb}\index{may'@\textbf{may'}!Modalverb}\index{kom@\textbf{kom}!Modalverb}

\noindent\NTeri{25/5/2011}{https://naviteri.org/2011/05/some-miscellaneous-vocabulary/}
\LNWiki{1/12/2010}{https://wiki.learnnavi.org/index.php/Canon/2010/October-December\%23As_X_as_Y.3B_Keep_on_keepin.27_on}
\LNWiki{2/2/2011}{https://wiki.learnnavi.org/index.php/Canon/2011/January-March\%23Stop.21}
\Ultxa{2/10/2010}{https://wiki.learnnavi.org/index.php/Canon/2010/UltxaAyharyu\%C3\%A4\%23.C3.ACm.C3.ACy_and_modal_kan}
\LNWiki{13/12/2012}{https://wiki.learnnavi.org/index.php/Canon/2012/July-December\%23.22sto.22_has_the_same_syntax_as_.22new.22}
\NTeri{6/6/2019}{https://naviteri.org/2019/06/50a-liu-amip-40-new-words\%ef\%bb\%bf/}
\NTeri{23/12/2020}{https://naviteri.org/2020/12/mrra-tipangkxotsyip-five-little-discussions/}

\subsubsection{} Man beachte, dass Modalverben als intransitiv gelten und damit das Subjekt in der Verbalphrase im Absolutiv steht, unabhängig von der Transitivität des kontrollierten Vollverbs, \N{\uwave{oe} new yivom teylut} \E{ich möchte Teylu essen}. Siehe aber die Auswirkungen der Wortfolge darauf, \horenref{pragma:word-order-effects:modals}, für einige außergewöhnliche Muster.

Wenn das Modalverb jedoch einen Objektsatz nehmen kann, dann verhält es sich transitiv, \Npawl{oel new futa nga srew} \E{ich möchte, dass du tanzt} (\horenref{syn:modal:subclause}).

\subsubsection{}\label{syn:modal:neg}
Wenn \N{ke} mit Modalverben verwendet wird, kann es entweder vor dem Modalverb oder vor dem kontrollierten Vollverb stehen. Manchmal ändert sich dadurch die Bedeutung, wie bei \Npawl{ke zene kivä} \E{nicht gehen müssen} vs. \N{zene ke kivä} \E{nicht gehen dürfen} oder \Npawl{po ke tsun yivom} \E{sie sind nicht in der Lage, zu essen} vs. \Npawl{po tsun ke yivom} \E{sie sind in der Lage, nicht zu essen}.
\LNForum{23/7/2019}{https://forum.learnnavi.org/index.php?msg=665859}

\subsection{Modalverben mit Nebensätzen}
\label{syn:modal:new}\label{syn:modal:subclause}
Einige Modalverben können Objektsätze annehmen, die mit \N{futa} oder \N{a fì'ut} (\horenref{morph:fwa-tsawa}) eingeleitet werden. Wenn das Subjekt des Modalverbs und das kontrollierte Vollverb identisch sind, ist es normalerweise nicht notwendig, aber möglich, eine der Konjunktionen zu verwenden.
\index{fmi@\textbf{fmi}!mit \N{futa}} \index{kan@\textbf{kan}!mit \N{futa}}
\index{may'@\textbf{may'}!mit \N{futa}} \index{new@\textbf{new}!mit \N{futa}}
\index{nulnew@\textbf{nulnew}!mit \N{futa}} \index{sto@\textbf{sto}!mit \N{futa}}

\begin{quotation}
	\noindent\Npawl{Oe new srivew}.  \E{Ich möchte tanzen.} \\
	\noindent\Npawl{Oel new futa srew}.  \E{Ich möchte tanzen.} (wörtl. \E{ich möchte, dass ich tanze.})
\end{quotation}

\noindent Hier ist zu beachten, dass der Nebensatz nicht den Subjunktiv erfordert. Analog zu dem sehr häufigen Muster in Sätzen wie \N{oe new srivew} ist es jedoch erlaubt, auch den Subjunktiv zu verwenden, \N{oel new futa srivew}. Dies gilt auch, wenn das Subjekt des Nebensatzes ein anderes ist als das des Hauptsatzes,

\begin{quotation}
	\noindent\Npawl{Oel new futa nga srew}.  \E{Ich möchte, dass du tanzt.} \\
	\noindent\Npawl{Oel new futa nga srivew}.  \E{Ich möchte, dass du tanzt.} \\
	
	\noindent\N{Poel stolatso futa mefo tivaron tsaha'ngir.} \\
	\indent\E{Sie muss abgelehnt haben, an jenem Nachmittag zu jagen.}
\end{quotation}
\LNForum{29/11/2020}{https://forum.learnnavi.org/index.php?msg=673724}
\LNForum{13/12/2012}{https://forum.learnnavi.org/index.php?msg=567034}

\noindent Die Verben, die diese Variation im Nebensatz erlauben, sind \N{fmi, kan, may', new, nulnew} and \N{sto}.

\subsubsection{} Der Kausativ des transitiven Verbs \N{new} kann ebenfalls einen Objektsatz mit \N{futa} nehmen, \Npawl{pol oeru neykew futa oel yivom teyluti}
\E{er brachte mich dazu, Teylu essen zu wollen} (wörtl. \E{er brachte mich dazu, zu wollen, Teylu zu essen}). \index{new@\textbf{new}!Kausativ}
\LNWiki{3/10/2010}{https://wiki.learnnavi.org/Canon/2010/UltxaAyharyu\%C3\%A4\#He\_made\_me\_want\_to\_make\_you\_eat\_teylu}
\LNForum{29/11/2020}{https://forum.learnnavi.org/index.php?msg=673724}

%%%%%%%%%%%%%%%%%%%%%%%%%%%%%%%%%%%%%%%%%%%%%%%%%%%%%%%%%%%%
%%% 2020dec02 - preserved for examples, just in case

%\subsection{New} In addition to the simple modal use given above
%(\horenref{syn:modals}), \N{new} \E{want} may also intro\-duce a
%subclause with a different subject than that of the \N{new} clause.
%The verb is transitive in this construction, and the subclause is
%attached to \N{a fì'ut} or \N{futa} (\horenref{syn:clause-nom}) and
%takes the subjunctive. \label{syn:modal:new}\index{new@\textbf{new}}
%\LNWiki{20/1/2010}{https://wiki.learnnavi.org/index.php/Canon\%23Extracts_from_various_emails}
%
%\begin{quotation}
%\noindent \N{Oel new futa po kivä} \E{I want him to go} (lit. \E{I
	%  want that he go}). \\
%\noindent \N{Ngal tslivam a fì'ut new oel} \E{I want you to understand}.
%\end{quotation}
%
%\subsubsection{} The modal use of \N{kan} \E{aim} for \E{intend}
%follows the same syntax, \N{oe kan kivä} \E{I intend to go} and \N{oel
	%kan futa po kivä} \E{I intend him to go}.  \index{kan@\textbf{kan}}
%
%\subsubsection{} The verb \N{sto} \E{refuse} also follows the syntax
%of \N{new}, \N{stolo po hivum fohu} \E{she refused to leave with
	%them,} \N{poel stolatso futa mefo tivaron tsaha'ngir}
%\E{she must have refused (their request) to hunt that afternoon}.
%\index{sto@\textbf{sto}}

%%% 2020dec02 - preserved for examples, just in case   ^ ^ ^ ^ ^ ^ ^ ^
%%%%%%%%%%%%%%%%%%%%%%%%%%%%%%%%%%%%%%%%%%%%%%%%%%%%%%%%%%%%

\subsection{Andere Verwendungen} Der Subjunktiv wird auch in finalen Nebensätzen zusammen mit \N{fte} (\horenref{syn:purpose}), in Konditionalsätzen (\horenref{syn:conditionals}) sowie mit der Konjunktion \N{tsnì} zusammen mit bestimmten Verben verwendet (\horenref{syn:tsni}). 

\section{Partizipien und Gerundien}

\subsection{Partizipien} \index{Partizip!Gebrauch}
Na'vi-Partizipien sind in ihrer Verwendung eingeschränkt -- sie dürfen nur attributiv verwendet werden, niemals prädikativ. Da sie syntaktisch als Adjektive gelten, werden sie durch das attributive Affix \N{-a-} mit dem beschriebenen Substantiv verbunden (\horenref{morph:adj-attr}), \Npawl{\uwave{pa\-lu\-lu\-kan atusaron} lu lehrrap} \E{ein \uwave{jagender Thanator} ist gefährlich}.\label{syn:part:attr}

\subsubsection{} Einige abgeleitete Wörter sind lexikalisierte Partizipien. Diese  können prädikativ verwendet werden, wie in \Npawl{lu nga txantslusam} \E{du bist weise}, mit dem Partizip Aktiv \N{tsl\INF{us}am} darin.

\subsubsection{} Die Partizipien von \N{si}-Verben werden als ein einziges Wort gezählt. Sie werden mit einem Bindestrich geschrieben, der \N{si} und das andere Wort verbindet, sodass das Affix \N{-a-} an den gesamten Ausdruck angehängt wird, nicht nur an \N{si}, \label{syn:participle:si-const}
\index{si-Konstruktion@\textbf{si}-Konstrution!Partizip}

\begin{quotation}
	\noindent\N{srung-susi\uwave{a} tute}\\
	\noindent\N{tute \uwave{a}srung-susi}
\end{quotation}

\noindent Beide Ausdrücke bedeuten \E{eine helfende Person}.

\subsection{Gerundien} Jedes Verb kann frei in ein Gerundium verwandelt werden, d. h. in ein Substantiv, das die Handlung des Verbs beschreibt (\horenref{lingop:gerund}). Sie können mit Adverbien verwendet werden (\horenref{syn:adverbs:gerund}), aber sie dürfen weder Subjekte noch direkte Objekte annehmen, \Npawl{tìyusom 'o' lu} \E{(zu) essen macht Spaß}.\label{syn:gerund}
\LNWiki{18/6/2010}{https://wiki.learnnavi.org/index.php/Canon/2010/March-June\%23Fwa_with_adpositions}

\subsubsection{} Im Deutschen, das kein Gerundium kennt, wird anstelle der Nominalisierung des Verbs meist eine Infinitivkonstruktion mit ``zu'' verwendet (``einen Marathon \uwave{zu laufen} ist schwierig''). Auch im Na'vi kann das entsprechende Verb in einen mit \N{fì’u} oder \N{tsa'u} eingeleiteten Nebensatz ausgelagert werden (\horenref{syn:clause-nom}), \Npawl{\uwave{fwa yom teylut} 'o' lu}, \E{\uwave{Teylu zu essen} macht Spaß}. \index{Gerundium!Gebrauch}
\Ultxa{3/10/2010}{https://wiki.learnnavi.org/index.php/Canon/2010/UltxaAyharyu\%C3\%A4\%23Gerunds_vs._Fwa}

\section{Reflexiv}
\index{Reflexiv!Syntax}
\subsection{Reflexiv} Das Infix für den Reflexiv \N{\INF{äp}} zeigt an, dass das Subjekt des Verbs eine Handlung an sich selbst ausführt. Das Subjekt steht im Absolutiv, nicht im Agens, wie in \Npawl{oe tsäpe'a} \E{ich sehe mich (selbst)}.
\LNWiki{1/2/2010}{https://wiki.learnnavi.org/index.php/Canon\%23Reflexives_and_Naming}

\subsection{Intransitive Reflexive} Bei intransitiven Verben, die Dativobjekte nehmen, wird das Reflexivpronomen \N{sno} verwendet:
\index{Reflexiv!intransitiver Verben}

\begin{quotation}
	\noindent\N{Po yawne lu snor.} \E{Er liebt sich (selbst).}
\end{quotation} \index{sno@\textbf{sno}!mit intransitiven Reflexiven}
\noindent \NTeri{31/12/2011}{https://naviteri.org/2011/12/one-more-for-2011/}

\subsection{Detransitiv} Das Reflexivinfix macht transitive Verben intransitiv.\footnote{Lerner romanischer Sprachen wird dies bekannt vorkommen: \textit{je lave ma voiture} (transitiv) vs. \textit{je me lave} (intransitiv).} Es kann aber in einige wenige \N{si}-Verben gesetzt werden, um neue, lexikalisierte Verben zu bilden, z. B. \N{win säpi} \E{sich beeilen}.\footnote{Anm. d. Ü.: Im Gegensatz zum Englischen, wo die Reflexivität oft verloren geht (\E{(to) hurry}), bleibt sie im Deutschen erhalten.}

\subsection{Wechselseitigkeit} Steht ein reflexives Verb zusammen mit dem Adverb \N{fìtsap} \E{einander, gegenseitig}, drückt es Wechselseitigkeit aus, \Npawl{mefo fìtsap mäpoleyam tengkrr tsngawvìk} \E{die beiden umarmten sich (gegenseitig) und weinten}.
\NTeri{30/10/2011}{https://naviteri.org/2011/10/more-vocabulary-a-bit-of-grammar/}
\index{fiìtsap@\textbf{fìtsap}}\index{Wechselseitigkeit}

\subsubsection{} Bei intransitiven Verben, die Dativobjekte nehmen, gibt es zwei Möglichkeiten,

\begin{quotation}
	\noindent\Npawl{Moe smon moeru fìtsap.} \E{Wir beide kennen uns gegenseitig.} \\
	\noindent\Npawl{Moe smon fìtsap.} \E{Wir beide kennen uns gegenseitig.}
\end{quotation}

\noindent Bei Pronomen der dritten Person beliebiger Anzahl wird der Dativ von \N{sno} verwendet:

\begin{quotation}
	\noindent\Npawl{Fo smon (snoru) fìtsap nìwotx.} \E{Sie alle kennen sich gegenseitig.}
\end{quotation}

\noindent \NTeri{31/12/2011}{https://naviteri.org/2011/12/one-more-for-2011/}

\section{Kausativ}
\noindent Das Kausativinfix \N{\INF{eyk}} erhöht die Transitivität eines Verbs, indem es ein weiteres Argument hinzufügt. Alle kausativen Verben sind somit transitiv und erfordern den Agens für das Subjekt.
\label{syn:causative}

\subsection{Kausativ intransitiver Verben} Wenn ein intransitives Verb in den Kausativ gesetzt wird, so wird das ursprüngliche Subjekt im Absolutiv zum direkten Objekt im Patiens,\index{Kausativ!intransitiver Verben}

\begin{quotation}
	\noindent\N{\uwave{Oe} kolä neto.} \E{Ich ging fort.}\\
	\noindent\N{Pol \uwave{oeti} keykolä neto.} \E{Sie brachte mich dazu, fortzugehen.}
\end{quotation}

\subsection{Kausativ transitiver Verben} Wenn ein transitives Verb in den Kausativ gesetzt wird, so wird das ursprüngliche Subjekt im Agens zum indirekten Objekt im Dativ. Dadurch bleibt das ursprüngliche direkte Objekt im Patiens erhalten, \index{Kausativ!transitiver Verben} \label{syn:trans-causative}
\index{Kasus!Dativ!mit Kausativ}

\begin{quotation}
	\noindent\N{\uwave{Neytiril} \uuline{yerikit} tolaron.} \E{Neytiri hat einen Hexapeden gejagt.}\\
	\noindent\Npawl{Eytukanìl \uwave{Neytirir} \uuline{yerikit} teykolaron.}\\
	\indent \E{Eytukan brachte Neytiri dazu, einen Hexapeden zu jagen.} 
\end{quotation}

\subsubsection{} Das indirekte Objekt im kausativen Satz kann auch mit der Adposition \N{fa} \E{mittels, durch} versehen werden. Dadurch gerät es aus dem Fokus, und stattdessen wird entweder das Subjekt oder das direkte Objekt in den Mittelpunkt gestellt. \index{fa@\textbf{fa}!mit Kausativ}

\begin{quotation}
	\noindent\N{\uwave{Neytiril} \uuline{yerikit} tolaron} \E{Neytiri hat einen Hexapeden gejagt.}.\\
	\noindent\Npawl{Eytukanìl \uwave{fa Neytiri} \uuline{yerikit} teykolaron} \\
	\indent \E{Eytukan ließ einen Hexapeden von Neytiri jagen.}
\end{quotation}

\subsection{Kausativ des Modalverbs} \index{Kausativ!des Modalverbs}
Modalverben werden nicht mit dem Kausativ markiert. Wenn jedoch \N{new} \E{wollen} als transitives Verb fungiert, kann es im Kausativ stehen und einen \N{futa}-Teilsatz nehmen, \Npawl{fìpamtseol oeru neykew futa srew} \E{diese Musik sorgt dafür, dass ich tanzen will}.
\LNForum{29/11/2020}{https://forum.learnnavi.org/index.php?msg=673724}

\section{Ambitransitivität}
\noindent Ein normalerweise transitives Verb kann ein Subjekt im Absolutiv statt im Agens-Fall haben, wenn das direkte Objekt als irrelevant angesehen wird und nur die verbale Handlung von Bedeutung ist. Beispielsweise ist \N{oe taron} \E{ich jage} eine allgemeine Aussage über die Tätigkeit, bei der es nicht darauf ankommt, was genau gejagt wird. \index{Antipassiv} \index{Ambitransitivität}\NTeri{28/3/2012}{https://naviteri.org/2012/03/spring-vocabulary-part-1/}

\begin{quotation}
	\noindent\Npawl{Ngal pelun faystxenut frakrr tsyär?}\\
	\indent\E{Warum lehnst du ständig diese Gaben ab?} \hskip3em vs. \\
	\noindent\Npawl{Nga pelun frakrr tsyär?}\\
	\indent\E{Warum lehnst du ständig (alles} oder \E{solche Sachen) ab?}
\end{quotation}

\noindent Dieses Muster kann auch als ``Antipassiv'' bezeichnet und im Na'vi frei verwendet werden.

\subsection{Ausgelassenes Objekt} Diese Verwendung ist von der Auslassung eines direkten Objekts, das im Diskurskontext etabliert ist, zu unterscheiden, z. B.

\begin{quotation}
	\noindent A: \N{Ngal ke tse'a txepit srak?} \E{Siehst du das Feuer nicht?}\\
	\noindent B: \N{Oel tse'a.} \E{Ich sehe (es).}
\end{quotation}

\noindent Hier entfällt das direkte Objekt nicht völlig, sondern es wird aus pragmatischen Gründen schlicht nicht erwähnt, sodass das Verb und das Subjekt weiterhin der normalen transitiven Syntax folgen.\index{direktes Objekt!ausgelassen}

\subsection{Kausativ} Es gibt keine Möglichkeit, das Antipassiv in kausativen Sätzen zu erkennen. Beispielsweise könnte die resultierende Handlung des Satzes \N{oel poru teykaron} \E{ich veranlasse ihn, zu jagen} entweder \N{po taron} \E{er jagt (etwas, das irrelevant ist)} oder \N{pol taron} \E{er jagt (etwas Bestimmtes)} sein.
\index{Antipassiv!beim Kausativ}
\Ultxa{3/10/2010}{https://wiki.learnnavi.org/index.php/Canon/2010/UltxaAyharyu\%C3\%A4\%23Causative_for_ambitransitive_verbs}

\section{Befehle}
\index{Befehl}\index{Imperativ}
\subsection{Unmarkierter Imperativ} Befehle im Na'vi erfordern kein besonderes Infix. Positive Befehle bestehen einfach aus dem Verbstamm, \N{Kä! Kä!} \E{Los! Los!}, \Nfilm{mefoti yìm} \E{fesselt sie beide!}. Das Pronomen kann auch explizit angegeben werden, \Npawl{'awpot set ftxey ayngal} \E{(ihr) wählt jetzt eines aus}.
\index{Befehl!unmarkiert}

\subsection{Mit dem Subjunktiv} Ein Befehl kann auch das Subjunktivinfix \N{\INF{iv}} verwenden. Frommer sagt dazu: ``An einem früheren Punkt in der Geschichte der Sprache gab es wahrscheinlich eine Unterscheidung zwischen höflich und vertraut (wobei die \N{\INF{iv}}-Form die höflichere war), aber das ist nicht mehr der Fall. Sie können beide austauschbar verwendet werden. Wenn man also `Los!' sagen will, kann man entweder \N{kivä} oder einfach \N{kä} sagen.''

\index{Subjunktiv!für Befehle}\index{Befehl!mit Subjunktiv}

\subsection{Prohibitiv} Negative Befehle (Verbote) verwenden nicht die übliche Negationspartikel \N{ke}, sondern \N{rä'ä}, wie in \N{rä'ä hahaw} \E{schlaf nicht}.
\label{syntax:prohibitions}\index{Befehl!negativ}\index{Prohibitiv}

\subsubsection{} \N{Rä'ä} kann zur besonderen Betonung dem Verb folgen, \Npawl{oeti 'ampi rä'ä, ma skxawng!} \E{fass mich nicht an, du Idiot!}.
\NTeri{27/11/2012}{https://naviteri.org/2012/11/renu-ayinanfyaya-the-senses-paradigm/}

\subsubsection{} Bei \N{si}-Verben wird \N{rä'ä} zwischen das nichtverbale Element und das \N{si} platziert, \N{txo-\linebreak pu rä'ä si} \E{habe keine Angst}, \N{tsakem rä'ä sivi} \E{tu das (jene Aktion) nicht} (siehe auch \horenref{syn:neg:si-const}).\index{si-Konstruktion@\textbf{si}-Konstruktion!Prohibitiv}

\section{Fragen}
\index{Frage}
\subsection{Entscheidungsfragen} \index{Frage!Entscheidungsfrage}\index{Frage!ja-nein}
Einfache Fragen, die mit ``Ja'' oder ``Nein'' beantwortet werden können, werden mit der Partikel \N{srak(e)} gekennzeichnet, die am Anfang oder am Ende des Satzes steht. Steht sie am Ende, wird sie normalerweise zu \N{srak} abgekürzt, das längere \N{srake} steht am Satzanfang, \Npawl{ngaru lu fpom srak?} \E{geht es dir gut?}.

\subsubsection{Negative Entscheidungsfragen} \index{Frage!negative Entscheidungsfrage}
Die Partikel \N{srak(e)} fordert eine Bestätigung der Wahrheit oder Unwahrheit der gesamten Aussage, an die sie angehängt ist. Die richtige Antwort auf \Npawl{nga ke lu Txewì srak?} \E{du bist nicht Txewi?} ist \N{srane}, wenn dem nicht so ist und \N{kehe}, wenn doch. Man beachte, dass das Deutsche diese Situation anders handhabt und dass Deutschsprachige darauf achten müssen, wie sie negative Fragen beantworten.\footnote{Eine Frage der Form \N{Srake `X'?} bzw. \N{`X' srak?}, bei der X eine beliebige Aussage ist, fragt schlicht danach, ob X wahr ist oder nicht. Dabei macht es keinen Unterschied, ob X eine positive oder negative Aussage ist -- eine Antwort mit \N{srane} bedeutet stets, dass X wahr ist; \N{kehe} bedeutet, dass X nicht wahr ist. Die Entscheidungsfrage impliziert keine vorbestehenden Annahmen aufseiten des Fragestellers.} \NTeri{28/2/2018}{https://naviteri.org/2018/02/negative-questions-in-navi/}

\subsection{Ftxey... fuke} Zusätzlich zu \N{srak(e)} kann eine Entscheidungsfrage idiomatisch mit \N{ftxey} \E{wählen} und \N{fu\ACC{ke}} \E{oder nicht} formuliert werden: Man kann entweder \Npawl{srake nga za'u?} \E{kommst du?} oder \Npawl{ftxey nga za'u fuke?} \E{kommst du oder nicht?} sagen.
\index{ftxey@\textbf{ftxey}}\index{fuke@\textbf{fuke}}\index{Frage!Entscheidungsfrage mit \textbf{ftxey... fuke}}\label{syn:question:ftxey}\LNWiki{24/3/2010}{https://wiki.learnnavi.org/index.php/Canon/2010/March-June\%23If_and_Whether}

\subsection{Ergänzungsfragen} Die Verwendung eines Fragewortes, das \N{-pe+} enthält, reicht aus, um eine Ergänzungsfrage zu bilden, \N{kempe si nga?} \E{was machst du?}. In vielen Sprachen muss ein Fragewort am Anfang des Satzes stehen; im Na'vi gibt es diese Restriktion nicht, \Nfilm{fìswiräti ngal \uwave{pelun} molunge fìtsenge?} \E{\uwave{warum} hast du diese Kreatur hergebracht?}.\index{Frage!Ergänzungsfrage}

\subsection{Refrainfragen} Die Na'vi-Refrainfrage (dt. ``oder?'', ``nicht wahr?''; engl. ``right?''; frz. ``n'est-ce pas?'') wird entweder mit \N{kefya srak} oder einfach \N{kefyak} markiert (letztlich abgeleitet von \N{ke lu fìfya srak?} \E{ist es nicht so?}).
\index{Frage!Refrainfrage}\index{kefyak, kefya srak@\textbf{kefyak, kefya srak}}\LNWiki{1/3/2010}{https://wiki.learnnavi.org/index.php/Canon\%23Tag_Question}

\subsection{Vermutungsfragen} Eine Frage, bei welcher der Sprecher vom Adressaten nicht erwartet, dass er die Antwort kennt, wird mit dem Infix für Vermutungen \N{\INF{ats}} gebildet, \Npawl{pol pesenget tatsok?} \E{wo mag sie wohl sein?}, \Npawl{srake pxefo li polähatsem?} \E{ob die drei wohl schon angekommen sind?}.
\index{Vermutung!in Fragen}\index{Frage!Vermutungsfrage}\NTeri{30/10/2011}{https://naviteri.org/2011/10/more-vocabulary-a-bit-of-grammar/}

\subsection{Alternativfragen} Eine Frage, in welcher der Sprecher Antwortmöglichkeiten zur Auswahl anbietet, wird gebildet, indem \N{fu} vor jede Wahlmöglichkeit gesetzt wird, \Npawl{nulnew ngal fu fì'ut fu tsa'ut?} \E{willst du lieber dieses oder jenes?}.
\index{Frage!Alternativfrage}
\NTeri{30/09/2019}{https://naviteri.org/2019/09/choice-statements-vs-choice-questions-and-some-insults/}

\section{Affekt und Evidenz}

\subsection{Affekt} Zwei Infixe, welche die zweite Infix-Position belegen, werden verwendet, um die Einstellung des Sprechers zu dem, was er sagt, zu markieren, \N{\INF{ei}} für eine positive und \N{\INF{äng}} für eine negative Einstellung, \Nfilm{oel ngati kameie} \E{ich sehe dich}, \Nfilm{oeri ta peyä fahew akewong ontu teya längu} \E{sein Aliengeruch erfüllt meine Nase}. \index{Affekt!positiv}\index{Affekt!negativ}

\subsubsection{} Wenn eine Aussage inhärent sehr positive oder negative Gefühle impliziert, entfällt das Infix meist, wie in \Npawl{nga yawne lu oer} \E{ich liebe dich}.
\LNWiki{1/2/2010}{https://wiki.learnnavi.org/index.php/Canon\%23Extracts_from_various_emails}

\subsection{Evidenz} Das Infix \N{\INF{ats}}, das ebenfalls Position 2 belegt, wird verwendet, um eine Vermutung zu kennzeichnen, die auf der Grundlage einer Evidenz angestellt wird,\footnote{Im Deutschen wird dieser Modus oft mit ``müssen'' (z. B. ``es muss geregnet haben'', ``er muss Probleme mit seinen Hausaufgaben haben'') oder dem Futur (``das wird schon richtig sein'') ausgedrückt.} \Npawl{'uol ikranit txopu sleykolatsu, taluna po tsìk yawo} \E{irgendetwas muss den Ikran erschreckt haben, denn er erhob sich plötzlich in die Luft.} \index{Evidenz}
\LNWiki{19/2/2010}{https://wiki.learnnavi.org/index.php/Canon\%23Evidential}
%\NTeri{9/1/2012}{https://naviteri.org/2012/01/mipa-zisit-ayliu-amip-new-words-for-the-new-year/}

% see def of yawo for example of evidential
%example of evidential 

\section{Verneinung}

\subsection{Einfache Verneinung} Das Adverb bzw. Partikel \N{ke} verneint einen Satz, \Npawl{fìtxon na ton alahe nìwotx pelun ke lu teng?} \E{warum ist diese Nacht nicht wie alle anderen Nächte?}. \index{Negation}\index{ke@\textbf{ke}}

\subsubsection{} Bei \N{si}-Verben steht das \N{ke} vor \N{si}, wie in \N{po pamrel ke si} \E{er schreibt nicht}. Der Satzakzent verlagert sich vom Substantiv- bzw. Adjektivelement des \N{si}-Verbs zu \N{ke}: \N{pamrel \ACC{ke} si} (siehe auch
\horenref{syntax:prohibitions}). \label{syn:neg:si-const}
\index{si-Konstruktion@\textbf{si}-Konstruktion!Negation}
\index{Negation!si-Konstruktion@\textbf{si}-Konstruktion}
\index{ke@\textbf{ke}!mit \textbf{si}-Konstruktion}

\subsubsection{} Imperative werden mit dem Adverb \N{rä'ä} negiert, siehe \horenref{syntax:prohibitions}. \index{Prohibitiv}

\subsubsection{} In einigen Fällen ändert die Position von \N{ke} in einer Modalverbkonstruktion die Bedeutung, siehe \horenref{syn:modal:neg}.

\subsection{Doppelte Verneinung} Wenn ein bereits negatives Adverb oder Pronomen (\horenref{morph:correlatives}) verwendet wird, muss das Verb dennoch mit \N{ke} verneint werden, \Npawl{ke'u ke lu ngay} \E{nichts ist wahr}, \N{slä ke stä'nì kawkrr} \E{aber (er) fängt (sie) niemals}.
\index{Negation!doppelte Verneinung}\label{syn:neg:pleon}\LNWiki{2/5/2010}{https://wiki.learnnavi.org/index.php/Canon/2010/March-June\%23Double_Negatives_Required}

\subsubsection{} Wenn das Präfix \N{fra-} negiert wird, muss auch das Verb negiert werden, \Npawl{ke frapo ke tslolam} \E{nicht jeder hat es verstanden}.
\index{fra-@\textbf{fra-}!mit \textbf{ke}}
\index{ke@\textbf{ke}!mit \textbf{fra-}}\Ultxa{3/10/2010}{https://wiki.learnnavi.org/index.php/Canon/2010/UltxaAyharyu\%C3\%A4\%23Ke_with_fra-}

\subsection{Kaw'it} Ein Wort oder eine Phrase kann für die Verneinung mit \N{ke... kaw\ACC{'it}} \E{überhaupt nicht} herausgegriffen werden, wie in \N{fo ke lu 'ewan kaw'it} \E{sie sind überhaupt nicht jung}.
\index{kaw'it@\textbf{kaw'it}}\LNWiki{6/4/2010}{https://wiki.learnnavi.org/index.php/Canon/2010/March-June\%23April_6_Miscellany}

\section{Komplexe Sätze}

%\subsection{Tense and Aspect in Dependent Subjunctives} 

\subsection{Absicht} Finalsätze werden mit der Konjunktion \N{fte} (negiert \N{fteke}) eingeleitet; das Verb steht im Subjunktiv, \N{sawtute zera'u fte fol Kelutralti skiva'a} \E{die Himmelsmenschen kommen, um den Heimatbaum zu zerstören}, \Npawl{makto kawl, ma samsiyu, fte tsivun pivähem nìwin} \E{reitet hart, Krieger, damit ihr schnell ankommt!}, \Npawl{tsun fko sivar hänit fte payoangit stivä'nì} \E{man kann ein Netz benutzen, um einen Fisch zu fangen}.
\label{syn:purpose}\index{Finalsatz}
\index{fte@\textbf{fte}}\index{fteke@\textbf{fteke}}

\subsubsection{} Im Na'vi werden Finalsätze in Situationen verwendet, in denen das Deutsche eine Infinitivkonstruktion mit ``zu'' verwenden würde, \Npawl{pxiset ke lu oeru krr \uwave{fte} tì'eyngit \uwave{tivìng}} \E{momentan habe ich nicht die Zeit, eine Antwort \uwave{zu geben}}.

\subsection{Asyndeton} Kurze, gleichförmige Sätze\footnote{D. h. Sätze, die demselben grammatikalischen Muster folgen.} können ohne eine Konjunktion verbunden werden, \Npawl{yola krr, txana krr, ke tsranten} \E{es spielt keine Rolle, wie lange es dauert} (wörtl. \E{kurze Zeit, lange Zeit, ist nicht wichtig}); \Npawl{'uo a fpi rey'eng \uwave{Eywa'evengmì 'Rrtamì} tsranten nìtxan awngaru nìwotx} \E{etwas, das jedem von uns zum Wohle des Gleichgewichts des Lebens sehr wichtig ist, \uwave{sowohl auf Pandora als auch auf der Erde}}; \Npawl{lora aylì'u, lora aysäfpìl} \E{schöne Worte, schöne Ideen}.
\index{Asyndeton}\index{Konjunktion!Ausfall}

\subsubsection{} \index{Verb!sequenziell}
Zwei Verben in Folge ohne Konjunktion bilden ein sequenzielles Verbpaar, \Npawl {za'u kaltxì si ko!} \E{komm (und dann) sag Hallo!}, \Nfilm{ngari hu Eywa salew tirea, tokx \uwave{'ì'awn slu} Na'viyä hapxì} \E{dein Geist geht mit Eywa und dein Körper \uwave{bleibt (und) wird} Teil des Volkes}, \Npawl{pol tsatxumit noläk terkup} \E{er trank das Gift (und) starb}.

Vollverben in Modalverbkonstruktionen können ebenfalls auf diese Weise aneinandergereiht werden, wobei beide im Subjunktiv stehen können. Beide der folgenden Möglichkeiten sind akzeptabel:

\begin{quotation}
	\noindent\Npawl{Tsun nekll zivup tsawng.} \E{Es kann auf den Boden fallen (und) zerbrechen}. \\
	\noindent\Npawl{Tsun nekll zivup tsivawng.} \E{Es kann auf den Boden fallen (und kann) zerbrechen}.
\end{quotation}

\noindent Zusätzliche Verben können mit \N{tsakrr} in die Sequenz eingeführt werden. 
\LNForum{2/5/2020}{https://forum.learnnavi.org/index.php?msg=670371}
\NTeri{21/4/2020}{https://naviteri.org/2020/04/aawa-u-amip-a-few-new-things/}

\section{Relativsätze und Teilsatzzuordnung}
\index{Relativsatz}
\subsection{Partikel ``a''} Na'vi-Relativsätze werden mit der Attributivpartikel \N{a} gebildet.\index{a@\textbf{a}}\label{syn:a} Wie bei der adjektivischen Attribution kann ein Relativsatz entweder vor oder nach dem Wort stehen, das er modifiziert, \Npawl{\uwave{po tsane karmä a tsengit} ke tsìme'a oel} \E{ich habe \uwave{die Stelle, zu der er gegangen ist,}\footnote{\normalfont Anm. d. Ü.: Zwischen einem nachgestellten Relativsatz und dem Wort, das er verändert, kann im Deutschen auch das Verb des Hauptsatzes stehen, ``ich habe die Stelle nicht \uwave{gesehen}, zu der er gegangen ist''.} nicht gesehen}, \Npawl{\uwave{palulukan a teraron} lu lehrrap} \E{\uwave{ein Thanator, der jagt}, ist gefährlich.}

\subsubsection{} Man beachte, dass das attributive \N{a} eine Partikel ist, aber kein Pronomen, und daher keine Kasusmarkierung annehmen kann.

\subsection{Verweishierarchie} Wenn der Kopf\footnote{Der Kopf des Relativsatzes ist das Substantiv im Hauptsatz, das vom Relativsatz modifiziert wird. Es hat eine syntaktische Rolle sowohl im Haupt- als auch im Relativsatz: In dem Satz ``ich sehe den Mann, der rennt'' ist das Wort ``Mann'' das direkte Objekt des Hauptsatzes ``ich sehe den Mann'', aber zugleich das Subjekt des Relativsatzes ``der Mann rennt''. Dieses Element, das beiden Sätzen gemeinsam ist, wird manchmal auch als ``syntaktischer Drehpunkt'' bezeichnet.} eines Relativsatzes das Subjekt oder das direkte Objekt dieses Relativsatzes ist, entfällt er,

\begin{quotation}
	\noindent \N{\uwave{Ngal tse'a a tute} lu eyktan}.
	\E{\uwave{Der Mann, den du siehst,} ist Anführer}.\\
	\noindent \N{\uwave{Ngati tse'a a tute} lu eyktan}.
	\E{\uwave{Der Mann, der dich sieht,} ist Anführer}. 
\end{quotation}
\noindent Für andere Fälle oder Adpositionalphrasen muss ein resumptives Pronomen\footnote{\href{https://de.wikipedia.org/wiki/Resumptives_Pronomen}{Vgl. Wikipedia: Resumptives Pronomen}} verwendet werden: \N{po} für belebte und \N{tsaw} für unbelebte Köpfe,
% Feb 18: https://wiki.learnnavi.org/index.php/Canon#More_extracts_from_various_emails

\begin{quotation}
	\noindent \Npawl{poru mesyal lu a ikran} \E{ein Ikran, der zwei Flügel hat}\\
	\noindent \Npawl{Po \uwave{tsane} karmä a tsengit ke tsìme'a oel}.\\
	\indent\E{Ich habe die Stelle nicht gesehen, \uwave{zu der} er gegangen ist}.\\
	\noindent \N{Fìpo lu tute a oe \uwave{pohu} perängkxo.} \E{Dies ist die Person, \uwave{mit der} ich spreche}.
\end{quotation}

\subsubsection{} Wenn der Kopf des Relativsatzes dessen direktes Objekt ist, muss das Subjekt des Verbs noch immer im Agens stehen, wie im obigen Satz \N{\uwave{ngal} tse'a a tute} \E{der Mann, den du siehst}, nicht *\N{\uwave{nga} tse'a a tute}; \Npawl{teylu a \uwave{oel} yerom lu ftxìlor} \E{das Teylu, das ich esse, ist köstlich}.
\NTeri{28/3/2012}{https://naviteri.org/2012/03/spring-vocabulary-part-1/}

\subsection{Andere attributive Teilsätze} Anstatt wie im Deutschen Substantive direkt mit Präpositionalphrasen zu modifizieren (``der Mann auf dem Mond''), werden im Na'vi Teilsätze dieser Art ebenfalls mit \N{a} an Substantive angehängt, wie in \Nfilm{fìpo lu \uwave{vrrtep a mì sokx atsleng}} \E{dies ist \uwave{ein Dämon in einem falschen Körper}}, \N{ngeyä \uwave{teri faytele a aysänumeri}} \E{deine \uwave{Anweisungen zu diesen Angelegenheiten}}.
\index{Adposition!im Attributivsatz}

\subsubsection{} Farbnuancen können mit der Adposition \N{na} \E{wie} präzisiert werden. Um eine solche Phrase attributiv zu verwenden, wird der gesamte Teilsatz mit Bindestrichen versehen und wie ein normales Adjektiv behandelt. Beispielsweise \Npawl{ean na ta'leng} \E{(Na'vi-)hautblau}:
\begin{quotation}
	\indent\N{Fìsyulang \uwave{aean-na-ta'leng} lor lu nìtxan.}\\
	\indent\N{Fìsyulang \uwave{ata'lengna-ean} lor lu nìtxan.}\\
	\indent\N{\uwave{Ean-na-ta'lenga} fìsyulang lor lu nìtxan.}\\
	\indent\N{\uwave{Ta'lengna-eana} fìsyulang lor lu nìtxan.}
\end{quotation}
\index{na@\textbf{na}!mit Farbadjektiven}\label{syn:attr:na}

\subsubsection{} Einzelne Adverbien können ebenfalls attributiv verwendet werden, \Npawl{ke zasyup lì'Ona ne kxutu \uwave{a mìfa} fu \uwave{a wrrpa}} \E{die lì'Ona werden weder durch den inneren noch durch den äußeren Feind untergehen}.
\index{Adverb!attributive Verwendung}

\subsection{Mit Adjektiven} \index{Relativsatz!mit attributivem Adjektiv}
Ein Relativsatz kann auf ein attributives Adjektiv auf der gleichen Seite des Substantivs folgen,

\begin{interlin}
	\glll Kaltxì, oeyä eylanur a'ewan a tok Toitslanti. \\
	kaltxì, oe-ä ay-'eylan-ur a-'ewan a tok Toitslan-ti \\
	hallo, 1\I{sg}-\I{gen} \I{pl}-Freund-\I{dat} \I{lig}-jung \I{rel} sein.in Deutschland-\I{pat}\\
	\trans{Hallo, meine jungen Freunde aus Deutschland.} \Ipawl{}
\end{interlin}

\noindent Eine andere akzeptable Wortfolge ist \Npawl{Kaltxì, oeyä 'ewana
	eylanur a tok Toitslanti}.
\NTeri{28/2/2021}{https://naviteri.org/2021/02/aysipawm-si-aysieyng-questions-and-answers/}

\subsection{Inhaltssatz} Ein ganzer Teilsatz kann zu einem einzigen Satzglied (dem Subjekt, Objekt oder Topik) zusammengefasst und mittels der attributiven Partikel \N{a} in den Satzbau eines anderen Satzes integriert werden, wobei entweder \N{fì'u} oder \N{tsa'u} den Teilsatz (Nebensatz) im Hauptsatz verankert. Dies ist so üblich, dass bestimmte Kombinationen von Demonstrativpronomen und Attributivpartikel verschmelzen (siehe \horenref{morph:fwa-tsawa}). \label{syn:clause-nom}
\index{fwa@\textbf{fwa}!Gebrauch}
\index{fula@\textbf{fula}!Gebrauch}
\index{futa@\textbf{futa}!Gebrauch}
\index{furia@\textbf{furia}!Gebrauch}

\subsubsection{} Genau wie in einem Relativsatz wird das Pronomen entsprechend seiner Rolle im Hauptsatz dekliniert -- zum Beispiel im Absolutiv (\N{fwa}) als intransitives Subjekt von \N{lu}:

\begin{quotation}
	\noindent\Npawl{Law lu oeru \uwave{fwa nga mì reltseo nolume nìtxan}}.\\
	\indent\E{Es ist mir klar, \uwave{dass du im Bereich der Kunst viel gelernt hast}}.
\end{quotation}

\noindent Im Topik-Fall (\N{a fì'uri}) zusammen mit \N{irayo si}:
\begin{quotation}
	\noindent\Npawl{\uwave{Ngal oeyä 'upxaret aysuteru fpole' a fì'uri}, ngaru irayo seiyi oe nìtxan.}\\
	\indent\E{Ich danke dir sehr, \uwave{dass du den Leuten meine Nachricht überbracht hast}}.
\end{quotation}

\noindent Als direktes Objekt (\N{futa}) des Verbs \N{omum}:
\begin{quotation}
	\noindent\Npawl{Ulte omum oel \uwave{futa tìfyawìntxuri oeyä perey aynga nìwotx}.}\\
	\indent\E{Und ich weiß \uwave{dass ihr alle auf meine Anleitung wartet.}}
\end{quotation}

\subsubsection{} Sehr oft erfordern bestimmte Verben aufgrund ihrer Transitivität sowie auch Redewendungen einen bestimmten Nebensatzeinleiter. So brauchen Hauptsätze mit \N{omum} \E{wissen} im Allgemeinen einen Nebensatz im Patiens (üblicherweise \N{futa} oder \N{a fì'ut}).

\subsubsection{} Teilsätze können auch mit Formen von \N{tsa'u} eingeleitet werden. Der Unterschied zwischen \N{fì'u} und \N{tsa'u} besteht darin, dass die Form \N{tsa'u} verwendet werden kann, wenn der Satz, den sie verankert, sich auf alte Information im Diskurs bezieht, d. h. auf etwas, das zuvor thematisiert wurde. Diese Feinheit ist jedoch nicht erforderlich, und die Formen mit \N{fì'u} sind niemals ungrammatisch. 
\index{tsawa@\textbf{tsawa}!Gebrauch}\index{tsata@\textbf{tsata}!Gebrauch}\index{tsaria@\textbf{tsaria}!Gebrauch}\LNWiki{18/6/2010}{https://wiki.learnnavi.org/index.php/Canon/2010/March-June\%23The_contrast_between_fwa.2Ftsawa.2C_furia.2Ftsaria}

Formen mit \N{tsa'u} können auch verwendet werden, wenn der Nebensatz eine kontrastierende Information zum Hauptsatz vermittelt,

\begin{interlin}
	\glll Oel new \uwave{futa} fo kivä, slä sa'nokìl new \uwave{tsata} fo 'ivì'awn. \\
	oe-l new futa ay-po k\INF{iv}ä, slä sa'nok-ìl new tsata ay-po '\INF{iv}ì'awn\\
	\I{1sg}-\I{agt} wollen, dass 3\I{pl}-\I{an} gehen\INF{\I{subj}}, aber Mutter-\I{agt} wollen, dass 3\I{pl}-\I{an} bleiben\INF{\I{subj}} \\
	\trans{Ich will, dass sie gehen, aber Mutter will, dass sie bleiben.}
\end{interlin}
\LNForum{25/2/2022}{https://forum.learnnavi.org/index.php?msg=682691}

\subsubsection{} Das Substantiv \N{tìkin} \E{Bedarf, Bedürfnis, Erfordernis, Notwendigkeit} wird mit einem Attributivsatz für ``müssen'' verwendet, \Npawl{awngaru lu tìkin a nume nì'ul} \E{wir müssen mehr lernen} (wörtl. ``wir haben die Notwendigkeit, mehr zu lernen''). \index{tìkin@\textbf{tìkin}!mit Attributivsatz}
Es kann auch unpersönlich verwendet werden, \N{lu tìkin a \dots} \E{es besteht die Notwendigkeit für\dots}

\subsection{Inhaltssätze mit Adpositionen} Einige Adpositionen können Teilsätze einleiten, wo im Deutschen eine Infinitivkonstruktion mit ``zu'' steht, \N{oe ke tsun stivawm fayfnelì'ut \uwave{luke fwa sngä'i tsngivawvìk}} \E{ich kann solche Worte nicht hören, \uwave{ohne anzufangen zu weinen}}.
\label{syn:rel:nom-adp}
\index{Adposition!mit Inhaltssatz}\LNForum{18/6/2010}{https://forum.learnnavi.org/index.php?msg=240892}

% Mungwrr fwa seen:
%  https://naviteri.org/2020/02/some-words-for-leap-year-day/
% Vat fwa seen:
%  https://naviteri.org/2012/03/spring-vocabulary-part-1/

\subsection{Weitere teilsatzeinleitende Konjunktionen} Einige Na'vi-Konjunktionen bestehen aus einem Substantiv und der attributiven Partikel \N{-a-}. Aufgrund dieser Konstruktion haben diese Konjunktionen zwei Formen -- eine für den Anfang und eine für das Ende des Nebensatzes. Oft sind beide Elemente zu einem Wort kontrahiert, manchmal mit Lautveränderungen.

\begin{center}
	\begin{tabular}{rlll}
		& am Anfang & am Ende \\
		\hline
		\E{nach} & \N{mawkrra} & \N{akrrmaw} & aus \N{maw krr a} \\
		\E{weil} & \N{talun(a)} & \N{alunta} & aus \N{ta lun a} \\
		\E{weil} & \N{taweyk(a)} & \QUAESTIO{\N{aweykta}} & aus \N{ta oeyk a}\\
		\E{wenn} & \N{krra} & \N{a krr} \\
		\E{dass} (als Ergebnis) & \N{kuma} & \N{akum} \\
		\E{seit} (ab dem Zeitpunkt) & \N{takrra} & \N{akrrta} & aus \N{ta krr a}\\
	\end{tabular}
\end{center}\label{syn:attr:takrra}\label{syn:attr:kuma}
\index{mawkrra@\textbf{mawkrra}}\index{akrrmaw@\textbf{akrrmaw}}
\index{talun(a)@\textbf{talun(a)}}\index{alunta@\textbf{alunta}}
\index{taweyk(a)@\textbf{taweyk(a)}}\index{aweykta@\textbf{aweykta}}
\index{takrra@\textbf{takrra}}\index{akrrta@\textbf{akrrta}}
\index{kuma@\textbf{kuma}}\index{akum@\textbf{akum}}
\index{krr@\textbf{krr}!mit attributivem \N{a}}
\NTeri{31/3/2012}{https://naviteri.org/2012/03/spring-vocabulary-part-2/}
\NTeri{19/6/2012}{https://naviteri.org/2012/06/spring-vocabulary-part-3/}

\noindent So kann beispielsweise \Npawl{tì'eyngit oel tolel \uwave{a krr}, ayngaru payeng} \E{\uwave{wenn} ich eine Antwort erhalten habe, werde ich es dich wissen lassen} auch als \N{\uwave{krra} tì'eyngit oel tolel, \dots} formuliert werden.
\LNWiki{1/2/2010}{https://wiki.learnnavi.org/index.php/Canon\%23Some_Conjunctions_and_Adverbs}
\LNWiki{1/2/2010}{https://wiki.learnnavi.org/index.php/Canon\%23Extracts_from_various_emails}
\NTeri{15/8/2011}{https://naviteri.org/2011/08/new-vocabulary-clothing/comment-page-1/\%23comment-986}

\section{Konditionalsätze}
\noindent Konditionalsätze werden im Na'vi mit der Konjunktion \N{txo} \E{wenn} für die Bedingung und optional mit \N{tsakrr} \E{dann} für die Folge eingeleitet.
\label{syn:conditionals}\index{Konditionalsatz}
\index{txo@\textbf{txo}}\index{tsakrr@\textbf{tsakrr}}

\subsection{Allgemein} Allgemeine Bedingungen beschreiben Situationen, die universell wahr sind, wie z. B. ``wenn es nicht regnet, leiden Pflanzen und Tiere''. Im Na'vi \QUAESTIO{wird eine allgemeine Bedingung mit \N{txo} ausgedrückt, wobei in der Bedingung der Subjunktiv und in der Konsequenz ein Indikativ ohne Tempusmarkierung verwendet wird}, \Npawl{txo fkol ke fyivel uranit paywä, zene fko slivele} \E{wenn man ein Boot nicht gegen Wasser abdichtet, muss man schwimmen.}
\index{Konditionalsatz!allgemein}
\NTeri{19/6/2012}{https://naviteri.org/2012/06/spring-vocabulary-part-3/}

\subsection{Zukunft} Im Deutschen verwenden Konditionalsätze, die sich auf die Zukunft beziehen, für die Bedingung das Präsens und für die Folge das Futur I, ``wenn du dieses machst, werde ich jenes machen''. Im Na'vi hingegen steht die Bedingung im Subjunktiv und die Konsequenz in der Zukunft, \Npawl{pxan \uwave{l\INF{iv}u} txo nì'aw oe ngari / Tsakrr nga Na'viru \uwave{yomt\INF{ìy}ìng}} \E{nur wenn ich deiner würdig \uwave{bin} / \uwave{wirst} du das Volk versorgen}. \index{Konditionalsatz!Zukunft}

\subsection{Gebrauch des Subjunktivs} \index{Konditionalsatz!Subjunktiv}
Die Verwendung des Subjunktivs im \N{txo}-Satz hängt vom Status der Bedingung ab. Wenn der Sprecher mit Sicherheit weiß, dass eine Bedingung erfüllt ist, so hat \N{txo} die Bedeutung „da, weil“ und der Subjunktiv ist nicht erforderlich. In Beispiel \ref{txo:subj:ex01} weiß der Sprecher nicht sicher, ob sein Gesprächspartner müde ist:

\begin{interlin} \label{txo:subj:ex01}
	\glll Txo nga ngeyn 'ivefu, tsurokx. \\
	txo nga ngeyn '\INF{iv}efu, tsurokx \\ 
	wenn 2\I{sg} müde \INF{\I{subj}}fühlen, erholen \\
	\trans{Wenn du müde bist, erhole dich.}
\end{interlin}

\noindent Wenn der Sprecher jedoch weiß, dass sein Gesprächspartner müde ist (etwa, weil er es ihm gesagt hat), kann der Subjunktiv entfallen,

\begin{interlin} \label{txo:subj:ex02}
	\glll Txo nga ngeyn 'efu, tsurokx. \\
	txo nga ngeyn 'efu, tsurokx \\
	wenn 2\I{sg} müde fühlen, erholen \\
	\trans{Da du müde bist, erhole dich.}
\end{interlin}

\noindent Die Verwendung des Subjunktivs ist daher recht flexibel, und es wird Situationen geben, in denen beide Verwendungsweisen gerechtfertigt sein können.
\LNForum{14/1/2022}{https://forum.learnnavi.org/index.php?msg=678016}

\subsection{Irrealis} Der Irrealis bzw. Kontrafaktual (d. h. ein Konditionalsatz mit einer Bedingung, deren Erfüllung nicht (mehr) möglich ist, weil sie der Realität widerspricht) wird mit einer eigenen Gruppe von Konjunktionen gebildet, \N{zun} \E{wenn} und \N{zel} \E{dann}. Der Subjunktiv wird in beiden Sätzen verwendet, mit den folgenden Zeitformen:
\index{Konditionalsatz!Irrealis (kontrafaktual)}
\index{zun@\textbf{zun}}\index{zel@\textbf{zel}}

\begin{center}
	\begin{tabular}{lll}
		Vergangenheit & Gegenwart & Zukunft \\
		\hline
		\N{\INF{imv}}, \N{\INF{ilv}} & 
		\N{\INF{iv}}, \N{\INF{irv}} & 
		\N{\INF{ìyev}, \INF{iyev}}
	\end{tabular}
\end{center}

\noindent Für Situationen im Präsens wird der bloße Subjunktiv oder der Subjunktiv Imperfekt verwendet; für vergangene Situationen wird der Subjunktiv Perfekt oder der Subjunktiv der Vergangenheit verwendet; und für zukünftige Situationen wird der Subjunktiv der Zukunft gebraucht (siehe \horenref{morph:verb:first-position} für die Infixformen).\footnote{Anm. d. Ü.: Im Deutschen wird der Irrealis mit ``wenn'' bzw. ``falls'' und dem Konjunktiv II gebildet, z. B. ``wenn du ging(e)st, wäre ich traurig''. Sehr häufig kann im Deutschen ganz auf ``wenn'' und ``dann'' verzichtet werden, ``gingest du, wäre ich traurig''. Alltagssprachlich wird anstelle des Konjunktivs oft auf eine Umschreibung mit ``würde'' zurückgegriffen, ''würdest du gehen, würde ich traurig sein''.}

\begin{quotation}
	\noindent\Npawl{Zun oe yawne livu ngar, zel 'ivefu oe nitram nì'aw.}\\
	\indent\E{Wenn du mich liebtest, wäre ich einfach nur glücklich.}\\
	\noindent\Npawl{Zun oe yawne limvu ngar, zel 'imvefu oe nitram nì'aw.}\\
	\indent\E{Wenn du mich geliebt hättest, wäre ich einfach nur glücklich gewesen.}\\
	\noindent\Npawl{Zun tompa zìyevup trray, zel fo srìyevew.}\\
	\indent\E{Wenn es morgen regnen würde, würden sie tanzen.}\\
	\noindent\Npawl{Zun ayoe livu tsamsiyu, zel tsakem ke simvi.}\\
	\indent\E{Wenn wir Krieger wären, hätten wir das nicht getan.}
\end{quotation}

\noindent Wenn das Tempus beider Sätze dasselbe ist, dann -- und nur dann -- kann das Verb im \N{zel}-Satz unmarkiert stehen,

\begin{quotation}
	\noindent\Npawl{Zun oe yawne livu ngar, zel \uwave{'efu} oe nitram nì'aw.}\\
	\indent\E{Wenn du mich liebtest, wäre ich einfach nur glücklich.}\\
	\noindent\Npawl{Zun oe yawne limvu ngar, zel \uwave{'efu} oe nitram nì'aw.}\\
	\indent\E{Wenn du mich geliebt hättest, wäre ich einfach nur glücklich gewesen.}\\
	\noindent\Npawl{Zun tompa zìyevup trray, zel fo \uwave{srew}.}\\
	\indent\E{Wenn es morgen regnen würde, würden sie tanzen.}
\end{quotation}
\NTeri{4/30/2013}{https://naviteri.org/2013/04/zun-zel-counterfactual-conditionals/}

\subsection{Potentialis} \index{Konditionalsatz!Potentialis}
Konditionalsätze mit einer Bedingung, deren Erfüllung denkbar bzw. möglich ist, werden mit \N{zun} \E{wenn} und \N{zel} \E{dann} zusammen mit einem Subjunktiv ohne Tempusmarkierung gebildet. Die folgenden Beispiele illustrieren den Unterschied zwischen Irrealis und Potentialis:

{\small
	\begin{interlin}\label{hyp:ex01}\small
		\glll Zun ngal oey tsmuket tsive'a, zel am'aluke ivomum futa poe nga'prrnen.\\
		zun nga-l oe-yä tsmuke-t ts\INF{iv}e'a, zel am'aluke \INF{iv}omum futa poe nga'prrnen \\
		wenn \I{2sg}-\I{agt} \I{1sg}-\I{gen} Schwester-\I{pat} sehen\INF{\I{subj}}, dann sicher wissen\INF{\I{subj}}, dass sie schwanger\\
		\trans{Wenn du meine Schwester sähest, wüsstest du sicher, dass sie schwanger ist.} \Ipawl{}
	\end{interlin}
	
	\begin{interlin}\label{hyp:ex02}\small
		\glll Zun ngal oey tsmuket tsimve'a, zel am'aluke imvomum futa poe nga'prrnen.\\
		zun nga-l oe-yä tsmuke-t ts\INF{imv}e'a, zel am'aluke \INF{imv}omum futa poe nga'prrnen \\
		wenn \I{2sg}-\I{agt} \I{1sg}-\I{gen} Schwester-\I{pat} sehen\INF{\I{pst.subj}}, dann sicher wissen\INF{\I{pst.subj}}, dass sie schwanger\\
		\trans{Wenn du meine Schwester gesehen hättest, hättest du sicher gewusst, dass sie schwanger ist.} \Ipawl{}
\end{interlin}}

\noindent Man beachte, dass Beispiel \ref{hyp:ex01} den einfachen Subjunktiv verwendet und als eine hypothetische, aber realistische Vorstellung übersetzt wird. Beispiel \ref{hyp:ex02} ist lediglich die übliche kontrafaktische Form, die in beiden Satzteilen Subjunktive der Vergangenheit verwendet.
\NTeri{28/2/2024}{http://naviteri.org/2024/02/mipa-ayliu-si-aylifyavi-niul-more-new-words-and-expressions/}

\subsection{Imperative in Bedingungen} Wenn ein Imperativ (oder Prohibitiv) die Folge einer Bedingung bildet, setzen die Regeln für den Imperativ die üblichen Konditionalmuster außer Kraft. Ein auf die Zukunft gerichteter Konditionalsatz mit imperativischer Konsequenz wäre etwa \Npawl{txo \uwave{tsive'a} ayngal keyeyt, rutxe oeru \uwave{piveng}} \E{wenn du Fehler siehst, sag es mir bitte}. \index{Befehl!in Konditionalsätzen}

\section{Konjunktionen}
\noindent In diesem Abschnitt werden Konjunktionen erläutert, die an anderer Stelle noch nicht behandelt wurden, die aber dennoch Erwähnung verdienen. Konjunktionen, die keinen besonderen Kommentar erfordern, werden hier nicht aufgeführt.

\subsection{Alu} N{alu} wird in erster Linie für Substantive in einer Apposition verwendet, \Npawl{tskalepit oel tolìng oeyä \uwave{tsmukanur alu Ìstaw}} \E{ich gab den Bogen \uwave{meinem Bruder Ìstaw}}. Man beachte, dass das Substantiv nach \N{alu} im Absolutiv steht, nicht im Dativ wie \N{tsmukanur}.
\NTeri{16/7/2010}{https://naviteri.org/2010/07/vocabulary-update/}
\index{alu@\textbf{alu}}\label{syn:conj:alu}\index{Apposition}

\subsubsection{} \N{Alu} kann im Diskurs auch dazu verwendet werden, eine Neuformulierung zu markieren, wie im Deutschen ``das soll heißen'' oder ``mit anderen Worten'', \Npawl{txoa livu, yawne lu oer Sorewn...\ \uwave{alu}...\ ke tsun oeng muntxa slivu} \E{Entschuldigung, aber ich liebe Sorewn... \uwave{soll heißen}, du und ich können nicht heiraten.}

\subsubsection{} In Konversationen über Grammatik und Sprache kann \N{alu} das Wort oder die Konstruktion verdeutlichen, über die gesprochen wird, \Npawl{tsalsungay tsalì’u \uwave{alu zeykuso} lu eyawr} \E{dennoch ist das Wort `\uwave{\N{zeykuso}'} richtig}, \Npawl{lì’uri \uwave{alu tskxe} pamrel fyape?} \E{wie buchstabiert man das Wort \uwave{`\N{tskxe}'}?}.

\subsection{Ftxey} Zusätzlich zur Konstruktion von Entscheidungsfragen mit (\N{ftxey... fuke}, \horenref{syn:question:ftxey}), kann \N{ftxey} benutzt werden, um Möglichkeiten aufzuzählen, \Npawl{sìlpey oe, \dots\ frapo --- \uwave{ftxey sngä'iyu ftxey tsulfätu} -- tsìyevun fìtsenge rivun 'uot lesar} \E{ich hoffe, ... jeder -- \uwave{ob Anfänger order Experte}, --- kann hier etwas Nützliches finden}.
\index{ftxey@\textbf{ftxey}}

\subsection{Fu} Die Konjunktion \N{fu} \E{oder} kann verwendet werden, um Nominal- oder Verbalphrasen miteinander zu verbinden, \Npawl{ke zasyup lì'Ona ne kxutu a mìfa fu a wrrpa} \E{die lì’Ona werden weder durch den inneren noch durch den äußeren Feind untergehen}, \Npawl{rä'ä fmivi livok fu emkivä ayekxanit a fkol ngolop fpi sìkxuke ayfrrtuä sì ayioangä} \E{versuche nicht, dich Absperrungen zu nähern oder sie zu überqueren, die der Sicherheit von Gästen und Tieren dienen}.
\index{fu@\textbf{fu}}

\subsubsection{} Bei einer Aussage mit Wahlmöglichkeit wird \N{fu} einmal verwendet, \Npawl{nulnew oel fì'ut \uwave{fu} tsa'ut} \E{ich möchte dieses \uwave{oder} jenes}. Bei einer Alternativfrage wird \N{fu} vor jede Wahlmöglichkeit gestellt, \Npawl{nulnew ngal \uwave{fu} fì'ut \uwave{fu} tsa'ut?} \E{willst du lieber dieses oder jenes?}.

\subsection{Ki} Die Konjunktion \N{ki} \E{sondern} wird stets zusammen mit dem negativen Adverb \N{ke} verwendet und ist von \N{slä} \E{aber} zu unterscheiden, \Npawl{nga plltxe ke nìfyeyntu ki nì'eveng} \E{du sprichst nicht wie ein Erwachsener, sondern wie ein Kind.}
\NTeri{16/7/2010}{https://naviteri.org/2010/07/vocabulary-update/}
\index{ki@\textbf{ki}}

\subsection{Sì} Die Konjunktion \N{sì} \E{und} wird verwendet, um Auflistungen zu formulieren und Elemente desselben Konzeptes zu verbinden. Sie wird nicht verwendet, um Teilsätze zu verbinden, denn das ist die Aufgabe von \N{ulte}, \horenref{syn:ulte}. \Npawl{Lu pìlokur pxesìkan sì pxefne’upxare} \E{der Blog hat drei Ziele und drei Arten von Nachrichten}, \Npawl{ma smukan sì smuke} \E{Brüder und Schwestern}.\index{siì@\textbf{sì}}\label{syn:sì}

\subsubsection{} Obwohl \N{sì} am häufigsten in Verbindung mit Substantiven, Pronomen und Adjektiven gebraucht wird, kann es auch Verben verbinden, die eng miteinander in Beziehung stehen, \Nfilm{sänume sivi poru fte \uwave{pivlltxe sì tivìran} nìayoeng} \E{lehre ihn, wie wir \uwave{zu sprechen und zu gehen}}.

\subsubsection{} Mit \N{fwa}, \N{futa} usw. eingeleitete Nebensätze (\horenref{syn:clause-nom}) können ebenfalls Element einer mit \N{sì} verbundenen Liste von Substantiven sein, wie in \N{sunu poru syulang sì mauti sì fwa tsway\-on yaka} \E{er mag Blumen, Früchte und durch die Luft zu fliegen}.
\LNWiki{23/1/2018}{https://wiki.learnnavi.org/Canon/2018}

\subsubsection{} \N{Sì} kann auch enklitisch stehen (\horenref{lands:stress:enclisis}). In diesem Fall wird es an das Wort oder Satzglied, das es der Liste hinzufügt, angehängt, \Npawl{ta 'eylan \uwave{karyusì} ayngeyä, Pawl} \E{von deinem Freund \uwave{und Lehrer}, Paul}, \Npawl{tsakrr paye'un sweya fya'ot a zamivunge oel ayngar aylì'ut \uwave{horentisì} lì'fyayä leNa'vi} \E{und ich werde dann entscheiden, welches der beste Weg ist, euch die Wörter \uwave{und Regeln} des Na'vi zu überbringen}. \index{siì@\textbf{sì}!enklitisch}

%\subsection{Tengfya} \QUAESTIO{Needed?}

\subsection{Tengkrr} Die Bedeutung \N{tengkrr} \E{während, solange, zur gleichen Zeit wie} macht normalerweise den Gebrauch des Imperfekts erforderlich, \Npawl{fìtxon yom \uwave{tengkrr teruvon}} \E{diese Nacht essen wir, \uwave{während wir uns anlehnen}}.
\index{tengkrr@\textbf{tengkrr}}
\LNWiki{14/3/2010}{https://wiki.learnnavi.org/index.php/Canon/2010/March-June\%23A_Collection}

\subsection{Tìk} \index{tìk@\textbf{tìk}} \label{syn:tìk}
Dieses Adverb mit der Bedeutung \E{sofort, schleunigst, unmittelbar, augenblicklich, auf der Stelle} kann auch als Konjunktion verwendet werden, die anzeigt, dass eine zweite Handlung unmittelbar auf eine erste folgt, \Npawl{fìioang ke tsun slivele; nemfapay zup tìk spakat} \E{dieses Tier kann nicht schwimmen; es fällt ins Wasser (und es) ertrinkt}.
\NTeri{31/12/2021}{https://naviteri.org/2021/12/zolau-niprrte-ma-3746-welcome-2022/}

\subsection{Tsnì} \label{syn:tsni}\index{tsnì@\textbf{tsnì}}
Die Konjunktion \N{tsnì} \E{dass} leitet Nebensätze bei gewissen Verben im Hauptsatz ein, \N{ätxäle si tsnì livu oheru Uniltaron} \E{ich bitte höflichst um die Traumjagd}, \Npawl{sìlpey oe tsnì fìtìoeyktìng law livu ngaru set} \E{ich hoffe, dass diese Erklärung für dich jetzt klar ist}.
\NTeri{20/2/2011}{https://naviteri.org/2011/02/new-vocabulary-part-2/}

\subsubsection{} Verben, von denen bekannt ist, dass sie \N{tsnì} nehmen: \N{ätxäle si}, \N{rangal} (marginal verwendet, da der Gebrauch von \N{nìrangal} üblicher ist), \N{sìlpey}, \N{la'um}, \N{mowar si}, \N{fe'pey}, \N{leymfe'}, \N{leymkem}, \N{srefey} und \N{srefpìl}. Einige Verben, wie \N{sìlpey} \E{hoffen}, erfordern den Subjunktiv im \N{tsnì}-Nebensatz, während andere wie \N{la'um} \E{vortäuschen, vorspielen, heucheln} ohne Subjunktiv stehen. Das Wörterbuch ist der geeignete Ort, um dies im Einzelfall zu überprüfen.
\LNForum{18/8/2011}{https://forum.learnnavi.org/index.php?msg=487559/}
\NTeri{1/3/2020}{https://naviteri.org/2020/02/some-words-for-leap-year-day/\#comment-30895}

\subsection{Ulte} Diese Konjunktion verbindet Teilsätze, \N{oel ngati kameie, ma tsmukan, \uwave{ulte} ngaru seiyi irayo} \E{ich sehe dich, Bruder, \uwave{und} danke dir}. Nicht mit \N{sì} zu verwechseln  (\horenref{syn:sì}).
\index{ulte@\textbf{ulte}}\label{syn:ulte}

\section{Zitation}
\label{syn:direct-quote}

\subsection{San... sìk} Na'vi kennt keine indirekte Rede (anders als z. B. das Deutsche, wie in ``er sagte, \uwave{dass er gegangen sei}''), sondern gibt stets den Wortlaut des Gesagten in direkter Rede wieder, wobei das Zitat zwischen den Partikeln \N{san} und \N{sìk} steht, wie in \Npawl{slä nì'i'a tsun oe pivlltxe \uwave{san Zola'u nìprrte' ne pìlok Na'viteri sìk}!} \E{aber jetzt kann ich endlich sagen: ``\uwave{Willkommen auf dem Blog Na'viteri}!''}. \index{Zitation!direkt}
\index{san@\textbf{san}}\index{siìk@\textbf{sìk}}\NTeri{31/8/2011}{https://naviteri.org/2011/08/reported-speech-reported-questions/}

\subsubsection{} Wenn der Anfang oder das Ende eines Zitats mit dem Anfang oder dem Ende der gesamten Äußerung zusammenfällt, kann entsprechend \N{san} oder \N{sìk} entfallen:

\begin{quotation}
	\noindent 1. \Npawl{Poltxe Eytukan \uwave{san} oe kayä \uwave{sìk}, slä oel pot ke spaw.}\\
	\indent\E{Eytukan sagte: ``ich werde gehen'', aber ich glaube ihm nicht.}\\
	\noindent 2. \N{Poltxe Eytukan \uwave{san} oe kayä.}\\
	\indent\E{Eytukan sagte: ``ich werde gehen''.}
\end{quotation}

\noindent Da im zweiten Satz nach dem Zitat nichts mehr gesagt wird, muss das Zitat nicht mit \N{sìk} abgeschlossen werden. Ebenso kann \N{san} weggelassen werden, sofern dies nicht zu Mehrdeutigkeiten führt, \Npawl{frawzo sìk, slä oel poet ke spaw} \E{(sie sagte) ``alles gut'', aber ich glaube ihr nicht}. 
\LNWiki{21/1/2010}{https://wiki.learnnavi.org/index.php/Canon\%23Extracts_from_various_emails}
\LNForum{4/8/2020}{https://forum.learnnavi.org/index.php?msg=672419}

\subsection{Fragen} Indirekte Fragen werden ebenfalls direkt zitiert, \Npawl{polawm po san srake Säli holum sìk} \E{er fragte, ob Sally gegangen ist} (wörtl. \E{er fragte: ``ist Sally gegangen?''}).
\LNWiki{24/3/2010}{https://wiki.learnnavi.org/index.php/Canon/2010/March-June\%23If_and_Whether}

\subsubsection{} Bei \N{pawm}, aber nicht bei anderen Verben des Sprechens, kann \N{san... sìk} weggelassen werden, \Npawl{polawm po, Neytiri kä pesengne?} \E{er hat gefragt, wohin Neytiri geht} (wörtl. \E{er hat gefragt: ``wohin geht Neytiri?''}).
\NTeri{31/8/2011}{https://naviteri.org/2011/08/reported-speech-reported-questions/}

\subsection{Transitivität} Wenn ein Verb des Sprechens \N{san... sìk} verwendet, ist der Satz intransitiv, \N{po poltxe san srane} \E{sie sagte ``ja''.} \index{Transitivität!mit Verben des Sprechens}
\Ultxa{2/10/2010}{https://wiki.learnnavi.org/index.php/Canon/2010/UltxaAyharyu\%C3\%A4\%23Transitivity_with_Speaking_Verbs}

\subsubsection{} Wenn das Verb des Sprechens ein direktes Objekt hat, ist der Satz transitiv, \Npawl{ke poltxe pol tsaylì'ut} \E{sie sagte das nicht}, \N{oel poru pasyawm tsat} \E{ich werde ihn das fragen}.
\NTeri{31/8/2011}{https://naviteri.org/2011/08/reported-speech-reported-questions/}

\subsection{Zitat im Nebensatz} Neben dem \N{san... sìk}-Paar kann eine Äußerung auch mit den Substantiven \N{fmawn} \E{Neuigkeit}, \N{tì'eyng} \E{Antwort} und \N{faylì'u} \E{diese Worte} und der attributiven Partikel \N{a} eingeleitet werden (siehe \horenref{morph:fmawn} für die Verschmelzungen). Die folgende Tabelle gibt einen Überblick darüber, welche Verben des Sprechens mit welchen Zitationswörtern gebraucht werden:
\NTeri{31/8/2011}{https://naviteri.org/2011/08/reported-speech-reported-questions/}

\begin{center}
	\begin{tabular}{ll}
		Verb & Zitat \\
		\hline
		\N{plltxe} \E{sagen} & \N{san... sìk}, \N{faylì'u} \\
		\N{stawm} \E{hören}, \N{peng} \E{erzählen, berichten} & \N{fmawn} \\
		\N{pawm} \E{fragen} & \N{san... sìk}, \N{tì'eyng}, nichts \\
		\N{vin} \E{anfragen, bitten} & \N{tì'eyng} 
	\end{tabular}
\end{center}
\index{pawm@\textbf{pawm}}\index{stawm@\textbf{stawm}}
\index{plltxe@\textbf{plltxe}}\index{vin@\textbf{vin}}

\noindent Die Zitate, die mit diesen Wörtern eingeleitet werden, stehen nach wie vor in der direkten Form,

\begin{quotation}
	\noindent\Npawl{Poltxe pol fayluta oe new kivä.} \\ 
	\indent\E{Sie sagte, sie wollte gehen}. (Wörtl. \E{Sie sagte: ``Ich will gehen.''})\\
	\noindent\Npawl{Ngal poleng oer fmawnta po tolerkup.} \\
	\indent\E{Du hast mir gesagt, dass er gestorben sei.} (Wörtl. \E{Du hast mir gesagt: ``Er ist gestorben.''})\\
	\noindent\N{Volin pol tì’eyngit a Neytiri kä pesengne.} \\
	\indent\E{Er hat gefragt, wohin Neytiri geht.} (Wörtl. \E{Er hat gefragt: ``Wohin geht Neytiri?''})
\end{quotation}
\label{syn:quot:nominalized}

\subsubsection{} Andere Verben, die indirekte Fragen einleiten, können \N{tì'eyng} verwenden:

\begin{quotation}
	\noindent\Npawl{Ke omum oel teyngta fo kä pesengne.} \E{Ich weiß nicht, wohin sie gehen.} \\
	\noindent\Npawl{Teynga lumpe fo holum ke lu law.} \E{Es ist nicht klar, warum sie gegangen sind.}
\end{quotation}

\subsubsection{} Die Wiedergabe von Befehlen (mit dem Verb \N{kxìm} \E{befehlen, anordnen, verlangen, anweisen}) wird mit \N{tsonta}, einer verkürzten Form von \N{tsonit a} \E{die Pflicht, welche}, eingeleitet,

\begin{interlin}
	\glll Ayevengur kxolìm sa'nokìl tsonta payit zamunge. \\
	ay-eveng-ur kx\INF{ol}ìm sa'nok-ìl tsonta pay-it zamunge \\
	\I{pl}-Kind-\I{dat} anweisen\INF{\I{pfv}} Mutter-\I{agt} Pflicht.welche Wasser-\I{pat} bringen \\
	\trans{Die Kinder wurden von ihrer Mutter angewiesen, Wasser zu holen.} \Ipawl{}
\end{interlin}

\noindent\NTeri{2/10/2014}{http://naviteri.org/2014/10/tson-si-fpomron-obligation-and-mental-health/}

\section{Partikel}

\subsection{Ko} Die Partikel \N{ko} wird verwendet, um Zustimmung verschiedener Art einzufordern, wo im Deutschen etwa ``lasst uns'', ``meinst du nicht?'', ``warum machst du nicht?, warum mache ich nicht?'' formuliert wird. Sie steht immer am Ende eines Satzes. Im Film häufig gesagt wird etwa \Nfilm{makto ko} \E{lasst uns reiten}.
\index{ko@\textbf{ko}}\label{syn:particle:ko}

\subsubsection{} Im besonderen Fall von \N{siva ko} \E{stelle dich der Herausforderung!}, kann der Satz als ein einziges Wort geschrieben werden, \N{sivako}.
\NTeri{3/8/2019}{https://naviteri.org/2016/06/mrrvola-lifyavi-amip-forty-new-expressions/comment-page-1/\%23comment-30185}

\subsection{Nang} Diese Partikel kennzeichnet Überraschung, Exklamation oder Ermutigung. Sie steht immer am Satzende und zusammen mit Adverbien des Maßes oder der Zustimmung, wie etwa \N{nìngay}, \N{nìtxan}, \N{fìtxan}, usw. \N{Txantsana sìpawm apxay fìtxan lu ngaru nang!} \E{du hast so viele ausgezeichnete Fragen!}, \Npawl{ngari tswintsyìp sevin nìtxan lu nang!} \E{was du für ein schönes Zöpfchen hast!}
\index{nang@\textbf{nang}}

\subsection{Pak} \index{pak@\textbf{pak}}
Diese Partikel folgt dem Wort oder dem Satz, mit dem sie inhaltlich verbunden ist, und kennzeichnet Verunglimpfung, \Nfilm{tsamsiyu pak!} \E{ein Krieger? Schon klar!}.

\begin{interlin}
	\gll Poan pak?! Ke lu po tsamsiyu kaw'it! \\
	\I{3sg.masc} \I{dispar}! nicht sein \I{3sg} Krieger überhaupt.nicht \\
	\trans{Er?! Er hat keinen einzigen Knochen eines Kriegers im ganzen Körper!}
\end{interlin}

\begin{interlin} \label{pak:ex02}
	\gll Tsaw ke ley kaw'it pak! \\
	das nicht wert überhaupt.nicht \I{dispar} \\
	\trans{Das ist überhaupt nichts wert!} \Ipawl{}
\end{interlin}

\noindent Normalerweise gehört \N{pak} zu einem einzelnen Substantiv oder Pronomen, das verachtet wird, aber es kann auch in einer loseren Weise verwendet werden, um eine ganze Idee oder Situation zu verunglimpfen, wie in Beispiel \ref{pak:ex02}. Die Partikel steht immer am Ende des Satzes.

Das negative Affektinfix \N{\INF{äng}} zusammen mit \N{pak} zu verwenden, ist stilistisch ungünstig, da \N{pak} allein ausreicht, um die Unzufriedenheit anzuzeigen. 
\NTeri{31/3/2014}{https://naviteri.org/2014/03/value-and-worth/}
\LNForum{6/8/2010}{https://forum.learnnavi.org/index.php?msg=174572}
\LNForum{16/6/2023}{https://forum.learnnavi.org/index.php?msg=686933}

\subsection{Tut} Hierbei handelt es sich um eine Diskurspartikel, die bisher nur bei ``zurückgeworfenen'' Fragen in Dialogen zu beobachten ist: 

\begin{quotation}
	\noindent A: \N{Ngaru lu fpom srak?} \E{Geht es dir gut?} \\
	\noindent B: \N{Oeru lu fpom.  \uwave{Ngaru tut?}} \E{Mir geht es gut. \uwave{(Und) dir?}}
\end{quotation}
\index{tut@\textbf{tut}}

\subsection{Tse} Diese Rezeptionspartikel zeigt ein Zögern in einer Unterhaltung an, \E{also, nun ja}. \QUAESTIO{Im Englischen wird ``well'' in unterschiedlicher Weise mit Glück in Verbindung gebracht.}
\index{tse@\textbf{tse}}

\section{Andere erwähnenswerte Wörter}

\subsection{Sweylu} \label{syn:sweylu} Die Syntax dieses Verbs mit der Bedeutung ``sollte'' (von \N{swey lu} \E{es ist am besten}) ändert sich je nachdem, ob sich die Verpflichtung auf etwas bezieht, das noch nicht geschehen ist, oder ob sie sich auf ein Ereignis bezieht, das bereits stattgefunden hat.
\index{sweylu@\textbf{sweylu}}

\subsubsection{} Mit Bezug auf die Zukunft wird \N{sweylu} zusammen mit \N{txo} und dem Subjunktiv verwendet, \Npawl{sweylu txo nga kivä} oder \N{nga sweylu txo kivä} \E{du solltest gehen}. Man beachte, dass die Verneinung im \N{txo}-Satz steht, \Npawl{sweylu txo nga ke kivä} oder \N{nga sweylu txo ke kivä} \E{du solltest nicht gehen}.

\subsubsection{} Mit Bezug auf die Vergangenheit wird \N{sweylu} zusammen mit \N{fwa} oder \N{tsawa} und dem Indikativ der Vergangenheit oder des Perfekts verwendet,

\begin{quotation}
	\noindent Tsenu: \Npawl{Spaw oe, fwa po kolä längu kxeyey.} \\
	\indent\E{Ich glaube, es war ein Fehler von ihm, zu gehen.} \\
	
	\noindent Kamun: \N{Kehe, kehe! Sweylu fwa po kolä!}\\
	\indent\E{Nein, nein! Es war besser, dass er gegangen ist!}
\end{quotation}

\noindent Man beachte, dass sich dies auf eine vergangene Handlung bezieht, die tatsächlich stattgefunden hat und richtig war, und nicht auf einen unerfüllten Wunsch in der Vergangenheit (was eine andere Verwendung von ``sollte'' im Deutschen ist).
\NTeri{5/4/2011}{https://naviteri.org/2011/04/\%e2\%80\%99a\%e2\%80\%99awa-li\%e2\%80\%99fyavi-amip\%e2\%80\%94a-few-new-expressions/}